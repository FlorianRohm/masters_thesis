\documentclass[]{article}
\usepackage{amsmath}
\usepackage{bm}
\usepackage{multicol}

\title{Abstract of ``Generalized Lattice-Boltzmann Equations''}
\author{Florian Rohm}
\begin{document}

\maketitle

\begin{abstract}
Notes while reading the paper
\end{abstract}

\section{Arbitrary notes}

Lattice boltzmann equation:
\[
N_i(\vec{r}_{*} + \vec{c}_i, t_{*} + 1) = N_i (\vec{r}_{*}, t_{*}) + \sum_j \Lambda_{ij} [N_j(\vec{r}_{*}, t_{*}) - N_j^{\text{eq}} (\vec{r}_{*}, t_{*})] \quad i = 0, \dots, b
\]
\begin{enumerate}
  \item Conserved quantity: an eigenvector of $\Lambda$
  \item $N_j^{\text{eq}}$ is a function of conserved quantities
  \item bold vectors are b dimensional vectors in phase-space
\end{enumerate}

\paragraph{Alternative representation}
\label{par:Alternative representation}

\begin{enumerate}
  \item choose b orthonormal vectors $e_k$ in the $N_i$ space
  \item choose a collision operator $\hat{\Lambda}$ which is diagonal in this basis
  \item $\mathcal{T}$ is the transformation to the new basis
  \item thus, $\boldsymbol{\hat{N}} = \mathcal{T} \boldsymbol{N}$
\end{enumerate}

\subsection{Chapman-Enskog}
\label{sub:Chapman-Enskog}
Introduce a pertubation length $\epsilon^{-1}$.
Then expand the distribution $\boldsymbol{N}$ around $\boldsymbol{N}^{(0)} = \boldsymbol{N}^{(\text{eq})}$ on two time scales:

\noindent\begin{minipage}{.5\linewidth}
  \begin{align*}
    \boldsymbol{N} &= \boldsymbol{N}^{(0)} + \epsilon \boldsymbol{N}^{(1)}+ \epsilon^2 \boldsymbol{N}^{(2)} \\
    \partial_t &= \epsilon\partial_{t_1} + \epsilon^2\partial_{t_2}
  \end{align*}
\end{minipage}
\noindent\begin{minipage}{.5\linewidth}
  \begin{align*}
      \boldsymbol{\hat{N}} &= \boldsymbol{\hat{N}}^{(0)} + \epsilon\boldsymbol{\hat{N}}^{(1)}+ \epsilon^2 \boldsymbol{\hat{N}}^{(2)} \\
      \partial_\alpha &= \epsilon\partial_{\alpha'}
  \end{align*}
\end{minipage}

\subsubsection{Taylor Expansion}
\label{subs:Taylor Expansion}
Let $N_i = N_i(\vec{r}_{*}, t_{*})$.
\begin{align*}
  N_i(\vec{r}_{*} + \vec{c}_i, t_{*} + 1) &= \\
  N_i + \partial_t N_i &+ c_{i\alpha}\partial_\alpha N_i + \frac{1}{2} [\partial_t\partial_t N_i + 2c_{i\alpha}\partial_t\partial_\alpha N_i + c_{i\alpha}c_{i\beta}\partial_\alpha\partial_\beta N_i] + \mathcal{O}(\partial^3 N_i)
\end{align*}


\end{document}
