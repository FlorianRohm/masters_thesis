\documentclass[]{article}
\usepackage{amsmath}
\usepackage{bm}
\usepackage{multicol}
\usepackage[toc,page]{appendix}
\usepackage{xcolor}

\newcommand{\highlight}[1]{%
  \colorbox{red!20}{$\displaystyle#1$}}

\title{Abstract of ``Generalized Lattice-Boltzmann Equations''}
\author{Florian Rohm}
\begin{document}

\maketitle

\begin{abstract}
Notes while reading the paper
\end{abstract}

\section{Arbitrary notes}

Lattice boltzmann equation:
\begin{equation}
	\label{eq:latticeBoltzmann}
	N_i(\vec{r}_{*} + \vec{c}_i, t_{*} + 1) = N_i (\vec{r}_{*}, t_{*}) - \sum_j \Lambda_{ij} [N_j(\vec{r}_{*}, t_{*}) - N_j^{\text{eq}} (\vec{r}_{*}, t_{*})], \quad i = 0, \dots, b
\end{equation}
\begin{enumerate}
  \item Conserved quantity: an eigenvector of $\Lambda$ for the zero eigenvalue
  \item $N_j^{\text{eq}}$ is a function of conserved quantities
  \item bold vectors are b dimensional vectors in phase-space
\end{enumerate}

\paragraph{Alternative representation}
\label{par:Alternative representation}

\begin{enumerate}
  \item choose b orthonormal vectors $e_k$ in the $N_i$ space
  \item choose a collision operator $\hat{\Lambda}$ which is diagonal in this basis
  \item $\mathcal{T}$ is the transformation to the new basis
  \item thus, $\boldsymbol{\hat{N}} = \mathcal{T} \boldsymbol{N}$
\end{enumerate}
\label{sub:Chapman-Enskog}
Introduce a pertubation length $\epsilon^{-1}$.
Then expand the distribution $\boldsymbol{N}$ around $\boldsymbol{N}^{(0)} = \boldsymbol{N}^{(\text{eq})}$ on two time scales:

\begin{align}
  \boldsymbol{N} &= \boldsymbol{N}^{(0)} + \epsilon \boldsymbol{N}^{(1)}+ \epsilon^2 \boldsymbol{N}^{(2)}    \label{eq:expandN} \\
  \partial_t &= \epsilon\partial_{t_1} + \epsilon^2\partial_{t_2} \label{eq:expandPartialt}\\
	\partial_\alpha &= \epsilon\partial_{\alpha'} \label{eq:expandPartialAlpha}
\end{align}

\paragraph{Taylor Expansion}
\label{par:Taylor Expansion}
Let $N_i = N_i(\vec{r}_{*}, t_{*})$.
\begin{equation}
  \begin{aligned}
    \label{eq:taylor}
    N_i(\vec{r}_{*} + \vec{c}_i, t_{*} + 1) &= \\
    N_i + \partial_t N_i &+ c_{i\alpha}\partial_\alpha N_i + \frac{1}{2} [\partial_t\partial_t N_i + 2c_{i\alpha}\partial_t\partial_\alpha N_i + c_{i\alpha}c_{i\beta}\partial_\alpha\partial_\beta N_i] + \mathcal{O}(\partial^3 N_i)
  \end{aligned}
\end{equation}

\paragraph{Piecing together}
\label{par:Piecing together}

With some calculations, see~\ref{sec:Derivation of Chapman Enskog}, this yields with $S_\alpha = \text{diag}(c_{1\alpha}, \dots, c_{b\alpha})$
\begin{equation}
	\begin{aligned}
		\label{eq:First order Chapman Enskog as Vector}
		\partial_{t_1} \boldsymbol{N}^{(0)} + S_{\alpha}\partial_{\alpha'} \boldsymbol{N}^{(0)} = -  \Lambda \boldsymbol{N}^{(1)}
	\end{aligned}
\end{equation}
and
\begin{equation}
	\label{eq:Second order Chapman Enskog as Vector}
	\begin{aligned}
		& \partial_{t_2} \boldsymbol{N}^{(0)} + \partial_{t_1} \boldsymbol{N}^{(1)} + S_\alpha \partial_{\alpha'} \boldsymbol{N}^{(1)} + \frac{1}{2} \partial_{t_1}\partial_{t_1} \boldsymbol{N}^{(0)} +  S_\alpha
		 \partial_{\alpha'}\partial_{t_1} \boldsymbol{N}^{(0)} \\
		+ &\; \frac{1}{2} S_\alpha S_\beta\partial_{\alpha'}\partial_{\beta'} \boldsymbol{N}^{(0)} \\
		= &\; - \Lambda  \boldsymbol{N}^{(2)}
	\end{aligned}
\end{equation}
with the alternative representation of~\eqref{eq:Second order Chapman Enskog as Vector} using~\eqref{eq:First order Chapman Enskog as Vector}

\begin{equation}
	\begin{aligned}
		\partial_{t_2} \boldsymbol{N}^{(0)} + (\mathcal{I}-\frac{\Lambda}{2} )\partial_{t_1} \boldsymbol{N}^{(1)} + S_\alpha  (\mathcal{I}-\frac{\Lambda}{2} ) \partial_{\alpha'}  \boldsymbol{N}^{(1)}
		= - \Lambda  \boldsymbol{N}^{(2)}
	\end{aligned}
\end{equation}


\newpage
\begin{appendices}
 \section{Derivation of Chapman Enskog}
\label{sec:Derivation of Chapman Enskog}

Inserting~\eqref{eq:taylor} into~\eqref{eq:latticeBoltzmann} yields
\begin{equation*}
	\begin{aligned}
		&N_i + \partial_t N_i + c_{i\alpha}\partial_\alpha N_i + \frac{1}{2} [\partial_t\partial_t N_i + 2c_{i\alpha}\partial_t\partial_\alpha N_i + c_{i\alpha}c_{i\beta}\partial_\alpha\partial_\beta N_i] + \mathcal{O}(\partial^3 N_i) = \\ &N_i  - \sum_j \Lambda_{ij} [N_j - N_j^{\text{eq}} ], \quad i = 0, \dots, b
	\end{aligned}
\end{equation*}
which is, using~\eqref{eq:expandPartialt} and~\eqref{eq:expandPartialAlpha} equivalent to
\begin{equation*}
	\begin{aligned}
		& (\epsilon\partial_{t_1} + \epsilon^2\partial_{t_2}) N_i \\
		+ &\; \epsilon c_{i\alpha}\partial_{\alpha'} N_i \\
		+ &\; \frac{1}{2} [(\epsilon\partial_{t_1} + \epsilon^2\partial_{t_2})(\epsilon\partial_{t_1} + \epsilon^2\partial_{t_2}) N_i \\
		+ &\; 2 \epsilon c_{i\alpha}(\epsilon\partial_{t_1} + \epsilon^2\partial_{t_2}) \partial_{\alpha'} N_i \\
		+ &\; \epsilon^2 c_{i\alpha}c_{i\beta}\partial_{\alpha'}\partial_{\beta'} N_i]\\
		= &\; - \sum_j \Lambda_{ij} [N_j - N_j^{\text{eq}} ], \quad i = 0, \dots, b
	\end{aligned}
\end{equation*}

and using~\eqref{eq:expandN} with $\boldsymbol{N}^{(0)} = \boldsymbol{N}^{(\text{eq})}$

\begin{equation*}
 \begin{aligned}
	 & (\epsilon\partial_{t_1} + \epsilon^2\partial_{t_2}) (N_i^{(0)} + \epsilon N_i^{(1)}+ \epsilon^2 N_i^{(2)}) \\
	 + &\; \epsilon c_{i\alpha}\partial_{\alpha'} (N_i^{(0)} + \epsilon N_i^{(1)}+ \epsilon^2 N_i^{(2)}) \\
	 + &\; \frac{1}{2} [(\epsilon\partial_{t_1} + \epsilon^2\partial_{t_2})(\epsilon\partial_{t_1} + \epsilon^2\partial_{t_2}) (N_i^{(0)} + \epsilon N_i^{(1)}+ \epsilon^2 N_i^{(2)}) \\
	 + &\; 2 \epsilon c_{i\alpha}(\epsilon\partial_{t_1} + \epsilon^2\partial_{t_2}) \partial_{\alpha'} (N_i^{(0)} + \epsilon N_i^{(1)}+ \epsilon^2 N_i^{(2)}) \\
	 + &\; \epsilon^2 c_{i\alpha}c_{i\beta}\partial_{\alpha'}\partial_{\beta'} (N_i^{(0)} + \epsilon N_i^{(1)}+ \epsilon^2 N_i^{(2)})]\\
	 = &\; - \sum_j \Lambda_{ij} [ \epsilon N_j^{(1)}+ \epsilon^2 N_j^{(2)} ], \quad i = 0, \dots, b
 \end{aligned}
\end{equation*}

Now, collect the terms first order in $\epsilon$

\begin{equation*}
	\begin{aligned}
		\partial_{t_1} N_i^{(0)} + c_{i\alpha}\partial_{\alpha'} N_i^{(0)} = - \sum_j \Lambda_{ij}  N_j^{(1)}
	\end{aligned}
\end{equation*}
or, with $S_\alpha = \text{diag}(c_{1\alpha}, \dots, c_{b\alpha})$ as a vector
\begin{equation*}
	\begin{aligned}
		\partial_{t_1} \boldsymbol{N}^{(0)} + S_{\alpha}\partial_{\alpha'} \boldsymbol{N}^{(0)} = -  \Lambda \boldsymbol{N}^{(1)}
	\end{aligned}
\end{equation*}
which is~\eqref{eq:First order Chapman Enskog as Vector}

And second order in $\epsilon$
\begin{equation*}
	\begin{aligned}
		& \partial_{t_2} N_i^{(0)} + \partial_{t_1} N_i^{(1)} + c_{i\alpha}\partial_{\alpha'} N_i^{(1)} + \frac{1}{2} [\partial_{t_1}\partial_{t_1} N_i^{(0)} + 2 c_{i\alpha}\partial_{\alpha'}\partial_{t_1} N_i^{(0)} \\
		+ &\; c_{i\alpha}c_{i\beta}\partial_{\alpha'}\partial_{\beta'} N_i^{(0)}] \\
		= &\; - \sum_j \Lambda_{ij}  N_j^{(2)}
	\end{aligned}
\end{equation*}
or again as vector
\begin{equation*}
	\begin{aligned}
		& \partial_{t_2} \boldsymbol{N}^{(0)} + \partial_{t_1} \boldsymbol{N}^{(1)} + S_\alpha \partial_{\alpha'} \boldsymbol{N}^{(1)} + \frac{1}{2} \partial_{t_1}\partial_{t_1} \boldsymbol{N}^{(0)} +  S_\alpha
		 \partial_{\alpha'}\partial_{t_1} \boldsymbol{N}^{(0)} \\
		+ &\; \frac{1}{2} S_\alpha S_\beta\partial_{\alpha'}\partial_{\beta'} \boldsymbol{N}^{(0)} \\
		= &\; - \Lambda  \boldsymbol{N}^{(2)}
	\end{aligned}
\end{equation*}
which is~\eqref{eq:Second order Chapman Enskog as Vector}


With~\eqref{eq:First order Chapman Enskog as Vector} written as
\begin{equation*}
	\partial_{t_1} \boldsymbol{N}^{(0)} = (- S_{\alpha}\partial_{\alpha'} \boldsymbol{N}^{(0)} -  \Lambda \boldsymbol{N}^{(1)}),
\end{equation*}
equation~\eqref{eq:Second order Chapman Enskog as Vector} gets

\begin{equation*}
	\begin{aligned}
		& \partial_{t_2} \boldsymbol{N}^{(0)} + \partial_{t_1} \boldsymbol{N}^{(1)} + S_\alpha \partial_{\alpha'} \boldsymbol{N}^{(1)} + \frac{1}{2} \partial_{t_1}\highlight{(- S_{\alpha}\partial_{\alpha'} \boldsymbol{N}^{(0)} -  \Lambda \boldsymbol{N}^{(1)})} +  S_\alpha
		 \partial_{\alpha'}\partial_{t_1} \boldsymbol{N}^{(0)} \\
		+ &\; \frac{1}{2} S_\alpha S_\beta\partial_{\alpha'}\partial_{\beta'} \boldsymbol{N}^{(0)} \\
		= &\; - \Lambda  \boldsymbol{N}^{(2)}
	\end{aligned}
\end{equation*}

\begin{equation*}
	\begin{aligned}
		& \partial_{t_2} \boldsymbol{N}^{(0)} + \partial_{t_1} \boldsymbol{N}^{(1)} + S_\alpha \partial_{\alpha'} \boldsymbol{N}^{(1)} +
		\frac{1}{2} S_{\alpha}\partial_{\alpha'} \partial_{t_1} \boldsymbol{N}^{(0)} -  \Lambda \frac{1}{2} \partial_{t_1} \boldsymbol{N}^{(1)} \\
		+ &\; \frac{1}{2} S_\alpha S_\beta\partial_{\alpha'}\partial_{\beta'} \boldsymbol{N}^{(0)} \\
		= &\; - \Lambda  \boldsymbol{N}^{(2)}
	\end{aligned}
\end{equation*}

\begin{equation*}
	\begin{aligned}
		& \partial_{t_2} \boldsymbol{N}^{(0)} + \partial_{t_1} \boldsymbol{N}^{(1)} + S_\alpha \partial_{\alpha'} \boldsymbol{N}^{(1)} +
		\frac{1}{2} S_{\alpha}\partial_{\alpha'} \highlight{(- S_{\beta}\partial_{\beta'} \boldsymbol{N}^{(0)} -  \Lambda \boldsymbol{N}^{(1)})} -  \Lambda \frac{1}{2} \partial_{t_1} \boldsymbol{N}^{(1)} \\
		+ &\; \frac{1}{2} S_\alpha S_\beta\partial_{\alpha'}\partial_{\beta'} \boldsymbol{N}^{(0)} \\
		= &\; - \Lambda  \boldsymbol{N}^{(2)}
	\end{aligned}
\end{equation*}


\begin{equation*}
	\begin{aligned}
		& \partial_{t_2} \boldsymbol{N}^{(0)} + \partial_{t_1} \boldsymbol{N}^{(1)} + S_\alpha \partial_{\alpha'} \boldsymbol{N}^{(1)} -
		\frac{1}{2} S_{\alpha}\partial_{\alpha'} \Lambda \boldsymbol{N}^{(1)} -  \Lambda \frac{1}{2} \partial_{t_1} \boldsymbol{N}^{(1)} \\
		= &\; - \Lambda  \boldsymbol{N}^{(2)}
	\end{aligned}
\end{equation*}

\begin{equation*}
	\begin{aligned}
		\partial_{t_2} \boldsymbol{N}^{(0)} + (\mathcal{I}-\frac{\Lambda}{2} )\partial_{t_1} \boldsymbol{N}^{(1)} + S_\alpha  (\mathcal{I}-\frac{\Lambda}{2} ) \partial_{\alpha'}  \boldsymbol{N}^{(1)}
		= - \Lambda  \boldsymbol{N}^{(2)}
	\end{aligned}
\end{equation*}


\end{appendices}

\end{document}
