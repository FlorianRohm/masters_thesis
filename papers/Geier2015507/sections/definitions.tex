% !TEX root = ../notes.tex

In the following, only lattices with even cell spacing $c$ in $x$ and $y$ direction are considered.

\subsection{Particle distribution function}
\label{sub:Particle distribution function}

Continuous definition of $f$ using the Diraque delta
\begin{equation}
  \label{eq:Definition of f xi}
  f(\xi) = \sum_{i,j} f_{ij}\delta(ic - \xi_1)\delta(jc - \xi_2).
\end{equation}
Centralized $f$, where $u=\begin{pmatrix}u_1 \\ u_2\end{pmatrix}$ is the macroscopic speed in the point.
\begin{equation}
  \label{eq:Definition of f c xi}
  f_c(\xi) = f(\xi-u) = \sum_{i,j} f_{ij}\delta(ic - (\xi_1-u_1))\delta(jc - (\xi_2-u_2)).
\end{equation}

\subsection{Laplace Transform}
\label{sub:Laplace Transform}

Laplace transformation definition:
\begin{equation}
  \label{eq:Definition of Laplace}
  \mathcal{L}[g](\Xi) = \int_{-\infty}^\infty g(\xi) e^{-\Xi \cdot \xi}d\xi.
\end{equation}
%
Formulation of the Laplace transform of $f$
\begin{equation}
  \label{eq:Definition of F}
  \begin{aligned}
    F(\Xi_1, \Xi_2) & = \mathcal{L}[f](\Xi) = \int_{-\infty}^\infty f(\xi) e^{-\Xi \cdot \xi}d\xi \\
     & = \sum_{i,j}f_{ij} e^{-\Xi_1 ic} e^{-\Xi_2 jc}.
  \end{aligned}
\end{equation}
%
Definition of the Laplace transform of centralized $f$.
\begin{equation}
  \label{eq:Definition of centralized F}
  \begin{aligned}
    F_c(\Xi_1, \Xi_2) & = \mathcal{L}[f_c](\Xi) = \mathcal{L}[f(\cdot - u)](\Xi) \\
    & = \int_{-\infty}^\infty f(\xi-u) e^{-\Xi \cdot \xi}d\xi \\
    & = \int_{-\infty}^\infty f(\xi) e^{-\Xi \cdot (\xi+u)}d\xi \\
    & = \sum_{i,j}f_{ij} e^{-\Xi_1 (ic+u_1)} e^{-\Xi_2 (jc+u_2)} \\
    & = e^{-\Xi_1 u_1}e^{-\Xi_2 u_2}\sum_{i,j}f_{ij} e^{-\Xi_1 ic} e^{-\Xi_2 jc}\\
    & = e^{-\Xi_1 u_1}e^{-\Xi_2 u_2} F(\Xi_1, \Xi_2).
  \end{aligned}
\end{equation}
%
\subsection{Moments}
\label{sub:Moments}

Definition of the moments of $f$
\begin{equation}
  \label{eq:Definition of moments}
  m_{\alpha\beta} = \sum_{ij} i^\alpha j^\beta f_{ij}
\end{equation}
%
Definition of the centralized moments of $f$
\begin{equation}
  \begin{aligned}
    \label{eq:Definition of centralized moments}
    c_{\alpha\beta} & = \sum_{ij} {\left(i-\frac{m_{10}}{m_{00}}\right)}^\alpha {\left(j-\frac{m_{01}}{m_{00}}\right)}^\beta f_{ij} \\
    & = \sum_{ij} {\left(i-\frac{u_1}{c}\right)}^\alpha {\left(j-\frac{u_2}{c}\right)}^\beta f_{ij}
  \end{aligned}
\end{equation}

\subsection{Taylor expansions}
\label{sub:Taylor expansions}

\subsubsection{Moment generating function}
\label{subs:Moment generating function}

Taylor expansion of $F$
\begin{equation}
  \label{eq: taylor of F}
  \begin{aligned}
    F(\Xi_1, \Xi_2) & = \sum_{\alpha,\beta} \frac{1}{\alpha!\beta!} \frac{\partial^\alpha\partial^\beta}{{(\partial \Xi_1)}^\alpha{(\partial \Xi_2)}^\beta} F(\Xi_1, \Xi_2)\Bigr|_{\Xi_1=\Xi_2 = 0} \Xi_1^\alpha \Xi_2^\beta \\
    & = \sum_{\alpha,\beta} \frac{1}{\alpha!\beta!} \frac{\partial^\alpha\partial^\beta}
      {{(\partial \Xi_1)}^\alpha{(\partial \Xi_2)}^\beta}  \sum_{i,j}f_{ij} e^{-\Xi_1 ic} e^{-\Xi_2 jc} \Bigr|_{\Xi_1=\Xi_2 = 0} \Xi_1^\alpha \Xi_2^\beta \\
    & = \sum_{\alpha,\beta} \frac{1}{\alpha!\beta!} {(-c)}^{\alpha+\beta}  \sum_{i,j} i^\alpha j^\beta f_{ij} e^{-\Xi_1 ic} e^{-\Xi_2 jc} \Bigr|_{\Xi_1=\Xi_2 = 0} \Xi_1^\alpha \Xi_2^\beta \\
    & = \sum_{\alpha,\beta} \frac{1}{\alpha!\beta!} {(-c)}^{\alpha+\beta}  \sum_{i,j} i^\alpha j^\beta f_{ij} \Xi_1^\alpha \Xi_2^\beta \\
    & = \sum_{\alpha,\beta} \frac{1}{\alpha!\beta!} {(-c)}^{\alpha+\beta}  m_{\alpha\beta} \Xi_1^\alpha \Xi_2^\beta
  \end{aligned}
\end{equation}
%
Hence,
\begin{equation}
  \label{eq:alternative representation of moments}
  m_{\alpha\beta} = {(-c)}^{-\alpha-\beta} \frac{\partial^\alpha\partial^\beta}{{(\partial \Xi_1)}^\alpha{(\partial \Xi_2)}^\beta} F(\Xi_1, \Xi_2)\Bigr|_{\Xi_1=\Xi_2 = 0}
\end{equation}
and $F(\Xi_1, \Xi_2)$ is the moment generating function of $f(\xi_1, \xi_2)$.

\subsubsection{Centralized moment generating function}
\label{subs:Centralized moment generating function}

Taylor expansion of $F_c$
\begin{equation}
  \label{eq: taylor of Fc}
  \begin{aligned}
    F_c(\Xi_1, \Xi_2) & = \sum_{\alpha,\beta} \frac{1}{\alpha!\beta!} \frac{\partial^\alpha\partial^\beta}{{(\partial \Xi_1)}^\alpha{(\partial \Xi_2)}^\beta} F_c(\Xi_1, \Xi_2)\Bigr|_{\Xi_1=\Xi_2 = 0} \Xi_1^\alpha \Xi_2^\beta \\
    & = \sum_{\alpha,\beta} \frac{1}{\alpha!\beta!} \Bigg(
        \frac{\partial^\alpha\partial^\beta}{{(\partial \Xi_1)}^\alpha{(\partial \Xi_2)}^\beta} \\
          &\qquad\qquad
          \sum_{i,j}f_{ij} e^{-\Xi_1 (ic+u_1)} e^{-\Xi_2 (jc+u_2)} \Bigg)\Bigr|_{\Xi_1=\Xi_2 = 0} \Xi_1^\alpha \Xi_2^\beta \\
    & = \sum_{\alpha,\beta} \frac{1}{\alpha!\beta!} \sum_{i,j} {(ic+u_1)}^\alpha {(jc+u_2)}^\beta f_{ij} e^{-\Xi_1 ic} e^{-\Xi_2 jc} \Bigr|_{\Xi_1=\Xi_2 = 0} \Xi_1^\alpha \Xi_2^\beta \\
    & = \sum_{\alpha,\beta} \Bigg(
        \frac{1}{\alpha!\beta!} {(-c)}^{\alpha+\beta}\\
        &\qquad\qquad  \sum_{i,j} {\left(i+\frac{u_1}{c}\right)}^\alpha {\left(j+\frac{u_2}{c}\right)}^\beta f_{ij} e^{-\Xi_1 ic} e^{-\Xi_2 jc}
      \Bigg)\Bigr|_{\Xi_1=\Xi_2 = 0} \Xi_1^\alpha \Xi_2^\beta \\
    & = \sum_{\alpha,\beta} \frac{1}{\alpha!\beta!} {(-c)}^{\alpha+\beta}
      \sum_{i,j} {\left(i+\frac{u_1}{c}\right)}^\alpha {\left(j+\frac{u_2}{c}\right)}^\beta f_{ij} \Xi_1^\alpha \Xi_2^\beta \\
    & = \sum_{\alpha,\beta} \frac{1}{\alpha!\beta!} {(-c)}^{\alpha+\beta}  c_{\alpha\beta} \Xi_1^\alpha \Xi_2^\beta
  \end{aligned}
\end{equation}
%
Hence,
\begin{equation}
  \label{eq:alternative representation of central moments}
  c_{\alpha\beta} = {(-c)}^{-\alpha-\beta} \frac{\partial^\alpha\partial^\beta}{{(\partial \Xi_1)}^\alpha{(\partial \Xi_2)}^\beta} F_c(\Xi_1, \Xi_2)\Bigr|_{\Xi_1=\Xi_2 = 0}
\end{equation}


\subsection{Cumulants}
\label{sub:Cumulants}

Cumulants are defined as the coefficients of the Taylor-expansion of the logarithm of the moment generating function:
\begin{equation}
  \label{eq:Definition of cumulants}
  \kappa_{\alpha\beta} = {(-c)}^{-\alpha-\beta} \frac{\partial^\alpha\partial^\beta}{{(\partial \Xi_1)}^\alpha{(\partial \Xi_2)}^\beta} \ln(F(\Xi_1, \Xi_2))\Bigr|_{\Xi_1=\Xi_2 = 0}
\end{equation}
