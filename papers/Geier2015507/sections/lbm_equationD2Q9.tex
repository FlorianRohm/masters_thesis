% !TEX root = ../notes.tex

With $c$ defined as the cell width and heigth, velocities are given by
\begin{equation}
  c_{ij}= c\begin{pmatrix}i \\ j\end{pmatrix}, \quad i,j\in \{-1, 0, 1\}
\end{equation}
and the lattice Boltzmann equation is given by
\begin{equation}
  f_{ij}(t + \Delta t, x + i \Delta x , y + j \Delta y) = f^*_{ij}(t,x,y), \quad i,j\in \{-1, 0, 1\}
\end{equation}
or equivalently by shifting $x$ and $y$
\begin{equation}
  \label{eq:Lattice Boltzmann Equation}
  f_{ij}(t + \Delta t, x, y) = f^*_{ij}(t,x - i\Delta x , y - j\Delta y), \quad i,j\in \{-1, 0, 1\}
\end{equation}
where $f_{ij}$ are the pre-collission distributions and $f^*_{ij}$ the post collission distributions.
The relation between $\Delta x$, $\Delta y$ and $\Delta t$ should be
\begin{equation}
  \Delta x = \Delta y = c \Delta t
\end{equation}

\subsection{Aliasing of moments}
\label{sub:Aliasing of moments}

As the D2Q9 only has $9$ independent velocities, the moment-space spanned by those can only host $9$ independent moments.
In D2Q9, the moments are defined, according to~\eqref{eq:Definition of moments} as
\begin{equation*}
  m_{\alpha\beta} = \sum_{i,j \in \{-1,0,1\}} i^\alpha j^\beta f_{ij}.
\end{equation*}
Hence
\begin{equation}
  \label{eq:aliasing of moments 1}
  \sum_{i,j \in \{-1,0,1\}} i^\alpha j^\beta f_{ij} = \sum_{i,j \in \{-1,0,1\}} i^{(\alpha+2k)} j^\beta f_{ij} \quad \forall k, \forall \alpha \neq 0
\end{equation}
and
\begin{equation}
  \label{eq:aliasing of moments 2}
  \sum_{i,j \in \{-1,0,1\}} i^\alpha j^\beta f_{ij} = \sum_{i,j \in \{-1,0,1\}} i^\alpha j^{(\beta+2l)} f_{ij} \quad \forall l, \forall \beta \neq 0.
\end{equation}
According to~\eqref{eq:aliasing of moments 1} and~\eqref{eq:aliasing of moments 2}, the list of moments reads

\begin{table} [h!]
  \centering
  \begin{tabular}{c cc ccc cc c}
    \toprule
    \multicolumn{9}{c}{independent moments}  \\
    \cmidrule(lr){1-9} \\
    $m_{00}$   &   $m_{10}$ & $m_{01}$   &   $m_{11}$ & $m_{20}$ & $m_{02}$   &   $m_{21}$ & $m_{12}$   &   $m_{22}$ \\
    \bottomrule
  \end{tabular}
  \newline
  \vspace*{.5 cm}
  \newline
  \begin{tabular}{ccccc c}
    \toprule
    \multicolumn{6}{c}{dependent moments}   \\
    \cmidrule(lr){1-5}   \\
    $m_{30}$   & $m_{03}$   & $m_{31}$  & $m_{13}$  & $m_{40}$  & \ldots \\
    = $m_{10}$ & = $m_{01}$ & = $m_{11}$ & = $m_{11}$ & = $m_{20}$ & \\
    \bottomrule
  \end{tabular}
  \caption{D2Q9 moments}\label{table:D2Q9 moments}
\end{table}
