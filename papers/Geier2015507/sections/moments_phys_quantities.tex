% !TEX root = ../notes.tex

The following connections can be made:
\todo[inline]{Cite Harris}
\paragraph{Density}
\label{par:Density}
\begin{equation}
  \rho = \sum_{i,j} f_{ij} = m_{00}
\end{equation}

\paragraph{Macroscopic velocity}
\label{par:Macroscopic velocity}
\begin{equation}
  \begin{aligned}
    v_x & = \frac{1}{\rho}\sum_{i,j} i f_{ij} = \frac{ m_{10} }{ m_{00} }\\
    v_y & = \frac{1}{\rho}\sum_{i,j} j f_{ij} = \frac{ m_{10} }{ m_{00} }
  \end{aligned}
\end{equation}

\paragraph{Microscopic velocities}
\label{par:Microscopic velocities}

The microscopic velocity describes the speed at \\which the particles move. The so called  peculiar velocity $\mathbf{v}_0$ is the particle velocity in a frame of reference moving with the flow, i.e. $i_0 \defined i_0(i) = i-v_x$, $j_0 \defined j_0(j) = j-v_y$.

In this frame, the observed macroscopic velocity of course would be zero.

\paragraph{Pressure}
\label{par:Pressure}
The pressure tensor $\mathbf{P}$ is given as
\begin{equation}
  \mathbf{P}_{xy} = \sum_{i,j} i_0 j_0 f_{ij}
\end{equation}
with the macroscopic pressure defined as
\begin{equation}
  \label{eq:pressure}
  p = \frac{1}{3}\mathbf{P}_{xx} = \frac{1}{3}\sum_{i,j} i_0^2 f_{ij} = \frac{1}{3} \left(m_{20} - \frac{ m_{10}^2 }{ m_{00} } \right)
\end{equation}
