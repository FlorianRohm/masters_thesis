% !TEX root = ../notes.tex

\subsection{Expanding the Lattice Boltzmann Equation}
\label{sub:Expanding the Lattice Boltzmann Equation}

A Taylor expansion of~\eqref{eq:Lattice Boltzmann Equation} in both time and space gives
\begin{equation*}
  \sum_{\tau = 0}^\infty \frac{{\Delta t}^\tau }{\tau!} \frac{\partial^\tau}{{\partial t}^\tau} f_{ij}(t, x, y) =
  \sum_{m,n = 0}^\infty \frac{{(-i\Delta x)}^m{(-j\Delta x)}^n} {m!n!} \frac{\partial^m \partial^n}{ {\partial x}^m{\partial y}^n} f^*_{ij}(t, x, y)
\end{equation*}
which is equivalent to
\begin{equation}
  \label{eq:Taylor LB1}
  \sum_{\tau = 0}^\infty \frac{{\Delta t}^\tau }{\tau!} \frac{\partial^\tau}{{\partial t}^\tau} f_{ij}(t, x, y) =
    \sum_{m,n = 0}^\infty \frac{{(-c\Delta t)}^{m+n}} {m!n!} \frac{\partial^m \partial^n}{ {\partial x}^m{\partial y}^n} i^m j^n f^*_{ij}(t, x, y),
\end{equation}
using $\frac{\Delta x} {\Delta{t}} = c$.

Equation~\eqref{eq:Taylor LB1} holds for all $f_{ij}$ and thus also for all linear combinations
\begin{equation*}
  \sum_{k}\lambda_k f_{ij},
\end{equation*}
especially for those, which yield the moments, $m_{\alpha\beta} $, of $f$:
\begin{equation*}
  \sum_{k}\lambda_k f_{ij} = \sum_{i,j}i^\alpha j^\beta f_{ij}=m_{\alpha\beta},
\end{equation*}
where the summation indices $i$, $j$ are not related to the velocities of~\eqref{eq:Taylor LB1}.
Therefore, we rewrite the system of equations for the $f_{ij}$ to a system of equations for the $m_{\alpha\beta}$.

\begin{align}
    \nonumber
    & & \sum_{ij} \sum_{\tau = 0}^\infty \frac{{\Delta t}^\tau }{\tau!} \frac{\partial^\tau}{{\partial t}^\tau} i^\alpha j^\beta f_{ij} &=
    \sum_{ij}\sum_{m,n = 0}^\infty \frac{i^m j^n {(-c\Delta t)}^{m+n}} {m!n!} \frac{\partial^m \partial^n}{ {\partial x}^m{\partial y}^n}i^\alpha j^\beta f^*_{ij}\\
    \nonumber
    &\equivalent &
    \sum_{\tau = 0}^\infty \frac{{\Delta t}^\tau }{\tau!} \frac{\partial^\tau}{{\partial t}^\tau} \sum_{ij}i^\alpha j^\beta f_{ij} &=
    \sum_{m,n = 0}^\infty \frac{{(-c\Delta t)}^{m+n}} {m!n!} \frac{\partial^m \partial^n}{ {\partial x}^m{\partial y}^n}\sum_{ij}i^{(\alpha + m)} j^{(\beta+n)} f^*_{ij}\\
      \label{eq: Taylor of moments}
     &\equivalent &
     \sum_{\tau = 0}^\infty \frac{{\Delta t}^\tau }{\tau!} \frac{\partial^\tau}{{\partial t}^\tau} m_{\alpha\beta} &=
    \sum_{m,n = 0}^\infty \frac{{(-c\Delta t)}^{m+n}} {m!n!} \frac{\partial^m \partial^n}{ {\partial x}^m{\partial y}^n} m^*_{(\alpha + m)(\beta + n)}
\end{align}
For the scale analysis, we introduce a dimensionless scaling parameter $\epsilon$.
Adapting the widely used diffusive scaling, we nondimensionalize the equations with characteristic length and timescales, $L$ and $\iota$
\begin{equation}
  \label{eq:nondimensionalisation}
  \begin{aligned}
    \Delta x & = L\epsilon \\
    \Delta t & = \iota\epsilon^2 \\
    c & = \frac{L}{\iota\epsilon} \\
    \{x, y\} & \rightarrow \{\frac{x}{L}, \frac{y}{L}\} \\
    t & \rightarrow \frac{t}{\iota}.
  \end{aligned}
\end{equation}
Additionally, the moments are expanded according to the scale $\epsilon$
\begin{align}
    \label{eq:expansion of m}
    m_{\alpha\beta} & = \sum_{p=0}^{\infty} \epsilon^p m_{\alpha\beta}^{(p)} \\
    \label{eq:expansion of m*}
    m^*_{\alpha\beta} & = \sum_{p=0}^{\infty} \epsilon^p m_{\alpha\beta}^{*(p)}.
\end{align}
Those are still dependent of space and time, but as they are separated on their length scales, they are not allowed to depend on $\epsilon$ neither implicitly nor explicitly.
Inserting~\eqref{eq:nondimensionalisation},~\eqref{eq:expansion of m} and~\eqref{eq:expansion of m*} into~\eqref{eq: Taylor of moments} yields

\begin{align}
  \phantom{\equivalent\qquad}
  &\begin{aligned}
  \nonumber
    \mathllap{\equivalent\qquad} & \sum_{\tau = 0}^\infty \frac{{\Delta t}^\tau }{\tau!}  \frac{\partial^\tau}{{\partial t}^\tau} \sum_{p=0}^{\infty} \epsilon^p m_{\alpha\beta}^{(p)} \\
    = &{}\sum_{m,n = 0}^\infty \frac{{(-c\Delta t)}^{m+n}} {m!n!} \frac{\partial^m \partial^n}{ {\partial x}^m{\partial y}^n} \sum_{p=0}^{\infty} \epsilon^p m_{(\alpha + m)(\beta + n)}^{*(p)}
  \end{aligned}\\
  %
  &\begin{aligned}
    \label{eq:nondimensionalised expansion series}
    \mathllap{\equivalent\qquad} &
    \sum_{\tau = 0}^\infty \frac{{(\iota\epsilon^2)}^\tau }{\tau!} \frac{\partial^\tau}{{\partial t}^\tau \iota^\tau} \sum_{p=0}^{\infty} \epsilon^p m_{\alpha\beta}^{(p)} \\
    = &{}\sum_{m,n = 0}^\infty \frac{{(-\frac{L}{\iota\epsilon}(\iota\epsilon^2))}^{m+n}} {m!n!}
    \frac{\partial^m \partial^n}{ {\partial x}^m{\partial y}^n L^{m+n}} \sum_{p=0}^{\infty} \epsilon^p m_{(\alpha + m)(\beta + n)}^{*(p)}
  \end{aligned} \\
 %
  &\begin{aligned}
  \nonumber
    \mathllap{\equivalent\qquad} &
    \sum_{\tau = 0}^\infty \frac{\epsilon^{2\tau} }{\tau!} \frac{\partial^\tau}{{\partial t}^\tau} \sum_{p=0}^{\infty} \epsilon^p m_{\alpha\beta}^{(p)} \\
    = &{}\sum_{m,n = 0}^\infty \frac{{(-\epsilon)}^{m+n}} {m!n!}
    \frac{\partial^m \partial^n}{ {\partial x}^m{\partial y}^n } \sum_{p=0}^{\infty} \epsilon^p m_{(\alpha + m)(\beta + n)}^{*(p)}
  \end{aligned} \\
 %
  &\begin{aligned}
    \label{eq:final nondimensionalised expansion series}
    \mathllap{\equivalent\qquad} &
    \sum_{\tau, p = 0}^\infty\epsilon^{2\tau + p} \frac{1} {\tau!} \frac{\partial^\tau}{{\partial t}^\tau} m_{\alpha\beta}^{(p)} \\
    = &{} \sum_{m,n,p' = 0}^\infty  \epsilon^{m+n+p'} \frac{{(-1)}^{m+n}} {m!n!}
    \frac{\partial^m \partial^n}{ {\partial x}^m{\partial y}^n } m_{(\alpha + m)(\beta + n)}^{*(p')}
  \end{aligned}
\end{align}
Where in~\eqref{eq:final nondimensionalised expansion series}, the summation indices are changed to be unique for better recognition.

\subsection{Matching the equating coefficients}
\label{sub:Matching the equating coefficients}
As $\epsilon$ is arbitrary and the terms of the expansion are not allowed to depend on $\epsilon$, equality has to hold in every order of $\epsilon$.

\subsubsection{Zero'th order in \texorpdfstring{$\epsilon$}{epsilon}}
\label{subs:Zero'th order in epsilon}

To achieve $\epsilon^0$, we have to choose $\tau=p=p'=m=n=0$, yielding
\begin{equation}
  \label{eq:zeroth order in epsilon}
  m_{\alpha\beta}^{(0)} = m_{\alpha\beta}^{*(0)}.
\end{equation}


\subsubsection{First order in \texorpdfstring{$\epsilon$}{epsilon}}
\label{subs:First order in epsilon}

For the first order terms, one has to choose $p=1$ on the left side and one of $m$, $n$ and $p'$ equal to one.

\begin{equation}
  \label{eq:first order in epsilon}
  \begin{aligned}
  m_{\alpha\beta}^{(1)}
  & = m_{\alpha\beta}^{*(1)}
  - \partial_x m_{(\alpha+1)\beta}^{*(0)} - \partial_y m_{\alpha(\beta+1)}^{*(0)} \\
  & = m_{\alpha\beta}^{*(1)}
  \end{aligned}
\end{equation}
The second line is due to the fact, that the zeroth order terms have to be constant in space and thus all spatial derivatives vanish.

\subsubsection{Second order in \texorpdfstring{$\epsilon$}{epsilon}}
\label{subs:Second order in epsilon}

Going to the second order requires on the left side $\tau=1$ or $p=2$.
On the right side, we have six possibilities:
\begin{center}
  \begin{tabular} {r || c | *{2}{c} | *{2}{c} | c}
    m  & 0 & 1 & 2 & 0 & 0 & 1 \\
    n  & 0 & 0 & 0 & 1 & 2 & 1 \\
    p' & 2 & 1 & 0 & 1 & 0 & 0
  \end{tabular}
\end{center}
Resulting in
\begin{equation}
  \label{eq:second order in epsilon}
  \begin{aligned}
    \partial_t m_{\alpha\beta}^{(0)} + m_{\alpha\beta}^{(2)}
    & =  m_{\alpha\beta}^{*(2)} \\
    &\quad - \partial_x m_{(\alpha+1)\beta}^{*(1)} + \partial_{xx} m_{(\alpha+2)\beta}^{*(0)}/2 \\
    &\quad - \partial_y m_{\alpha(\beta+1)}^{*(1)} + \partial_{yy} m_{\alpha(\beta+2)}^{*(0)}/2 \\
    &\quad + \partial_{xy} m_{(\alpha+1)(\beta+1)}^{*(0)}\\
    & =  m_{\alpha\beta}^{*(2)} - \partial_x m_{(\alpha+1)\beta}^{*(1)} - \partial_y m_{\alpha(\beta+1)}^{*(1)}
  \end{aligned}
\end{equation}


\subsubsection{Third order in \texorpdfstring{$\epsilon$}{epsilon}}
\label{subs:Third order in epsilon}

Left hand side, either $\begin{pmatrix}\tau \\ p\end{pmatrix} = \begin{pmatrix} 1 \\ 1 \end{pmatrix}$ or $\begin{pmatrix}\tau \\ p\end{pmatrix} = \begin{pmatrix} 0 \\ 3 \end{pmatrix}$.\newline
Right hand side:
\begin{center}
  \begin{tabular} {r || c | *{3}{c} | *{3}{c} | *{3}{c} }
    m  & 0 & 1 & 2 & 3 & 0 & 0 & 0 & 1 & 2 & 1 \\
    n  & 0 & 0 & 0 & 0 & 1 & 2 & 3 & 1 & 1 & 2 \\
    p' & 3 & 2 & 1 & 0 & 2 & 1 & 0 & 1 & 0 & 0
  \end{tabular}
\end{center}

This gives
\begin{equation}
  \label{eq:third order in epsilon}
  \begin{aligned}
    \frac{\partial}{\partial t} m_{\alpha\beta}^{(1)} + m_{\alpha\beta}^{(3)}
    & =  m_{\alpha\beta}^{*(3)} \\
    &\quad - \partial_x m_{(\alpha+1)\beta}^{*(2)} + \partial_{xx} m_{(\alpha+2)\beta}^{*(1)}/2 - \partial_{xxx} m_{(\alpha+3)\beta}^{*(0)}/6 \\
    &\quad - \partial_y m_{\alpha(\beta+1)}^{*(2)} + \partial_{yy} m_{\alpha(\beta+2)}^{*(1)}/2 - \partial_{yyy} m_{\alpha(\beta+3)}^{*(0)}/6 \\
    &\quad + \partial_{xy} m_{(\alpha+1)(\beta+1)}^{*(1)} \\
    &\quad - \partial_{xxy} m_{(\alpha+2)(\beta+1)}^{*(0)}/2 - \partial_{xyy} m_{(\alpha+1)(\beta+2)}^{*(0)}/2 \\
    & =  m_{\alpha\beta}^{*(3)} \\
    &\quad - \partial_x m_{(\alpha+1)\beta}^{*(2)} - \partial_y m_{\alpha(\beta+1)}^{*(2)}  \\
    &\quad  + \partial_{xx} m_{(\alpha+2)\beta}^{*(1)}/2 + \partial_{xy} m_{(\alpha+1)(\beta+1)}^{*(1)} + \partial_{yy} m_{\alpha(\beta+2)}^{*(1)}/2
  \end{aligned}
\end{equation}

\subsection{Expanding the normalized cumulants}
\label{sub:Expanding the normalized cumulants}
We start by expanding the first nontrivial normalized cumulant, $K_{11}$, which was defined in~\eqref{eq: K_11 from moments}.
\begin{equation}
  \begin{aligned}
    K_{11} & = m_{11} - \frac{m_{10}m_{01}}{m_{00}}\\
    \Leftrightarrow
    \sum_{i=0}^\infty \epsilon^i K_{11}^{(i)}
    & = \sum_{i=0}^\infty \epsilon^i m_{11}^{(i)} -
    \frac{\sum_{i,j=0}^\infty \epsilon^{i+j} m_{10}^{(i)}m_{01}^{(j)}}
         {\sum_{i=0}^\infty \epsilon^i m_{00}^{(i)}}
  \end{aligned}
\end{equation}
Now is a good time to fix the frame of reference for this analysis. As we'll see, the analysis will get way more easy when we choose a reference frame, moving like the constant background flow. This gives us
\begin{equation}
  \label{eq: frame of reference}
  m_{10}^{(0)}=m_{10}^{(0)} = 0.
\end{equation}

\subsubsection{\texorpdfstring{$K_{11}$}{K 11}}
\label{subs:K_11}

Now to analyse the cumulant $K_{11}$. We can get rid of the denominator, using the geometric series, which leads to
\begin{equation}
  \begin{aligned}
    K_{11} & = m_{11} - \frac{m_{10}m_{01}}{m_{00}}\\
    \Leftrightarrow
    \sum_{i=0}^\infty \epsilon^i K_{11}^{(i)}
    & = \sum_{i=0}^\infty \epsilon^i m_{11}^{(i)} -
    \frac{\sum_{i,j=0}^\infty \epsilon^{i+j} m_{10}^{(i)}m_{01}^{(j)}}
        {\sum_{i=0}^\infty \epsilon^i m_{00}^{(i)}}\\
    \Leftrightarrow
    \sum_{i=0}^\infty \epsilon^i K_{11}^{(i)}
    & = \sum_{i=0}^\infty \epsilon^i m_{11}^{(i)} -
    \frac{\sum_{i,j=0}^\infty \epsilon^{i+j} m_{10}^{(i)}m_{01}^{(j)}}
        {m_{00}^{(0)}}
        \frac{1}{1 - \sum_{i=1}^\infty \epsilon^i \frac{ - m_{00}^{(i)}}{ m_{00}^{(0)}}}\\
    \Leftrightarrow
    \sum_{i=0}^\infty \epsilon^i K_{11}^{(i)}
    & = \sum_{i=0}^\infty \epsilon^i m_{11}^{(i)} -
    \frac{\sum_{i,j=0}^\infty \epsilon^{i+j} m_{10}^{(i)}m_{01}^{(j)}}
        {m_{00}^{(0)}}
    \sum_{j=0}^\infty {\left(\sum_{i=1}^\infty \epsilon^i \frac{ - m_{00}^{(i)}}{ m_{00}^{(0)}}\right)}^j
  \end{aligned}
\end{equation}
For zeroth order terms, we get
\begin{equation}
  K_{11}^{(0)} = m_{11}^{(0)} - \frac{m_{01}^{(0)}m_{10}^{(0)}}{m_{00}^{(0)}} = m_{11}^{(0)}
\end{equation}
by our choice of the frame of reference,~\eqref{eq: frame of reference}.
Analogously, for first order terms, we get
\begin{equation}
  \begin{aligned}
    K_{11}^{(1)} & = m_{11}^{(1)} \\
    &\quad - \frac{m_{01}^{(1)}m_{10}^{(0)}+m_{01}^{(0)}m_{10}^{(1)}}{m_{00}^{(0)}}
    -\frac{m_{01}^{(0)}m_{10}^{(0)}m_{00}^{(0)}}{m_{00}^{(0)2}}  \\
    &= m_{11}^{(0)}.
  \end{aligned}
\end{equation}
Thus, the normalized cumulant $K_{11}$ is equal to the corresponding moment at order zero and one.

For the second order terms, we get
\begin{equation}
  \begin{aligned}
    K_{11}^{(2)}
    &= m_{11}^{(2)} \\
    &\quad-
    \frac{
      m_{10}^{(1)}m_{01}^{(1)}
    + m_{10}^{(0)}m_{01}^{(2)}
    + m_{10}^{(2)}m_{01}^{(0)}
    }{m_{00}^{(0)}}\underbrace{1}_{j=0}\\
    &\quad
    +(m_{10}^{(1)}m_{01}^{(0)} + m_{10}^{(0)}m_{01}^{(1)})\underbrace{ \frac{m_{00}^{(1)}}{m_{00}^{(0)2}}}_{j=1,i=1}\\
    &\quad
    + m_{10}^{(0)}m_{01}^{(0)}
    \left(
      \underbrace{\frac{m_{00}^{(2)}}{{m_{00}^{(0)2}}}}_{j=1,i=2}
      - \underbrace{\frac{m_{00}^{(1)2}}{m_{00}^{(0)2}}}_{j=2,i=1}
    \right)\\
    &= m_{11}^{(2)} \\
    &\quad-
    \frac{
      m_{10}^{(1)}m_{01}^{(1)}
    + m_{10}^{(0)}m_{01}^{(2)}
    + m_{10}^{(2)}m_{01}^{(0)}
    + m_{10}^{(0)}m_{01}^{(0)}m_{00}^{(1)2}
    }{m_{00}^{(0)}}\\
    &\quad
    + \frac{m_{00}^{(1)}(m_{10}^{(1)}m_{01}^{(0)} + m_{10}^{(0)}m_{01}^{(1)} - m_{10}^{(0)}m_{01}^{(0)}m_{00}^{(1)})}{m_{00}^{(0)2}}  \\
    &\quad
    + \frac{m_{00}^{(2)}m_{10}^{(0)}m_{01}^{(0)}}{m_{00}^{(0)2}} \\
    & = m_{11}^{(2)}
    - \frac{ m_{10}^{(1)}m_{01}^{(1)}}{m_{00}^{(0)}}.
  \end{aligned}
\end{equation}

\subsubsection{\texorpdfstring{$K_{20}$}{K 20} and \texorpdfstring{$K_{02}$}{K 02}}
\label{subs:K_20}
Going further to $K_{20}$, we get with the same reasoning
\begin{equation}
  \begin{aligned}
    K_{20} & = m_{20} - \frac{m_{10}^2}{m_{00}}\\
    \Leftrightarrow
    \sum_{i=0}^\infty \epsilon^i K_{20}^{(i)}
    & = \sum_{i=0}^\infty \epsilon^i m_{20}^{(i)} -
    \frac{\sum_{i,j=0}^\infty \epsilon^{i+j} m_{10}^{(i+j)}}
        {m_{00}^{(0)}}
    \sum_{j=0}^\infty {\left(\sum_{i=1}^\infty \epsilon^i \frac{ - m_{00}^{(i)}}{ m_{00}^{(0)}}\right)}^j
  \end{aligned}
\end{equation}
As before, the zeroth and first order terms of the cumulant and corresponding moment coincide.
For the second order terms, we get
\begin{equation}
  K_{20}^{(2)} = m_{20}^{(2)}
  - \frac{ m_{10}^{(1)2}}{m_{00}^{(0)}},
\end{equation}
and consequently
\begin{equation}
  K_{02}^{(2)} = m_{02}^{(2)}
  - \frac{ m_{01}^{(1)2}}{m_{00}^{(0)}}.
\end{equation}

\subsubsection{\texorpdfstring{$K_{21}$}{K 21} and \texorpdfstring{$K_{12}$}{K 12}}
\label{subs:K_21}
As before, we take the defining equation for the normalized cumulant, in this case~\eqref{eq: K_21 from moments}, and plug in all expansions with the geometric series expansion like in~\ref{subs:K_11}.
\begin{equation}
  \begin{aligned}
    \sum_{i=0}^\infty \epsilon^i K_{21}^{(i)}
    & = m_{21}
    \\ &\quad
    - \frac{\sum_{i,j=0}^\infty \epsilon^{i+j} m_{20}^{(i)} m_{01}^{(j)}}{m_{00}^{(0)}}
      \sum_{j=0}^\infty {\left(\sum_{i=1}^\infty \epsilon^i \frac{ - m_{00}^{(i)}}{ m_{00}^{(0)}}\right)}^j
    \\ &\quad
     - 2\frac{\sum_{i,j=0}^\infty \epsilon^{i+j} m_{10}^{(i)} m_{11}^{(j)}}{m_{00}^{(0)}}
      \sum_{j=0}^\infty {\left(\sum_{i=1}^\infty \epsilon^i \frac{ - m_{00}^{(i)}}{ m_{00}^{(0)}}\right)}^j
     \\ &\quad
     + 2\frac{\sum_{i,j,k=0}^\infty \epsilon^{i+j+k} m_{10}^{(i)} m_{10}^{(j)} m_{01}^{(k)}}{m_{00}^{(0)2}}
      {\left(
        \sum_{j=0}^\infty {\left(\sum_{i=1}^\infty \epsilon^i \frac{ - m_{00}^{(i)}}{ m_{00}^{(0)}}\right)}^j
      \right)}^2\\
   \end{aligned}
\end{equation}
Collecting the zeroth order terms, we find again
\begin{equation}
  K_{21}^{(0)} = m_{21}^{(0)}.
\end{equation}
In contrary to the aforementioned, the first order expansions differ:
\begin{equation}
  \begin{aligned}
    K_{21}^{(1)} & = m_{21}^{(1)}
    \\ &\quad
    - \frac{m_{20}^{(0)} m_{01}^{(1)}}{m_{00}^{(0)}}
    \\ &\quad
    - 2\frac{m_{10}^{(1)} m_{11}^{(0)}}{m_{00}^{(0)}}.
  \end{aligned}
\end{equation}

\subsubsection{\texorpdfstring{$K_{22}$}{K 22}}
\label{subs:K_22}
\begin{equation}
  \begin{aligned}
K_{22} & = m_{22} \\
     & \quad - 2 \frac{m_{10}m_{12}}{m_{00}} - 2\frac{m_{21}m_{01}}{m_{00}}
      - 2 \frac{m_{11}^2}{m_{00}} - \frac{m_{20}m_{02}}{m_{00}} \\
     & \quad + 2 \frac{m_{10}^2 m_{02}}{m_{00}^2} + 8 \frac{m_{10}m_{11}m_{01}}{m_{00}^2}
      + 2 \frac{m_{20}m_{01}^2}{m_{00}^2} \\
     & \quad - 6 \frac{m_{10}^2 m_{01}^2}{m_{00}^3}
   \end{aligned}
 \end{equation}
