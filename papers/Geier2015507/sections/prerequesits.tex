% !TEX root = ../notes.tex

Before starting to analyse the terms, we need to set up
\subsection{Absence of sources}
\label{sub:Absence of sources}
The described cumulant method shall not have any sources and sinks and be incompressible. Hence
\begin{equation}
    \rho  = m_{00} = m_{00}^*
\end{equation}
and thus
\begin{equation}
    \rho = m_{00}^{(p)} = m_{00}^{*(p)}
\end{equation}

Velocity is no conserved quantity in our setting, because forcing terms like gravity can alter it.

\subsection{Zeroth order moments are constant in space}
\label{sub:Zeroth order moments are constant in space}
Zeroth order terms are independent of our expansion parameter $\epsilon$ and thus have to be constant in space. Hence
\begin{equation}
  \partial_x m_{\alpha\beta}^{(0)} = 0
\end{equation}
\begin{equation}
  \partial_y m_{\alpha\beta}^{(0)} = 0
\end{equation}


\subsection{Collision invariants}
\label{sub:Collision invariants}
As the zeroth order terms of the expansion have to be constant in space, we get
\begin{equation}
  \label{eq:first order collision invariant}
  \begin{aligned}
    m_{\alpha\beta}^{(1)}
    & = m_{\alpha\beta}^{*(1)}
    - \partial_x m_{(\alpha+1)\beta}^{*(0)}
    - \partial_y m_{\alpha(\beta+1)}^{*(0)} \\
    & = m_{\alpha\beta}^{*(1)}
  \end{aligned}
\end{equation}
Thus, all moments of order zero,~\eqref{eq:zeroth order in epsilon},
and one,~\eqref{eq:first order collision invariant}, in $\epsilon$, are collision invariants.

\subsection{Collision of moments}
\label{sub:Collision of moments}
Last but not least, we take a look at the collision and insert the findings of Section~\ref{sub:Expanding the normalized cumulants}.
Starting from~\eqref{eq:collision equation system full}, multiplying by $m_{00}$ we get
\begin{equation}
  \begin{aligned}
    K_{11}^{*} & = (1-\omega_1)K_{11} \\
    K_{20}^{*} - K_{02}^{*} & = (1-\omega_1) (K_{20} - K_{02}) \\
    K_{20}^{*} + K_{02}^{*} & = (1-\omega_2) (K_{20} + K_{02}) + \omega_2 \frac{2 m_{00}}{3} \\
    K_{21}^{*} & = (1-\omega_3)K_{21} \\
    K_{12}^{*} & = (1-\omega_3)K_{12} \\
    K_{22}^{*} & = (1-\omega_4)K_{22}.
  \end{aligned}
\end{equation}
now, insert the moment representations at first order to get
\begin{equation}
  \begin{aligned}
    m_{11}^{(1)*}
    & = (1-\omega_1)m_{11}^{(1)} \\
%
    m_{20}^{(1)*} - m_{02}^{(1)*}
    & = (1-\omega_1) (m_{20}^{(1)} - m_{02}^{(1)}) \\
%
    m_{20}^{(1)*} + m_{02}^{(1)*}
    & = (1-\omega_2) (m_{20}^{(1)} + m_{02}^{(1)}) + \omega_2 \frac{2 m_{00}^{(1)}}{3} \\
%
    m_{21}^{(1)*} - \theta m_{01}^{(1)*}
    & = (1-\omega_3)(m_{21}^{(1)} - \theta m_{01}^{(1)}) \\
%
    m_{12}^{(1)*} - \theta m_{10}^{(1)*}
    & = (1-\omega_3)(m_{12}^{(1)} - \theta m_{01}^{(1)}) \\
%
    m_{22}^{(1)*} - \theta (m_{20}^{(1)*} + m_{02}^{(1)*})
    & = (1-\omega_4)(m_{22}^{(1)} - \theta (m_{20}^{(1)} + m_{02}^{(1)})) \\
\Leftrightarrow
    m_{22}^{(1)*}
    & = (1-\omega_4)(m_{22}^{(1)} - \theta (m_{20}^{(1)} + m_{02}^{(1)}))
    \\&\quad
    + \theta ((1-\omega_2) (m_{20}^{(1)} + m_{02}^{(1)}) + \omega_2 \frac{2 m_{00}^{(1)}}{3}) .
  \end{aligned}
\end{equation}
Inserting the second order approximations
\begin{equation}
  \begin{aligned}
    m_{11}^{(2)*} - \frac{ m_{10}^{(1)*}m_{01}^{(1)*}}{m_{00}^{(0)*}} & = (1-\omega_1)\left(m_{11}^{(2)} - \frac{ m_{10}^{(1)}m_{01}^{(1)}}{m_{00}^{(0)}}\right) \\
    m_{20}^{(2)*}-m_{02}^{(2)*} - \frac{ m_{10}^{(1)*2} - m_{01}^{(1)*2}}{m_{00}^{(0)*}} & = (1-\omega_1) \left(m_{20}^{(2)}-m_{02}^{(2)} - \frac{ m_{10}^{(1)2} - m_{01}^{(1)2}}{m_{00}^{(0)}}\right) \\
    %
    m_{20}^{(2)*}+m_{02}^{(2)*} - \frac{ m_{10}^{(1)*2} + m_{01}^{(1)*2}}{m_{00}^{(0)*}} & = (1-\omega_2)
    \left(m_{20}^{(2)}+m_{02}^{(2)} - \frac{ m_{10}^{(1)2} + m_{01}^{(1)2}}{m_{00}^{(0)}}\right)
    \\&\quad
    + \omega_2 \frac{2 m_{00}^{(2)}}{3} \\
  \end{aligned}
\end{equation}

\paragraph{Reduced case}
\label{par:Reduced case}
In the reduced case where only $\omega_1$ is not one, we get
\begin{equation}
  \begin{aligned}
    m_{11}^{(1)*} & = (1-\omega_1) m_{11}^{(1)} \\
    m_{20}^{(1)*} - m_{02}^{(1)*}
      & = (1-\omega_1) (m_{20}^{(1)} - m_{02}^{(1)}) \\
    m_{20}^{(1)*} + m_{02}^{(1)*}
      & =  \frac{2 m_{00}}{3} \\
    m_{21}^{(1)*} & = \theta m_{01}^{(1)*} \\
    m_{12}^{(1)*} & = \theta m_{10}^{(1)*} \\
    m_{22}^{(1)*} & = \theta (m_{20}^{(1)*} + m_{02}^{(1)*}) = \theta \frac{2 m_{00}^{(1)}}{3}
  \end{aligned}
\end{equation}
And finally, inserting the second order terms into the first three equations
\begin{equation}
  \begin{aligned}
    m_{11}^{(2)*} - \frac{ m_{10}^{(1)*}m_{01}^{(1)*}}{m_{00}^{(0)*}} & = (1-\omega_1)m_{11}^{(2)} - \frac{ m_{10}^{(1)}m_{01}^{(1)}}{m_{00}^{(0)}} \\
    m_{20}^{(2)*} - \frac{ m_{10}^{(1)*2}}{m_{00}^{(0)*}} - m_{02}^{(2)} + \frac{ m_{01}^{(1)*2}}{m_{00}^{(0)*}}
      & = (1-\omega_1) (m_{20}^{(2)} - \frac{ m_{10}^{(1)2}}{m_{00}^{(0)}} - K_{02}) \\
    m_{20}^{(2)*} - \frac{ m_{10}^{(1)*2}}{m_{00}^{(0)*}} + m_{02}^{(2)*} - \frac{ m_{01}^{(1)*2}}{m_{00}^{(0)*}}
      & =  \frac{2 m_{00}^{(2)}}{3} \\
  \end{aligned}
\end{equation}
