% !TEX root = ../notes.tex

Before starting to analyse the terms, we need to set up
\subsection{Absence of sources}
\label{sub:Absence of sources}
The described cumulant method shall not have any sources and sinks and be incompressible. Hence
\begin{equation}
    \rho  = m_{00} = m_{00}^*
\end{equation}
and thus
\begin{equation}
    \rho = m_{00}^{(p)} = m_{00}^{*(p)}
\end{equation}

Velocity is no conserved quantity in our setting, because forcing terms like gravity can alter it.

\subsection{Zeroth order moments are constant in space}
\label{sub:Zeroth order moments are constant in space}
Zeroth order terms are independent of our expansion parameter $\epsilon$ and thus have to be constant in space. Hence
\begin{equation}
  \partial_x m_{\alpha\beta}^{(0)} = 0
\end{equation}
\begin{equation}
  \partial_y m_{\alpha\beta}^{(0)} = 0
\end{equation}


\subsection{Collision invariants}
\label{sub:Collision invariants}
As the zeroth order terms of the expansion have to be constant in space, we get
\begin{equation}
  \label{eq:first order collision invariant}
  \begin{aligned}
    m_{\alpha\beta}^{(1)}
    & = m_{\alpha\beta}^{*(1)}
    - \partial_x m_{(\alpha+1)\beta}^{*(0)}
    - \partial_y m_{\alpha(\beta+1)}^{*(0)} \\
    & = m_{\alpha\beta}^{*(1)}
  \end{aligned}
\end{equation}
Thus, all moments are collision invariants at zeroth,~\eqref{eq:zeroth order in epsilon}, and first order,~\eqref{eq:first order collision invariant}, in $\epsilon$.
