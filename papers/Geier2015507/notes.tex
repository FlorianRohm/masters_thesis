\documentclass{article}
\usepackage[utf8]{inputenc}
\usepackage{algorithm}
\usepackage{algorithmic}
\usepackage{amsfonts}
\usepackage{amsmath}
\usepackage{amssymb}
\usepackage{amsthm}
\usepackage{appendix}

\usepackage{bm}
\usepackage{booktabs}

\usepackage{framed}

\usepackage{graphicx}

\usepackage{hyperref}

\usepackage{mathtools}
\usepackage{multicol}
\usepackage{multirow}

\usepackage{tikz}
\usepackage{todonotes}

\usepackage{wrapfig}

\usepackage{xcolor}
\usepackage{xparse}


\usetikzlibrary{shapes,snakes}
\usetikzlibrary{positioning}


\renewcommand\thesection{\arabic{section}}

%Blindtext
\newcommand{\lorem}{Lorem ipsum dolor sit amet, consectetuer adipiscing
elit. Aenean commodo ligula eget dolor. Aenean massa. Cum sociis natoque
penatibus et magnis dis parturient montes, nascetur ridiculus mus. Donec
quam felis, ultricies nec, pellentesque eu, pretium quis, sem. Nulla
consequat massa quis enim. Donec pede justo, fringilla vel, aliquet nec,
vulputate eget, arcu. In enim justo, rhoncus ut, imperdiet a, venenatis
vitae, justo. Nullam dictum felis eu pede mollis pretium. Integer tincidunt.
Cras dapibus. Vivamus elementum semper nisi. Aenean vulputate eleifend
tellus. Aenean leo ligula, porttitor eu, consequat vitae, eleifend ac,
enim. Aliquam lorem ante, dapibus in, viverra quis, feugiat a, tellus.
Phasellus viverra nulla ut metus varius laoreet. Quisque rutrum. Aenean
imperdiet. Etiam ultricies nisi vel augue. Curabitur ullamcorper ultricies
nisi. Nam eget dui.}


%Arcusfunktionen
\newcommand{\arccsc}{\operatorname{arccsc}}
\newcommand{\arcsec}{\operatorname{arcsec}}
\newcommand{\arccot}{\operatorname{arccot}}

%Hyperbelfunktionen
\newcommand{\csch}{\operatorname{csch}}
\newcommand{\sech}{\operatorname{sech}}

%Areafunktionen
\newcommand{\arsinh}{\operatorname{arsinh}}
\newcommand{\arcosh}{\operatorname{arcosh}}
\newcommand{\artanh}{\operatorname{artanh}}
\newcommand{\arcsch}{\operatorname{arcsch}}
\newcommand{\arsech}{\operatorname{arsech}}
\newcommand{\arcoth}{\operatorname{arcoth}}

%Zahlenmengen
\newcommand{\N}{\ensuremath{\mathbb{N}}}
\newcommand{\R}{\ensuremath{\mathbb{R}}}
\newcommand{\Q}{\ensuremath{\mathbb{Q}}}
\newcommand{\C}{\ensuremath{\mathbb{C}}}
\newcommand{\RtoN}{\ensuremath{\R^n}}


\newcommand{\defined}{\mathrel{\mathop:}=}
\newcommand{\defines}{\mathrel{\mathop=}:}

\newcommand{\argmin}{\operatornamewithlimits{argmin}}
\newcommand{\argmax}{\operatornamewithlimits{argmax}}
\newcommand{\infimum}{\operatornamewithlimits{inf}}
\newcommand{\supremum}{\operatornamewithlimits{sup}}
\newcommand{\nin}{\notin}
\newcommand{\overbar}[1]{\mkern~1.5mu\overline{\mkern-1.5mu#1\mkern-1.5mu}\mkern~1.5mu}
\newcommand{\my}{\mu}
\newcommand{\ny}{\nu}
\newcommand{\boldepsilon}{\ensuremath{\boldsymbol\varepsilon}}
\newcommand{\op}{\operatorname}

%Helfer
\newcommand{\dotProd}[2]{\ensuremath{\langle #1,#2\rangle}}
\newcommand{\inAngles}[1]{\ensuremath{\langle #1\rangle}}

\newcommand{\mathMode}[1]{$ #1 $}
\newcommand{\abs}[1]{\left| #1 \right|}
\newcommand{\norm}[1]{ \left\Vert~#1 \right\Vert}
\newcommand{\newlinedouble}{\newline\newline}


%%%%%%%%%%%%%%%%%%%%%%%%%%%%%%%%%%%%%%%%%%%%%%%%%%%%%%%%%%%%%%%%%%%%%%%%%%%%%%
%                        Definitionen und Sätze                              %
%%%%%%%%%%%%%%%%%%%%%%%%%%%%%%%%%%%%%%%%%%%%%%%%%%%%%%%%%%%%%%%%%%%%%%%%%%%%%%



\DeclareDocumentCommand{\highlight}{ O{red} O{20} m }{
	\colorbox{#1!#2}{$\displaystyle#3$}
}


\newtheorem{mydef}{Definition}[subsection]
\newtheorem{mysatz}{Satz}[subsection]
\newtheorem{mylemma}{Lemma}[subsection]
\newtheorem{mybeispiel}{Beispiel}[subsection]
\newtheorem{mybemerkung}{Bemerkung}[subsection]
\newcommand{\myproof}[1]{
\begin{proof}[Beweis]
#1
\end{proof}
}
                                                                             %
\newtheoremstyle{definition}                                                 %
  {0.3cm}                 %Space above                                       %
  {0.3cm}                 %Space below                                       %
  {\normalfont}           %Body font                                         %
  {}                      %Indent amount                                     %
  {\normalfont\bfseries}  %Thm head font                                     %
  {}                      %Punctuation after thm head                        %
  {\newline}              %Space after thm head                              %
  {\thmname{#1}\thmnumber{ #2}: \thmnote{ #3}}                               %
                          %Thm head spec (can be left empty, meaning         %
                          %`normal')                                         %
                                                                             %
\newenvironment{defshaded}{                                                  %
\def\FrameCommand{\fcolorbox{defframecolor}{defshadecolor}}                  %
\MakeFramed~{\FrameRestore}}                                                 %
{\endMakeFramed}                                                             %
                                                                             %
\newenvironment{cdef}[1][]{\definecolor{defshadecolor}{RGB}{240,240,240}     % oder 224,255,255?
\definecolor{defframecolor}{RGB}{240,240,240}                                %
                                                                             %
\begin{defshaded}\begin{wdef}[#1]\hspace*{0.15cm}}{\end{wdef}\end{defshaded}}%
                                                                             %
    \theoremstyle{definition}                                                %
	\newtheorem{wdef}{Definition}[subsection]                                           %
	                                                                         %
%%%%%%%%%%%%%%%%%%%%%%%%%%%%%%%%%%%%%%%%%%%%%%%%%%%%%%%%%%%%%%%%%%%%%%%%%%%%%%
                                                                             %
\newenvironment{satzshaded}{                                                 %
\def\FrameCommand{\fcolorbox{satzframecolor}{satzshadecolor}}                %
\MakeFramed~{\FrameRestore}}                                                 %
{\endMakeFramed}                                                             %
                                                                             %
\newenvironment{csatz}[1][]{\definecolor{satzshadecolor}{RGB}{240,240,240}   %
\definecolor{satzframecolor}{RGB}{240,240,240}                               %
                                                                             %
\begin{satzshaded}\begin{satz}[#1]\hspace*{0.15cm}}{\end{satz}\end{satzshaded}}%
                                                                             %
    \theoremstyle{definition}                                                %
	\newtheorem{satz}{Theorem}[subsection]                                               %
                                                                             %
%%%%%%%%%%%%%%%%%%%%%%%%%%%%%%%%%%%%%%%%%%%%%%%%%%%%%%%%%%%%%%%%%%%%%%%%%%%%%%

%%%%%%%%%%%%%%%%%%%%%%%%%%%%%%%%%%%%%%%%%%%%%%%%%%%%%%%%%%%%%%%%%%%%%%%%%%%%%%
%                        Nummerierung	                            		 %
%%%%%%%%%%%%%%%%%%%%%%%%%%%%%%%%%%%%%%%%%%%%%%%%%%%%%%%%%%%%%%%%%%%%%%%%%%%%%%
																			 %
\numberwithin{equation}{subsection}											 %
																			 %
%%%%%%%%%%%%%%%%%%%%%%%%%%%%%%%%%%%%%%%%%%%%%%%%%%%%%%%%%%%%%%%%%%%%%%%%%%%%%%



\title{Notes concerning the paper}
\author{Florian Rohm}


\begin{document}
\maketitle
\tableofcontents
\newpage

\section{Definitions}
\label{sec:Definitions}
In the following, only lattices with even cell spacing $c$ in $x$ and $y$ direction are considered.

\subsection{Particle distribution function}
\label{sub:Particle distribution function}

Continuous definition of $f$ using the Diraque delta
\begin{equation}
  \label{eq:Definition of f xi}
  f(\xi) = \sum_{i,j} f_{ij}\delta(ic - \xi_1)\delta(jc - \xi_2).
\end{equation}
Centralized $f$, where $u=\begin{pmatrix}u_1 \\ u_2\end{pmatrix}$ is the macroscopic speed in the point.
\begin{equation}
  \label{eq:Definition of f_c xi}
  f_c(\xi) = f(\xi-u) = \sum_{i,j} f_{ij}\delta(ic - (\xi_1-u_1))\delta(jc - (\xi_2-u_2)).
\end{equation}

\subsection{Laplace Transform}
\label{sub:Laplace Transform}

Laplace transformation definition:
\begin{equation}
  \label{eq:Definition of Laplace}
  \mathcal{L}[g](\Xi) = \int_{-\infty}^\infty g(\xi) e^{-\Xi \cdot \xi}d\xi.
\end{equation}
%
Formulation of the Laplace transform of $f$
\begin{equation}
  \label{eq:Definition of F}
  \begin{aligned}
    F(\Xi_1, \Xi_2) & = \mathcal{L}[f](\Xi) = \int_{-\infty}^\infty f(\xi) e^{-\Xi \cdot \xi}d\xi \\
     & = \sum_{i,j}f_{ij} e^{-\Xi_1 ic} e^{-\Xi_2 jc}.
  \end{aligned}
\end{equation}
%
Definition of the Laplace transform of centralized $f$.
\begin{equation}
  \label{eq:Definition of centralized F}
  \begin{aligned}
    F_c(\Xi_1, \Xi_2) & = \mathcal{L}[f_c](\Xi) = \mathcal{L}[f(\cdot - u)](\Xi) \\
    & = \int_{-\infty}^\infty f(\xi-u) e^{-\Xi \cdot \xi}d\xi \\
    & = \int_{-\infty}^\infty f(\xi) e^{-\Xi \cdot (\xi+u)}d\xi \\
    & = \sum_{i,j}f_{ij} e^{-\Xi_1 (ic+u_1)} e^{-\Xi_2 (jc+u_2)} \\
    & = e^{-\Xi_1 u_1}e^{-\Xi_2 u_2}\sum_{i,j}f_{ij} e^{-\Xi_1 ic} e^{-\Xi_2 jc}\\
    & = e^{-\Xi_1 u_1}e^{-\Xi_2 u_2} F(\Xi_1, \Xi_2).
  \end{aligned}
\end{equation}
%
\subsection{Moments}
\label{sub:Moments}

Definition of the moments of $f$
\begin{equation}
  \label{eq:Definition of moments}
  m_{\alpha\beta} = \sum_{ij} i^\alpha j^\beta f_{ij}
\end{equation}
%
Definition of the centralized moments of $f$
\begin{equation}
  \begin{aligned}
    \label{eq:Definition of centralized moments}
    c_{\alpha\beta} & = \sum_{ij} {\left(i-\frac{m_{10}}{m_{00}}\right)}^\alpha {\left(j-\frac{m_{01}}{m_{00}}\right)}^\beta f_{ij} \\
    & = \sum_{ij} {\left(i-\frac{u_1}{c}\right)}^\alpha {\left(j-\frac{u_2}{c}\right)}^\beta f_{ij}
  \end{aligned}
\end{equation}

\subsection{Taylor expansions}
\label{sub:Taylor expansions}

\subsubsection{Moment generating function}
\label{subs:Moment generating function}

Taylor expansion of $F$
\begin{equation}
  \label{eq: taylor of F}
  \begin{aligned}
    F(\Xi_1, \Xi_2) & = \sum_{\alpha,\beta} \frac{1}{\alpha!\beta!} \frac{\partial^\alpha\partial^\beta}{{(\partial \Xi_1)}^\alpha{(\partial \Xi_2)}^\beta} F(\Xi_1, \Xi_2)\Bigr|_{\Xi_1=\Xi_2 = 0} \Xi_1^\alpha \Xi_2^\beta \\
    & = \sum_{\alpha,\beta} \frac{1}{\alpha!\beta!} \frac{\partial^\alpha\partial^\beta}
      {{(\partial \Xi_1)}^\alpha{(\partial \Xi_2)}^\beta}  \sum_{i,j}f_{ij} e^{-\Xi_1 ic} e^{-\Xi_2 jc} \Bigr|_{\Xi_1=\Xi_2 = 0} \Xi_1^\alpha \Xi_2^\beta \\
    & = \sum_{\alpha,\beta} \frac{1}{\alpha!\beta!} {(-c)}^{\alpha+\beta}  \sum_{i,j} i^\alpha j^\beta f_{ij} e^{-\Xi_1 ic} e^{-\Xi_2 jc} \Bigr|_{\Xi_1=\Xi_2 = 0} \Xi_1^\alpha \Xi_2^\beta \\
    & = \sum_{\alpha,\beta} \frac{1}{\alpha!\beta!} {(-c)}^{\alpha+\beta}  \sum_{i,j} i^\alpha j^\beta f_{ij} \Xi_1^\alpha \Xi_2^\beta \\
    & = \sum_{\alpha,\beta} \frac{1}{\alpha!\beta!} {(-c)}^{\alpha+\beta}  m_{\alpha\beta} \Xi_1^\alpha \Xi_2^\beta
  \end{aligned}
\end{equation}
%
Hence,
\begin{equation}
  \label{eq:alternative representation of moments}
  m_{\alpha\beta} = {(-c)}^{-\alpha-\beta} \frac{\partial^\alpha\partial^\beta}{{(\partial \Xi_1)}^\alpha{(\partial \Xi_2)}^\beta} F(\Xi_1, \Xi_2)\Bigr|_{\Xi_1=\Xi_2 = 0}
\end{equation}
and $F(\Xi_1, \Xi_2)$ is the moment generating function of $f(\xi_1, \xi_2)$.

\subsubsection{Centralized moment generating function}
\label{subs:Centralized moment generating function}

Taylor expansion of $F_c$
\begin{equation}
  \label{eq: taylor of Fc}
  \begin{aligned}
    F_c(\Xi_1, \Xi_2) & = \sum_{\alpha,\beta} \frac{1}{\alpha!\beta!} \frac{\partial^\alpha\partial^\beta}{{(\partial \Xi_1)}^\alpha{(\partial \Xi_2)}^\beta} F_c(\Xi_1, \Xi_2)\Bigr|_{\Xi_1=\Xi_2 = 0} \Xi_1^\alpha \Xi_2^\beta \\
    & = \sum_{\alpha,\beta} \frac{1}{\alpha!\beta!}
      \frac{\partial^\alpha\partial^\beta}{{(\partial \Xi_1)}^\alpha{(\partial \Xi_2)}^\beta} \sum_{i,j}f_{ij} e^{-\Xi_1 (ic+u_1)} e^{-\Xi_2 (jc+u_2)} \Bigr|_{\Xi_1=\Xi_2 = 0} \Xi_1^\alpha \Xi_2^\beta \\
    & = \sum_{\alpha,\beta} \frac{1}{\alpha!\beta!} \sum_{i,j} {(ic+u_1)}^\alpha {(jc+u_2)}^\beta f_{ij} e^{-\Xi_1 ic} e^{-\Xi_2 jc} \Bigr|_{\Xi_1=\Xi_2 = 0} \Xi_1^\alpha \Xi_2^\beta \\
    & = \sum_{\alpha,\beta} \frac{1}{\alpha!\beta!} {(-c)}^{\alpha+\beta}
      \sum_{i,j} {\left(i+\frac{u_1}{c}\right)}^\alpha {\left(j+\frac{u_2}{c}\right)}^\beta f_{ij} e^{-\Xi_1 ic} e^{-\Xi_2 jc} \Bigr|_{\Xi_1=\Xi_2 = 0} \Xi_1^\alpha \Xi_2^\beta \\
    & = \sum_{\alpha,\beta} \frac{1}{\alpha!\beta!} {(-c)}^{\alpha+\beta}
      \sum_{i,j} {\left(i+\frac{u_1}{c}\right)}^\alpha {\left(j+\frac{u_2}{c}\right)}^\beta f_{ij} \Xi_1^\alpha \Xi_2^\beta \\
    & = \sum_{\alpha,\beta} \frac{1}{\alpha!\beta!} {(-c)}^{\alpha+\beta}  c_{\alpha\beta} \Xi_1^\alpha \Xi_2^\beta
  \end{aligned}
\end{equation}
%
Hence,
\begin{equation}
  \label{eq:alternative representation of central moments}
  c_{\alpha\beta} = {(-c)}^{-\alpha-\beta} \frac{\partial^\alpha\partial^\beta}{{(\partial \Xi_1)}^\alpha{(\partial \Xi_2)}^\beta} F_c(\Xi_1, \Xi_2)\Bigr|_{\Xi_1=\Xi_2 = 0}
\end{equation}


\subsection{Cumulants}
\label{sub:Cumulants}

Cumulants are defined as the coefficients of the Taylor-expansion of the logarithm of the moment generating function:
\begin{equation}
  \label{eq:Definition of cumulants}
  \kappa_{\alpha\beta} = {(-c)}^{-\alpha-\beta} \frac{\partial^\alpha\partial^\beta}{{(\partial \Xi_1)}^\alpha{(\partial \Xi_2)}^\beta} \ln(F(\Xi_1, \Xi_2))\Bigr|_{\Xi_1=\Xi_2 = 0}
\end{equation}

\section{Moments and physical quantities}
\label{sec:Moments and physical quantities}

The following connections can be made: \todo{Look up in Harris}
\begin{table}

\end{table}

\section{Lattice Boltzmann Equation for D2Q9}
\label{sec:Lattice Boltzmann Equation for D2Q9}

With $c$ defined as the cell width, velocities are given by
\begin{equation}
  c_{ij}= c\begin{pmatrix}i \\ j\end{pmatrix}, \quad i,j\in \{-1, 0, 1\}
\end{equation}
and the lattice Boltzmann equation is given by
\begin{equation}
  f_{ij}(t + \Delta t, x + i \Delta x , y + j \Delta y) = f^*_{ij}(t,x,y), \quad i,j\in \{-1, 0, 1\}
\end{equation}
or equivalently by shifting $x$ and $y$
\begin{equation}
  \label{eq:Lattice Boltzmann Equation}
  f_{ij}(t + \Delta t, x, y) = f^*_{ij}(t,x - i\Delta x , y - j\Delta y), \quad i,j\in \{-1, 0, 1\}
\end{equation}
where $f_{ij}$ are the pre-collission distributions and $f^*_{ij}$ the post collission distributions.

\subsection{Aliasing of moments}
\label{sub:Aliasing of moments}

As the D2Q9 only has $9$ independant velocities, the moment-space spanned by those can only host $9$ independant moments.
In D2Q9, the moments are defined, according to~\eqref{eq:Definition of moments} as
\begin{equation*}
  m_{\alpha\beta} = \sum_{i,j \in \{-1,0,1\}} i^\alpha j^\beta f_{ij}.
\end{equation*}
Hence
\begin{equation}
  \sum_{i,j \in \{-1,0,1\}} i^\alpha j^\beta f_{ij} = \sum_{i,j \in \{-1,0,1\}} i^{(\alpha+2k)} j^\beta f_{ij}
\end{equation}
for $\alpha\neq 0$ and
\begin{equation}
  \sum_{i,j \in \{-1,0,1\}} i^\alpha j^\beta f_{ij} = \sum_{i,j \in \{-1,0,1\}} i^\alpha j^{(\beta+2l)} f_{ij}
\end{equation}
for $\beta\neq 0$. According to this, the list of moments is as follows

\begin{table} [h!]
  \centering
  \begin{tabular}{c cc ccc cc c}
    \toprule
    \multicolumn{9}{c}{independant moments}  \\
    \cmidrule(lr){1-9} \\
    $m_{00}$   &   $m_{10}$ & $m_{01}$   &   $m_{11}$ & $m_{20}$ & $m_{02}$   &   $m_{21}$ & $m_{12}$   &   $m_{22}$ \\
    \bottomrule
  \end{tabular}
  \newline
  \vspace*{.5 cm}
  \newline
  \begin{tabular}{ccccc c}
    \toprule
    \multicolumn{6}{c}{dependant moments}   \\
    \cmidrule(lr){1-5}   \\
    $m_{30}$   & $m_{03}$   & $m_{31}$  & $m_{13}$  & $m_{40}$  & \ldots \\
    = $m_{10}$ & = $m_{01}$ & = $m_{11}$ & = $m_{11}$ & = $m_{20}$ & \\
    \bottomrule
  \end{tabular}
  \caption{D2Q9 moments}\label{table:D2Q9 moments}
\end{table}


\section{Transformations}
\label{sec:Transformations}

\subsection{Cumulants from moments}
\label{sub:Cumulants from moments}

We seek a representation of the cumulants $\kappa_{\alpha\beta}$ in terms of moments, as the later ones are much more easy to compute, cfg.~\eqref{eq:Definition of centralized moments}. This link can be established with the chain rule for differentiaton.
From~\eqref{eq:alternative representation of moments} and~\eqref{eq:Definition of cumulants} follows
\begin{equation}
  \kappa_{00} = \ln(F(0,0)) = \ln(m_{00}) = \ln(\rho).
\end{equation}
For brevity, the following notations are introduced
\begin{equation}
  \label{eq:abbreviations for deriving cumulants from moments}
  \begin{aligned}
    F & \defined F(\Xi_1, \Xi_2) \\
    \Xi &\defined \begin{pmatrix}\Xi_1 \\ \Xi_2  \end{pmatrix} \\
    \partial_i &\defined \frac{\partial}{\partial \Xi_i}
  \end{aligned}
\end{equation}

\subsubsection{Cumulants from moments}
\label{subs:Cumulants from moments}

\begin{equation*}
  \begin{aligned}
    \kappa_{10} & = {(-c)}^{-1} \left.\partial_1 \ln(F) \right|_{\Xi = 0} \\
    & = {(-c)}^{-1} \left. \frac{1}{F} \partial_1 F \right|_{\Xi = 0} \\
    & = \frac{m_{10}}{m_{00}}
  \end{aligned}
\end{equation*}
The cumulant $\kappa_{01}$ follows from symmetrie and is listed in~\ref{subs:Listing the cumulants from moments}

\begin{align*}
  \phantom{\kappa_{00}}
  &\begin{aligned}
  \nonumber
    \mathllap{\kappa_{11}} & = {(-c)}^{-2} \left.\partial_1 \partial_2 \ln(F) \right|_{\Xi = 0} \\
    & = {(-c)}^{-2} \left.\partial_2 \left( \frac{1}{F} \partial_1 F \right) \right|_{\Xi = 0} \\
    & = {(-c)}^{-2} \left.\left( -\frac{1}{F^2} \partial_2 F \partial_1 F + \frac{1}{F} \partial_1\partial_2 F \right) \right|_{\Xi = 0} \\
    & = {(-c)}^{-2} \left(
      - \frac{1}{m_{00}} \frac{m_{01}}{{(-c)}^{-1}} \frac{m_{10}}{{(-c)}^{-1}}
      + \frac{1}{m_{00}} \frac{m_{11}}{{(-c)}^{-2}}
      \right) \\
    & = \frac{m_{11}}{m_{00}} - \frac{m_{10}m_{01}}{m_{00}^2}
  \end{aligned} \\
  \\
  &\begin{aligned}
  \nonumber
  \mathllap{\kappa_{20}} & = {(-c)}^{-2} \left.\partial_1^2 \ln(F) \right|_{\Xi = 0} \\
  & = {(-c)}^{-2} \left.\partial_1 \left( \frac{1}{F} \partial_1 F \right) \right|_{\Xi = 0} \\
  & = {(-c)}^{-2} \left.\left( -\frac{1}{F^2} {(\partial_1 F)}^2 + \frac{1}{F} \partial_1^2 F \right) \right|_{\Xi = 0} \\
  & = {(-c)}^{-2} \left(
    - \frac{1}{m_{00}^2} \frac{m_{10}^2}{{(-c)}^{-2}}
    + \frac{1}{m_{00}} \frac{m_{20}}{{(-c)}^{-2}}
    \right)\\
  & = \frac{m_{20}}{m_{00}} - \frac{m_{10}^2}{m_{00}^2}
  \end{aligned} \\
  \\
  &\begin{aligned}
  \nonumber
  \mathllap{\kappa_{21}} & = {(-c)}^{-3} \left. \partial_2 \partial_1^2 \ln(F) \right|_{\Xi = 0} \\
  & = {(-c)}^{-3} \left.\left(\partial_2\left( -\frac{1}{F^2} {(\partial_1 F)}^2 + \frac{1}{F} \partial_1^2 F \right)\right) \right|_{\Xi = 0} \\
  & = {(-c)}^{-3} \left.\left(
      \frac{2}{F^3} \partial_2 F {(\partial_1 F)}^2
    - \frac{1}{F^2} \partial_1 F \partial_1 \partial_2 F
    - \frac{1}{F^2} \partial_2 F \partial_1^2 F
    - \frac{1}{F} \partial_2 \partial_1^2 F
    \right)\right|_{\Xi = 0}\\
  & = {(-c)}^{-3} \left(
      \frac{2}{m_{00}^3} \frac{m_{01}}{{(-c)}^{-1}} \frac{m_{10}^2}{{(-c)}^{-2}}
    - \frac{1}{m_{00}^2} \frac{m_{10}}{{(-c)}^{-1}} \frac{m_{11}}{{(-c)}^{-2}}
    - \frac{1}{m_{00}^2} \frac{m_{01}}{{(-c)}^{-1}} \frac{m_{20}}{{(-c)}^{-2}}
    - \frac{1}{m_{00}}   \frac{m_{21}}{{(-c)}^{-3}}
    \right)\\
  & = \frac{m_{21}}{m_{00}} - 2\frac{m_{20}m_{01}}{m_{00}^2}
       - 2\frac{m_{10}m_{11}}{m_{00}^2} - 2\frac{m_{10}^2 m_{01}}{m_{00}^3}
  \end{aligned}\\
  \\
  &\begin{aligned}
  \nonumber
  \mathllap{\kappa_{22}} & = {(-c)}^{-4} \left. \partial_2^2 \partial_1^2 \ln(F) \right|_{\Xi = 0} \\
  & = {(-c)}^{-3} \left.\left(\partial_2\left(
      \frac{2}{F^3} \partial_2 F {(\partial_1 F)}^2
    - \frac{1}{F^2} \partial_1 F \partial_1 \partial_2 F
    - \frac{1}{F^2} \partial_2 F \partial_1^2 F
    - \frac{1}{F} \partial_2 \partial_1^2 F
    \right)\right)\right|_{\Xi = 0}\\
  %
  & = {(-c)}^{-4} \bigg(
    - \frac{6}{F^4} {(\partial_2 F)}^2 {(\partial_1 F)}^2
    + \frac{2}{F^3} \partial_2^2 F {(\partial_1 F)}^2
    + \frac{4}{F^3} \partial_1 F \partial_2 F \partial_1 \partial_2 F \\
  & \qquad\qquad\quad
    + \frac{4}{F^3} \partial_1 F \partial_2 F \partial_1 \partial_2 F
    - \frac{2}{F^2} {(\partial_1 \partial_2 F)}^2
    - \frac{2}{F^2} \partial_1 F \partial_1 \partial_2^2 F \\
  & \qquad\qquad\quad
    + \frac{2}{F^3} {(\partial_2 F)}^2  \partial_1^2 F
    - \frac{1}{F^2} \partial_1^2 F \partial_2^2 F
    - \frac{1}{F^2} \partial_2 F \partial_2 \partial_1^2 F \\
  & \qquad\qquad\quad
    - \frac{1}{F^2} \partial_2 F \partial_2 \partial_1^2 F
    + \frac{1}{F}   \partial_1^2 \partial_2^2 F
    \bigg)\bigg|_{\Xi = 0} \\
  %
  & = \frac{1}{m_{00}^4} m_{22} \\
  &\quad
    + \frac{1}{m_{00}^3}
      \left(
        2 m_{10}^2 m_{02} + 4 m_{10}m_{11}m_{01}
        + 4 m_{10}m_{11}m_{01} + 2 m_{20}m_{01}^2
      \right) \\
  &\quad
    - \frac{1}{m_{00}^2}
      \left(
        2 m_{11}^2 + 2 m_{10}m_{12} + m_{20}m_{02} + m_{21}m_{01} + m_{21}m_{01}
      \right) \\
  &\quad
    + \frac{1}{m_{00}} m_{22}\\
  %
  & = \frac{1}{m_{00}} m_{22} \\
    &\quad
    - \frac{1}{m_{00}^2}
    \left(
       2 m_{10}m_{12}  + m_{21}m_{01} + m_{21}m_{01} + 2 m_{11}^2 + m_{20}m_{02}
    \right) \\
    &\quad
    + \frac{2}{m_{00}^3}
      \left(
        m_{10}^2 m_{02} + 4 m_{10}m_{11}m_{01} + m_{20}m_{01}^2
      \right)\\
    &\quad
       + \frac{1}{m_{00}^4} m_{22}
  \end{aligned}
\end{align*}

\subsubsection{Listing the cumulants from moments}
\label{subs:Listing the cumulants from moments}
\begin{align}
  \kappa_{00} & = m_{00} \\
  \kappa_{10} & = \frac{m_{10}}{m_{00}} \\
  \kappa_{01} & = \frac{m_{01}}{m_{00}} \\
  \kappa_{11} & = \frac{m_{11}}{m_{00}} - \frac{m_{10}m_{01}}{m_{00}^2} \\
  \kappa_{20} & = \frac{m_{20}}{m_{00}} - \frac{m_{10}^2}{m_{00}^2} \\
  \kappa_{02} & = \frac{m_{02}}{m_{00}} - \frac{m_{01}^2}{m_{00}^2} \\
  \kappa_{21} & = \frac{m_{21}}{m_{00}} - 2\frac{m_{20}m_{01}}{m_{00}^2}
            - 2\frac{m_{10}m_{11}}{m_{00}^2} - 2\frac{m_{10}^2 m_{01}}{m_{00}^3} \\
  \kappa_{12} & = \frac{m_{12}}{m_{00}} - 2\frac{m_{10}m_{02}}{m_{00}^2}
            - 2\frac{m_{11}m_{01}}{m_{00}^2} - 2\frac{m_{10} m_{01}^2}{m_{00}^3} \\
  \kappa_{22} & = \frac{1}{m_{00}} m_{22} \\
    &\quad
    - \frac{1}{m_{00}^2}
    \left(
       2 m_{10}m_{12}  + m_{21}m_{01} + m_{21}m_{01} + 2 m_{11}^2 + m_{20}m_{02}
    \right) \\
    &\quad
    + \frac{2}{m_{00}^3}
      \left(
        m_{10}^2 m_{02} + 4 m_{10}m_{11}m_{01} + m_{20}m_{01}^2
      \right)\\
    &\quad
       + \frac{6}{m_{00}^4} m_{10}^2 m_{01}^2
\end{align}
For more convenient reading, the normalized moments $\hat{m}_{\alpha\beta} = \frac{m_{\alpha\beta}}{m_{00}} $ are introduced
\begin{align}
  \kappa_{00} & = m_{00} \\
  \kappa_{10} & = \hat{m}_{10} \\
  \kappa_{01} & = \hat{m}_{01} \\
  \kappa_{11} & = \hat{m}_{11} - \hat{m}_{10}\hat{m}_{01} \\
  \kappa_{20} & = \hat{m}_{20} - \hat{m}_{10}^2 \\
  \kappa_{02} & = \hat{m}_{02} - \hat{m}_{01}^2 \\
  \kappa_{21} & = \hat{m}_{21} - 2 \hat{m}_{20} \hat{m}_{01}
            - 2 \hat{m}_{10} \hat{m}_{11} - 2 \hat{m}_{10}^2\hat{m}_{01} \\
  \kappa_{12} & = \hat{m}_{12} - 2 \hat{m}_{10} \hat{m}_{02}
            - 2 \hat{m}_{11} \hat{m}_{01} - 2 \hat{m}_{10}\hat{m}_{01}^2 \\
  \kappa_{22} & = m_{22} \\
       & \quad - 2 m_{10}m_{12}  - m_{21}m_{01} - m_{21}m_{01} - 2 m_{11}^2 - m_{20}m_{02} \\
       & \quad +  2m_{10}^2 m_{02} + 8 m_{10}m_{11}m_{01} + 2m_{20}m_{01}^2 \\
       & \quad - 6 m_{10}^2 m_{01}^2
\end{align}

\subsection{Central moments from cumulants}
\label{sub:Central moments from cumulants}

This is a bit more complicated, as we can't break down the definition of the central moments to cumulants as easily as for raw moments.
Markus Muhr provides an easy remedy in writing
\begin{equation}
  F_c(\Xi_1, \Xi_2) = \exp(\ln(F(\Xi_1,\Xi_2)) - \Xi_1 u_1 - \Xi_2 u_2).
\end{equation}
In addition to the abbreviations in~\eqref{eq:abbreviations for deriving cumulants from moments}, we define the folowing
\begin{equation}
  \begin{aligned}
    E & \defined \exp(\ln(F(\Xi_1,\Xi_2)) - \Xi_1 u_1 - \Xi_2 u_2) \\
    L_i & \defined \partial_i \ln(F) - u_i
  \end{aligned}
\end{equation}
and therefore
\begin{equation}
  \begin{aligned}
    \partial_i E & = EL_i \\
    \partial_i L_j & = \partial_i\partial_j\ln(F)\\
    L_1\bigr|_{\Xi_1=\Xi_2 = 0} & = \bigg(
      \underbrace{{(-c)}^{-1} \partial_1\ln(F) \bigr|_{\Xi_1=\Xi_2 = 0}}_{=-\kappa_{10}=-\frac{m_{10}}{m_{00}}}
      - \underbrace{{(-c)}^{-1}u_1}_{=-\frac{m_{10}}{m_{00}}} \bigg) = 0 \\
    L_2\bigr|_{\Xi_1=\Xi_2 = 0} & = 0.
  \end{aligned}
\end{equation}

\subsubsection{Revisiting the central moment definition}
\label{subs:Revisiting the central moment definition}
Now, we can revisit the definition of the central moments,~\eqref{eq:alternative representation of central moments}, which now read
\begin{equation}
  \begin{aligned}
    c_{\alpha\beta}= {(-c)}^{-\alpha-\beta} \partial_1^\alpha \partial_2^\beta E\bigr|_{\Xi_1=\Xi_2 = 0}
  \end{aligned}
\end{equation}
%
From there on, we can express
\begin{align*}
  \phantom{c_{00}}
  &\begin{aligned}
  \nonumber
    \mathllap{c_{00}} & = E\bigr|_{\Xi_1=\Xi_2 = 0} = F(0,0) = m_{00}
  \end{aligned}\\
  %
  &\begin{aligned}
  \nonumber
    \mathllap{c_{10}} & = {(-c)}^{-1} \partial_1 E  \bigr|_{\Xi_1=\Xi_2 = 0} \\
    & = {(-c)}^{-1} EL_1  \bigr|_{\Xi_1=\Xi_2 = 0} \\
    & = 0
  \end{aligned}\\
  %
  &\begin{aligned}
  \nonumber
    \mathllap{c_{20}} & = {(-c)}^{-2} \partial_1 (EL_1)  \bigr|_{\Xi_1=\Xi_2 = 0} \\
    & = {(-c)}^{-2} \big( EL_1^2 + E \partial_1^2\ln(F)\big)  \bigr|_{\Xi_1=\Xi_2 = 0} \\
    & = {(-c)}^{-2} \big( 0 + \underbrace{E \bigr|_{\Xi_1=\Xi_2 = 0}}_{=m_{00}}
    \underbrace{\partial_1^2\ln(F) \bigr|_{\Xi_1=\Xi_2 = 0}}_{={(-c)}^{-2}\kappa_{20}} \big) \\
    & = m_{00}\kappa_{20}
  \end{aligned}\\
  %
  &\begin{aligned}
  \nonumber
    \mathllap{c_{11}} & = {(-c)}^{-2} \partial_2 (EL_1) \bigr|_{\Xi_1=\Xi_2 = 0} \\
    & = {(-c)}^{-2} \big( E L_1 L_2 + E \partial_1\partial_2\ln(F) \big)  \bigr|_{\Xi_1=\Xi_2 = 0} \\
    & = m_{00}\kappa_{11}
  \end{aligned}\\
  %
  &\begin{aligned}
  \nonumber
    \mathllap{c_{21}} & = {(-c)}^{-3} \partial_2 \big(EL_1^2 + E \partial_1^2\ln(F)\big)  \bigr|_{\Xi_1=\Xi_2 = 0} \\
    & = {(-c)}^{-3} \big( E L_1^2 L_2 + 2EL_1\partial_1\partial_2\ln(F) \\
      &\qquad\quad\ \ + EL_2 \partial_1^2\ln(F) + E \partial_1^2\partial_2\ln(F) \big)  \bigr|_{\Xi_1=\Xi_2 = 0} \\
    & = m_{00}\kappa_{21}
  \end{aligned}\\
  %
  &\begin{aligned}
  \nonumber
    \mathllap{c_{22}} & = {(-c)}^{-4} \partial_2 \big( E L_1^2 L_2 + 2EL_1\partial_1\partial_2\ln(F) \\
      &\qquad\qquad\ \ + EL_2 \partial_1^2\ln(F) + E \partial_1^2\partial_2\ln(F) \big)  \bigr|_{\Xi_1=\Xi_2 = 0} \\
    & = {(-c)}^{-4} \big(
      E L_1^2 L_2^2 + E L_2^2 \partial_2^2\ln(F) + 2 E L_1 L_2 \partial_1\partial_2 \ln(F) \\
    & \qquad\quad\ \ +
      2 E L_1 L_2 \partial_1\partial_2 \ln(F) + 2 E \partial_1\partial_2 \ln(F) \partial_1\partial_2 \ln(F) + 2 E L_1 \partial_1\partial_2^2 \ln(F) \\
    & \qquad\quad\ \ +
      E L_2^2 \partial_1^2 \ln(F) + E \partial_1^2 \ln(F) \partial_2^2 \ln(F) + E L_2 \partial_1\partial_2^2 \ln(F) \\
    & \qquad\quad\ \ +
      E L_2 \partial_1^2 \partial_2 \ln(F) + E\partial_1^2\partial_2^2 \ln(F)
      \big) \\
    & = m_{00}(\kappa_{22} + 2\kappa_{11}^2 + \kappa_{20}\kappa_{02})
  \end{aligned}
\end{align*}
The missing central moments follow from permutation and will be listed in the following

\subsubsection{Listing the central moments from cumulants}
\label{subs:Listing the central moments from cumulants}

\begin{align}
  \label{eq:all central moments from cumulants}
  c_{00} & = m_{00}\kappa_{00} \\
  c_{10} & = m_{00}\kappa_{10} \\
  c_{01} & = m_{00}\kappa_{01} \\
  c_{20} & = m_{00}\kappa_{20} \\
  c_{02} & = m_{00}\kappa_{02} \\
  c_{11} & = m_{00}\kappa_{11} \\
  c_{21} & = m_{00}\kappa_{21} \\
  c_{12} & = m_{00}\kappa_{12} \\
  c_{22} & = m_{00}(\kappa_{22} + 2\kappa_{11}^2 + \kappa_{20}\kappa_{02})
\end{align}
As we can se here, cumulants only differ, apart from scaling, from central moments in order higher than $3$.

\subsection{Cumulants from central moments}
\label{sub:Cumulants from central moments}

For implementing the cumulant LBM, we also need to get cumulants from central moments, as we want to have the collision in the cumulant space. Coming from~\ref{subs:Listing the central moments from cumulants} this is an easy task. The first eight equations in~\eqref{eq:all central moments from cumulants} can be inverted straight forward.
Inserting the resulting equations for $\kappa_{11}$, $\kappa_{02}$ and $\kappa_{20}$ in the last equation, we get

\begin{align}
  \label{eq:all cumulants from central moments}
  \kappa_{00} & = \frac{c_{00}}{m_{00}} \\
  \kappa_{10} & = \frac{c_{10}}{m_{00}} \\
  \kappa_{01} & = \frac{c_{01}}{m_{00}} \\
  \kappa_{20} & = \frac{c_{20}}{m_{00}} \\
  \kappa_{02} & = \frac{c_{02}}{m_{00}} \\
  \kappa_{11} & = \frac{c_{11}}{m_{00}} \\
  \kappa_{21} & = \frac{c_{21}}{m_{00}} \\
  \kappa_{12} & = \frac{c_{12}}{m_{00}} \\
  \kappa_{22} & = \frac{c_{22}}{m_{00}} - 2{\left(\frac{c_{11}}{m_{00}}\right)}^2 - \frac{c_{20}}{m_{00}}\frac{c_{02}}{m_{00}}
\end{align}

\subsection{Fast central moment transformations}
\label{sub:Fast central moment transformations}

As cumulants of low orders are easily derived from central moments, we seek a fast way to compute them and afterwards get the distribution functions back for streaming.

\subsubsection{Forward transformation}
\label{subs:Forward transformation}

The forward central moment transformation splits the definition for a two way calculation
\begin{equation}
  \begin{aligned}
    c_{\alpha\beta}
    & = \sum_{ij} {\left(i-\frac{u_1}{c}\right)}^\alpha {\left(j-\frac{u_2}{c}\right)}^\beta f_{ij} \\
    & = \sum_{i} {\left(i-\frac{u_1}{c}\right)}^\alpha \underbrace{\sum_{j} {\left(j-\frac{u_2}{c}\right)}^\beta f_{ij}}_{\defines c_{i|\beta}} \\
    & = \sum_{i} {\left(i-\frac{u_1}{c}\right)}^\alpha c_{i|\beta}
  \end{aligned}
\end{equation}
Thus, by first computing the $c_{i|\beta}$ and then the $c_{\alpha\beta}$ we need $6$ instead of $9$ summations. By chaining this technique in 3D, much more computing power is saved.

\subsubsection{Backward transformation}
\label{subs:Backward transformation}

For the inverse transformation, we first analyse the $c_{\alpha\beta}$ and $c_{i|\beta}$ terms:

\begin{equation}
  \begin{aligned}
    \label{eq:fast forward c_alpha beta expanded}
    c_{\alpha\beta} & = \sum_{i} {\left(i-\frac{u_1}{c}\right)}^\alpha c_{i|\beta} \\
    & = {\left(-1-\frac{u_1}{c}\right)}^\alpha c_{-1|\beta} + {\left(-\frac{u_1}{c}\right)}^\alpha c_{0|\beta} + {\left(1-\frac{u_1}{c}\right)}^\alpha c_{1|\beta}
  \end{aligned}
\end{equation}

\begin{equation}
  \begin{aligned}
    \label{eq:fast forward c_i pipe beta expanded}
    c_{i|\beta} & = \sum_{j} {\left(j-\frac{u_2}{c}\right)}^\beta f_{ij} \\
    & = {\left(-1-\frac{u_2}{c}\right)}^\beta f_{i-1} + {\left(-\frac{u_2}{c}\right)}^\beta f_{i0} + {\left(1-\frac{u_2}{c}\right)}^\beta f_{i1}
  \end{aligned}
\end{equation}
With~\eqref{eq:fast forward c_alpha beta expanded}, we can explicitly write
\begin{equation}
  \label{eq:fast forward c_alpha beta matrix}
  \begin{pmatrix}
    c_{0\beta} \\
    c_{1\beta} \\
    c_{2\beta}
  \end{pmatrix}
  =
  \begin{pmatrix}
    1 & 1 & 1 \\
    -1-\frac{u_1}{c} & - \frac{u_1}{c} &   1-\frac{u_1}{c} \\
    {\left(-1-\frac{u_1}{c}\right)}^2 & {\left(\frac{u_1}{c}\right)}^2 &  {\left(1-\frac{u_1}{c}\right)}^2
  \end{pmatrix}
  \begin{pmatrix}
    c_{-1|\beta} \\
    c_{0|\beta} \\
    c_{1|\beta}
  \end{pmatrix}
\end{equation}
and with~\eqref{eq:fast forward c_i pipe beta expanded}
\begin{equation}
  \label{eq:fast forward c_i pipe beta matrix}
  \begin{pmatrix}
    c_{i|0} \\
    c_{i|1} \\
    c_{i|2}
  \end{pmatrix}
  =
  \begin{pmatrix}
    1 & 1 & 1 \\
    -1-\frac{u_2}{c} & - \frac{u_2}{c} &   1-\frac{u_2}{c} \\
    {\left(-1-\frac{u_2}{c}\right)}^2 & {\left(\frac{u_2}{c}\right)}^2 &  {\left(1-\frac{u_2}{c}\right)}^2
  \end{pmatrix}
  \begin{pmatrix}
    f_{i-1} \\
    f_{i0} \\
    f_{i1}
  \end{pmatrix}
\end{equation}
Those systems are now solved via Sage with
\begin{minted}{python}
  u, c = var('u, c')
  M = Matrix([
    [1,1,1],
    [-1-u/c,-u/c, 1-u/c],
    [(-1-u/c)^2, (-u/c)^2, (1-u/c)^2]
    ])
  I = M.inverse().simplify_rational()
  I
\end{minted}
where~\eqref{eq:fast forward c_alpha beta matrix} becomes
\begin{equation}
  \begin{pmatrix}
    c_{-1|\beta} \\
    c_{0|\beta} \\
    c_{1|\beta}
  \end{pmatrix}
  =
  \begin{pmatrix}[1.5]
    -\frac{1}{2}(\frac{u_1}{c} - \frac{u_1}{c}^2) &
    -\frac{1}{2}(1 - 2\frac{u_1}{c}) &
    \frac{1}{2} \\
    1 - {(\frac{u_1}{c})}^2  &
    -2\frac{u_1}{c}  &
    -1 \\
    \frac{1}{2}(\frac{u_1}{c} + {(\frac{u_1}{c})}^2)  &
    \frac{1}{2}(1 + 2\frac{u_1}{c})  &
    \frac{1}{2}
  \end{pmatrix}
  \begin{pmatrix}
    c_{0\beta} \\
    c_{1\beta} \\
    c_{2\beta}
  \end{pmatrix}
\end{equation}
and~\eqref{eq:fast forward c_i pipe beta matrix} analogously
\begin{equation}
  \begin{pmatrix}
    f_{i-1} \\
    f_{i0} \\
    f_{i1}
  \end{pmatrix}
  =
  \begin{pmatrix}[1.5]
    -\frac{1}{2}(\frac{u_2}{c} - \frac{u_2}{c}^2) &
    -\frac{1}{2}(1 - 2\frac{u_2}{c}) &
    \frac{1}{2} \\
    1 - {(\frac{u_2}{c})}^2  &
    -2\frac{u_2}{c}  &
    -1 \\
    \frac{1}{2}(\frac{u_2}{c} + {(\frac{u_2}{c})}^2)  &
    \frac{1}{2}(1 + 2\frac{u_2}{c})  &
    \frac{1}{2}
  \end{pmatrix}
  \begin{pmatrix}
      c_{i|0} \\
      c_{i|1} \\
      c_{i|2}
    \end{pmatrix}
\end{equation}

\section{Collision}
\label{sec:Collision}

\subsection{Maxwell Distribution}
\label{sub:Maxwell Distribution}
The Maxwell distribuion in two dimensions reads \todo{cite Succi, p8}

\begin{equation}
  \label{eq:maxwell distribution raw}
  f_{m, \rho, u, T}^{\text{maxwell}}(\xi) = \rho \frac{m}{2\pi k_B T} \exp \left( - \frac{m\abs{\xi-u}^2}{2 k_B T}\right)
\end{equation}
where
\begin{center}
  \begin{tabular}{@{}ll@{}}
    \toprule
    Symbol & Quantity  \\
    \midrule
    $\xi$  & Microscopic speed  \\
    $\rho$ & Macroscopic density     \\
    $u$    & Macroscopic velocity   \\
    $T$    & Temperature   \\
    $k_B$  & Boltzmann constant \\
    $m$    & Mass of the particles   \\
    \bottomrule
  \end{tabular}
\end{center}
This distribution gives the probability of finding a particle with a certain speed at some point.
\subsubsection{Integration of exponential function}
\label{subs:Integration of exponential function}

Integration is again done via Sage:
\begin{minted}{python}
  from sage.symbolic.integration.integral import indefinite_integral
  from sage.symbolic.integration.integral import definite_integral

  C, u, xi, Xi = var('C, u, xi, Xi')
  assume(C>0)

  F(xi) = exp (-C*xi^2 + (2*C*u - Xi)*xi)

  definite_integral(F(xi), xi, -infinity, infinity).simplify_rational()
\end{minted}
which gives
\begin{equation}
  \label{eq:integrate exponential sage}
  \begin{aligned}
    \int_{-\infty}^{\infty} \exp \left(-C \xi^2 + (2Cu - \Xi)\cdot\xi \right) d\xi
    & = \sqrt{\frac{\pi}{C}}\exp \left(Cu^2 - \Xi u + \frac{1}{4} \frac{\Xi^2}{C}\right) \\
    & = \sqrt{\frac{\pi}{C}}\exp \left( \frac{{(2Cu-\Xi)}^2}{4C}\right)
  \end{aligned}
\end{equation}

\subsubsection{Laplace Transform}
\label{subs:Laplace Transform}
With this work, the two sided Laplace Transform of the Maxwell Distribution reads

\begin{equation}
  \begin{aligned}
    F^{eq}(\Xi) & = \mathcal{L}[f^{\text{maxwell}}](\Xi)
    = \int_{\R^2} \rho \frac{m}{2\pi k_B T} \exp \left( - \frac{m\abs{\xi-u}^2}{2 k_B T}\right) \cdot e^{-\Xi\cdot\xi} d\xi \\
    & = \rho  \frac{m}{2\pi k_B T} \int_{\R^2}
      \exp \left( - \underbrace{\frac{m}{2 k_B T}}_{\defines C} \left( \abs{\xi}^2 - 2\xi\cdot u + \abs{u}^2 \right)\right) \cdot e^{-\Xi\cdot\xi} d\xi \\
    & = \underbrace{\frac{\rho C} {\pi} e^{-C \abs{u}^2}}_{\defines D}
      \int_{\R^2}
      \exp \left( - C\abs{\xi}^2 + (2Cu -\Xi)\cdot\xi \right) d\xi \\
    & =  D
      \int_{\R} \int_{\R}
      \exp \left( - C\abs{\xi_1}^2 + (2Cu_1 -\Xi_1)\cdot\xi_1 \right)
      \exp \left( - C\abs{\xi_2}^2 + (2Cu_2 -\Xi_2)\cdot\xi_2 \right) d\xi_1 d\xi_2 \\
    & = D
      \int_{\R}
      \exp \left( - C\abs{\xi_1}^2 + (2Cu_1 -\Xi_1)\cdot\xi_1 \right) d\xi_1
      \int_{\R}
      \exp \left( - C\abs{\xi_2}^2 + (2Cu_2 -\Xi_2)\cdot\xi_2 \right) d\xi_2 \\
    & = \frac{\rho C} {\pi} e^{-C \abs{u}^2}
      \sqrt{\frac{\pi}{C}}\exp \left( \frac{{(2Cu_1-\Xi_1)}^2}{4C}\right)
      \sqrt{\frac{\pi}{C}}\exp \left( \frac{{(2Cu_2-\Xi_2)}^2}{4C}\right) \\
    & = \rho
      \exp \left( \frac{{(2Cu_1-\Xi_1)}^2}{4C} + \frac{{(2Cu_2-\Xi_2)}^2}{4C} -C \abs{u}^2 \right) \\
    & = \rho
      \exp \left( -\Xi_1 u_1 - \Xi_2 u_2 + \frac{1}{4C}\left(\Xi_1^2 + \Xi_2^2 \right)\right)
  \end{aligned}
\end{equation}

\subsubsection{Equilibrium Distribution for Cumulants}
\label{subs:Equilibrium Distribution for Cumulants}

Like in the derivation of the cumulants, our equilibrium cumulants need the Taylor series of the logarithm of the Laplace transformed equilibrium distribution. With the speed of sound $c_s$ defined as
\begin{equation}
  c_s \defined \sqrt{\frac{k_B T}{m}}
\end{equation}
we get
\begin{equation}
  \begin{aligned}
    \ln(F^{eq}(\Xi))
      & = \ln(\rho) - \Xi_1 u_1 - \Xi_2 u_2 + \frac{1}{4C}\left(\Xi_1^2 + \Xi_2^2 \right) \\
      & = \ln(\rho) - \Xi_1 u_1 - \Xi_2 u_2 + \frac{k_B T}{2m}\left(\Xi_1^2 + \Xi_2^2 \right) \\
      & = \ln(\rho) - \Xi_1 u_1 - \Xi_2 u_2 + \frac{c_s^2}{2}\left(\Xi_1^2 + \Xi_2^2 \right)
  \end{aligned}
\end{equation}

With~\eqref{eq:Definition of cumulants} follows
\begin{equation}
  \label{eq:equilibrium cumulants}
  \begin{aligned}
    \kappa_{00}^{eq} & = \ln(\rho) \\
    \kappa_{10}^{eq} & = - {(-c)}^{-1} u_1 \\
    \kappa_{01}^{eq} & = - {(-c)}^{-1} u_2 \\
    \kappa_{11}^{eq} & = 0 \\
    \kappa_{20}^{eq} & = {(-c)}^{-2} c_s^2  \\
    \kappa_{02}^{eq} & = {(-c)}^{-2} c_s^2  \\
    \kappa_{21}^{eq} & = 0 \\
    \kappa_{12}^{eq} & = 0 \\
    \kappa_{22}^{eq} & = 0.
  \end{aligned}
\end{equation}
Additionally, we can simplify the $\kappa_{02}$ and $\kappa_{20}$. As their equilibrium is the same, we could also say, that the difference has zero equilibrium and their sum double the equilibrium:
\begin{equation}
  \begin{aligned}
    \kappa_{20}^{eq} - \kappa_{02}^{eq} & = 0  \\
    \kappa_{20}^{eq} + \kappa_{02}^{eq} & = 2 {(-c)}^{-2} c_s^2
  \end{aligned}
\end{equation}
%
and the post collision cumulants $\kappa_{\alpha\beta}^*$
\begin{equation}
  \label{eq: post equilibrium cumulants}
  \begin{aligned}
    \kappa_{00}^{*} & = \kappa_{00} + \omega_1 \left( \ln(\rho) - \kappa_{00} \right) \\
    \kappa_{10}^{*} & = \kappa_{10} + \omega_2 \left( - {(-c)}^{-1} u_1 - \kappa_{10} \right) \\
    \kappa_{01}^{*} & = \kappa_{01} + \omega_3 \left( - {(-c)}^{-1} u_2 - \kappa_{01} \right) \\
    \kappa_{11}^{*} & = \kappa_{11} + \omega_4 \left( - \kappa_{11} \right) \\
    \kappa_{20}^{*} - \kappa_{02}^{*}
      & = \kappa_{20} - \kappa_{02} + \omega_5 \left( - \kappa_{20} + \kappa_{02} \right) \\
    \kappa_{20}^{*} + \kappa_{02}^{*}
      & = \kappa_{20} + \kappa_{02} + \omega_6 \left( 2 {(-c)}^{-2} c_s^2 - \kappa_{20} - \kappa_{02} \right) \\
    \kappa_{21}^{*} & = \kappa_{21} + \omega_7 \left( - \kappa_{21} \right) \\
    \kappa_{12}^{*} & = \kappa_{12} + \omega_8 \left( - \kappa_{12} \right) \\
    \kappa_{22}^{*} & = \kappa_{22} + \omega_9 \left( - \kappa_{22} \right)
  \end{aligned}
\end{equation}
which can be simplified to
\begin{equation}
  \begin{aligned}
    \kappa_{00}^{*} & = \kappa_{00} \\
    \kappa_{10}^{*} & = \kappa_{10} \\
    \kappa_{01}^{*} & = \kappa_{01} \\
    \kappa_{11}^{*} & = (1-\omega_4)\kappa_{11} \\
    \kappa_{20}^{*} - \kappa_{02}^{*}
      & = (1-\omega_5) (\kappa_{20} - \kappa_{02}) \\
    \kappa_{20}^{*} + \kappa_{02}^{*}
      & = \kappa_{20} + \kappa_{02} + \omega_6 \left( 2 {(-c)}^{-2} c_s^2 - \kappa_{20} - \kappa_{02} \right) \\
    \kappa_{21}^{*} & = (1-\omega_7)\kappa_{21} \\
    \kappa_{12}^{*} & = (1-\omega_8)\kappa_{12} \\
    \kappa_{22}^{*} & = (1-\omega_9)\kappa_{22}
  \end{aligned}
\end{equation}
%
The recovered post equilibrium values $\kappa_{20}^{*}$ and $\kappa_{02}^{*}$ are with
\begin{equation}
  \begin{aligned}
    a \defined \kappa_{20}^{*} - \kappa_{02}^{*}
      & = (1-\omega_5) (\kappa_{20} - \kappa_{02}) \\
    b \defined \kappa_{20}^{*} + \kappa_{02}^{*}
      & = \kappa_{20} + \kappa_{02} + \omega_6 \left( 2 {(-c)}^{-2} c_s^2 - \kappa_{20} - \kappa_{02} \right)
  \end{aligned}
\end{equation}
\begin{equation}
  \begin{pmatrix}
    \kappa_{20}^{*} \\
    \kappa_{02}^{*}
  \end{pmatrix}
  = \frac{1}{2}
  \begin{pmatrix}
    1 & 1 \\ -1 & 1
  \end{pmatrix}
  \begin{pmatrix}
    a\\
    b
  \end{pmatrix}
\end{equation}
To ensure rotational invariance, we have to relate some relaxation parameters, yielding (where the $\omega_i$ are not related to the ones before)
\begin{equation}
  \begin{aligned}
    \kappa_{00}^{*} & = \kappa_{00} \\
    \kappa_{10}^{*} & = \kappa_{10} \\
    \kappa_{01}^{*} & = \kappa_{01} \\
    \kappa_{11}^{*} & = (1-\omega_1)\kappa_{11} \\
    \kappa_{20}^{*} - \kappa_{02}^{*}
      & = (1-\omega_1) (\kappa_{20} - \kappa_{02}) \\
    \kappa_{20}^{*} + \kappa_{02}^{*}
      & = \kappa_{20} + \kappa_{02} + \omega_2 \left( 2 {(-c)}^{-2} c_s^2 - \kappa_{20} - \kappa_{02} \right) \\
    \kappa_{21}^{*} & = (1-\omega_3)\kappa_{21} \\
    \kappa_{12}^{*} & = (1-\omega_3)\kappa_{12} \\
    \kappa_{22}^{*} & = (1-\omega_4)\kappa_{22}
  \end{aligned}
\end{equation}

\subsubsection{Choosing the \texorpdfstring{$\omega_i$}{omega_i}}
\label{subs:Choosing the omega_i}


\section{Asymptotic Analysis}
\label{sec:Asymptotic Analysis}
\subsection{Preparation}
\label{sub:Preparation}

A Taylor expansion of~\eqref{eq:Lattice Boltzmann Equation} in both time and space gives
\begin{equation*}
  \sum_{\tau = 0}^\infty \frac{{\Delta t}^\tau }{\tau!} \frac{\partial^\tau}{{\partial t}^\tau} f_{ij}(t, x, y) =
  \sum_{m,n = 0}^\infty \frac{{(-i\Delta x)}^m{(-j\Delta x)}^n} {m!n!} \frac{\partial^m \partial^n}{ {\partial x}^m{\partial y}^n} f^*_{ij}(t, x, y)
\end{equation*}
which is equivalent to
\begin{equation}
  \label{eq:Taylor LB1}
  \sum_{\tau = 0}^\infty \frac{{\Delta t}^\tau }{\tau!} \frac{\partial^\tau}{{\partial t}^\tau} f_{ij}(t, x, y) =
    \sum_{m,n = 0}^\infty \frac{{(-c\Delta t)}^{m+n}} {m!n!} \frac{\partial^m \partial^n}{ {\partial x}^m{\partial y}^n} i^m j^n f^*_{ij}(t, x, y),
\end{equation}
using $\frac{\Delta x} {\Delta{t}} = c$.

Equation~\eqref{eq:Taylor LB1} holds for all $f_{ij}$ and thus also for all linear combinations
\begin{equation*}
  \sum_{k}\lambda_k f_{ij},
\end{equation*}
especially for those, which yield the moments, $m_{\alpha\beta} $, of $f$:
\begin{equation*}
  \sum_{k}\lambda_k f_{ij} = \sum_{i,j}i^\alpha j^\beta f_{ij}.
\end{equation*}
Therefore, we rewrite the system of equations for the $f_{ij}$ to a system of equations for the $m_{\alpha\beta}$.

\begin{align}
    \nonumber
    & & \sum_{ij} \sum_{\tau = 0}^\infty \frac{{\Delta t}^\tau }{\tau!} \frac{\partial^\tau}{{\partial t}^\tau} i^\alpha j^\beta f_{ij} &=
    \sum_{ij}\sum_{m,n = 0}^\infty \frac{i^m j^n {(-c\Delta t)}^{m+n}} {m!n!} \frac{\partial^m \partial^n}{ {\partial x}^m{\partial y}^n}i^\alpha j^\beta f^*_{ij}\\
    \nonumber
    &\equivalent &
    \sum_{\tau = 0}^\infty \frac{{\Delta t}^\tau }{\tau!} \frac{\partial^\tau}{{\partial t}^\tau} \sum_{ij}i^\alpha j^\beta f_{ij} &=
    \sum_{m,n = 0}^\infty \frac{{(-c\Delta t)}^{m+n}} {m!n!} \frac{\partial^m \partial^n}{ {\partial x}^m{\partial y}^n}\sum_{ij}i^{(\alpha + m)} j^{(\beta+n)} f^*_{ij}\\
      \label{eq: Taylor of moments}
     &\equivalent &
     \sum_{\tau = 0}^\infty \frac{{\Delta t}^\tau }{\tau!} \frac{\partial^\tau}{{\partial t}^\tau} m_{\alpha\beta} &=
    \sum_{m,n = 0}^\infty \frac{{(-c\Delta t)}^{m+n}} {m!n!} \frac{\partial^m \partial^n}{ {\partial x}^m{\partial y}^n} m^*_{(\alpha + m)(\beta + n)}
\end{align}
For the scale analysis, we introduce a dimensionless scaling parameter $\epsilon$.
Adapting the widely used diffusive scaling, we nondimensionalize the equations with characteristic length and timescales, $L$ and $\iota$
\begin{equation}
  \label{eq:nondimensionalisation}
  \begin{aligned}
    \Delta x & = L\epsilon \\
    \Delta t & = \iota\epsilon^2 \\
    c & = \frac{L}{\iota\epsilon} \\
    \{x, y\} & \rightarrow \{\frac{x}{L}, \frac{y}{L}\} \\
    t & \rightarrow \frac{t}{\iota}.
  \end{aligned}
\end{equation}
Additionally, the moments are expanded according to the scale $\epsilon$
\begin{align}
    \label{eq:expansion of m}
    m_{\alpha\beta} & = \sum_{p=0}^{\infty} \epsilon^p m_{\alpha\beta}^{(p)} \\
    \label{eq:expansion of m*}
    m^*_{\alpha\beta} & = \sum_{p=0}^{\infty} \epsilon^p m_{\alpha\beta}^{*(p)}.
\end{align}
Those are not allowed to depend on $\epsilon$ neither implicitly nor explicitly.
Inserting~\eqref{eq:nondimensionalisation},~\eqref{eq:expansion of m} and~\eqref{eq:expansion of m*} into~\eqref{eq: Taylor of moments} yields

\begin{align}
  \phantom{\equivalent\qquad}
  &\begin{aligned}
  \nonumber
    \mathllap{\equivalent\qquad} & \sum_{\tau = 0}^\infty \frac{{\Delta t}^\tau }{\tau!}  \frac{\partial^\tau}{{\partial t}^\tau} \sum_{p=0}^{\infty} \epsilon^p m_{\alpha\beta}^{(p)} \\
    = &{}\sum_{m,n = 0}^\infty \frac{{(-c\Delta t)}^{m+n}} {m!n!} \frac{\partial^m \partial^n}{ {\partial x}^m{\partial y}^n} \sum_{p=0}^{\infty} \epsilon^p m_{(\alpha + m)(\beta + n)}^{*(p)}
  \end{aligned}\\
  %
  &\begin{aligned}
    \label{eq:nondimensionalised expansion series}
    \mathllap{\equivalent\qquad} &
    \sum_{\tau = 0}^\infty \frac{{(\iota\epsilon^2)}^\tau }{\tau!} \frac{\partial^\tau}{{\partial t}^\tau \iota^\tau} \sum_{p=0}^{\infty} \epsilon^p m_{\alpha\beta}^{(p)} \\
    = &{}\sum_{m,n = 0}^\infty \frac{{(-\frac{L}{\iota\epsilon}(\iota\epsilon^2))}^{m+n}} {m!n!}
    \frac{\partial^m \partial^n}{ {\partial x}^m{\partial y}^n L^{m+n}} \sum_{p=0}^{\infty} \epsilon^p m_{(\alpha + m)(\beta + n)}^{*(p)}
  \end{aligned} \\
 %
  &\begin{aligned}
  \nonumber
    \mathllap{\equivalent\qquad} &
    \sum_{\tau = 0}^\infty \frac{\epsilon^{2\tau} }{\tau!} \frac{\partial^\tau}{{\partial t}^\tau} \sum_{p=0}^{\infty} \epsilon^p m_{\alpha\beta}^{(p)} \\
    = &{}\sum_{m,n = 0}^\infty \frac{{(-\epsilon)}^{m+n}} {m!n!}
    \frac{\partial^m \partial^n}{ {\partial x}^m{\partial y}^n } \sum_{p=0}^{\infty} \epsilon^p m_{(\alpha + m)(\beta + n)}^{*(p)}
  \end{aligned} \\
 %
  &\begin{aligned}
    \label{eq:final nondimensionalised expansion series}
    \mathllap{\equivalent\qquad} &
    \sum_{\tau, p = 0}^\infty\epsilon^{2\tau + p} \frac{1} {\tau!} \frac{\partial^\tau}{{\partial t}^\tau} m_{\alpha\beta}^{(p)} \\
    = &{} \sum_{m,n,p' = 0}^\infty  \epsilon^{m+n+p'} \frac{{(-1)}^{m+n}} {m!n!}
    \frac{\partial^m \partial^n}{ {\partial x}^m{\partial y}^n } m_{(\alpha + m)(\beta + n)}^{*(p')}
  \end{aligned}
\end{align}
Where in~\eqref{eq:final nondimensionalised expansion series}, the summation indices are changed to be unique for better recognition.

\subsection{Derivation of the \texorpdfstring{$\epsilon$}{epsilon} terms}
\label{sub:Derivation of the epsilon terms}
As $\epsilon$ is arbitrary and the terms of the expansion are not allowed to depend on $\epsilon$, equality has to hold in every order of $\epsilon$.

\subsubsection{Zero'th order in \texorpdfstring{$\epsilon$}{epsilon}}
\label{subs:Zero'th order in epsilon}

To achieve $\epsilon^0$, we have to choose $\tau=p=p'=m=n=0$, yielding
\begin{equation}
  \label{eq:zeroth order in epsilon}
  m_{\alpha\beta}^{(0)} = m_{\alpha\beta}^{*(0)}.
\end{equation}


\subsubsection{First order in \texorpdfstring{$\epsilon$}{epsilon}}
\label{subs:First order in epsilon}

For the first order terms, one has to choose $p=1$ on the left side and one of $m$, $n$ and $p'$ equal to one.

\begin{equation}
  \label{eq:first order in epsilon}
  m_{\alpha\beta}^{(1)}
  = m_{\alpha\beta}^{*(1)}
  - \frac{\partial}{\partial x} m_{(\alpha+1)\beta}^{*(0)}
  - \frac{\partial}{\partial y} m_{\alpha(\beta+1)}^{*(0)}
\end{equation}


\subsubsection{Second order in \texorpdfstring{$\epsilon$}{epsilon}}
\label{subs:Second order in epsilon}

Going to the second order requires on the left side $\tau=1$ or $p=2$.
On the right side, we have six possibilities:
\begin{center}
  \begin{tabular} {r || c | *{2}{c} | *{2}{c} | c}
    m  & 0 & 1 & 2 & 0 & 0 & 1 \\
    n  & 0 & 0 & 0 & 1 & 2 & 1 \\
    p' & 2 & 1 & 0 & 1 & 0 & 0
  \end{tabular}
\end{center}
Resulting in
\begin{equation}
  \label{eq:second order in epsilon}
  \begin{aligned}
    \frac{\partial}{\partial t} m_{\alpha\beta}^{(0)} + m_{\alpha\beta}^{(2)}
    & =  m_{\alpha\beta}^{*(2)} \\
    &\quad - \frac{\partial}{\partial x} m_{(\alpha+1)\beta}^{*(1)} + \frac{\partial^2}{\partial x^2} m_{(\alpha+2)\beta}^{*(0)} \\
    &\quad - \frac{\partial}{\partial y} m_{\alpha(\beta+1)}^{*(1)} + \frac{\partial^2}{\partial y^2} m_{\alpha(\beta+2)}^{*(0)} \\
    &\quad + \frac{\partial\partial}{\partial x \partial y} m_{(\alpha+1)(\beta+1)}^{*(0)}
  \end{aligned}
\end{equation}


\subsubsection{Third order in \texorpdfstring{$\epsilon$}{epsilon}}
\label{subs:Third order in epsilon}

Left hand side, either $\begin{pmatrix}\tau \\ p\end{pmatrix} = \begin{pmatrix} 1 \\ 1 \end{pmatrix}$ or $\begin{pmatrix}\tau \\ p\end{pmatrix} = \begin{pmatrix} 0 \\ 3 \end{pmatrix}$.\newline
Right hand side:
\begin{center}
  \begin{tabular} {r || c | *{3}{c} | *{3}{c} | *{3}{c} }
    m  & 0 & 1 & 2 & 3 & 0 & 0 & 0 & 1 & 2 & 1 \\
    n  & 0 & 0 & 0 & 0 & 1 & 2 & 3 & 1 & 1 & 2 \\
    p' & 3 & 2 & 1 & 0 & 2 & 1 & 0 & 1 & 0 & 0
  \end{tabular}
\end{center}

This gives
\begin{equation}
  \label{eq:third order in epsilon}
  \begin{aligned}
    \frac{\partial}{\partial t} m_{\alpha\beta}^{(1)} + m_{\alpha\beta}^{(3)}
    & =  m_{\alpha\beta}^{*(3)} \\
    &\quad - \frac{\partial}{\partial x} m_{(\alpha+1)\beta}^{*(2)} + \frac{\partial^2}{\partial x^2} m_{(\alpha+2)\beta}^{*(1)} - \frac{\partial^3}{\partial x^3} m_{(\alpha+3)\beta}^{*(0)} \\
    &\quad - \frac{\partial}{\partial y} m_{\alpha(\beta+1)}^{*(2)} + \frac{\partial^2}{\partial y^2} m_{\alpha(\beta+2)}^{*(1)} - \frac{\partial^3}{\partial y^3} m_{\alpha(\beta+3)}^{*(0)} \\
    &\quad + \frac{\partial\partial}{\partial x \partial y} m_{(\alpha+1)(\beta+1)}^{*(1)} \\
    &\quad - \frac{\partial^2\partial}{\partial x^2 \partial y} m_{(\alpha+2)(\beta+1)}^{*(0)} - \frac{\partial\partial^2}{\partial x \partial y^2} m_{(\alpha+1)(\beta+2)}^{*(0)}
  \end{aligned}
\end{equation}

\subsection{Analysis of the Terms}
\label{sub:Analysis of the Terms}

\subsubsection{First order}
\label{subs:First order}
As the zeroth order terms are constant, we get
\begin{equation}
  \label{eq:first order analysis}
  \begin{aligned}
    m_{\alpha\beta}^{(1)}
    & = m_{\alpha\beta}^{*(1)}
    - \frac{\partial}{\partial x} m_{(\alpha+1)\beta}^{*(0)}
    - \frac{\partial}{\partial y} m_{\alpha(\beta+1)}^{*(0)} \\
    & = m_{\alpha\beta}^{*(1)}
  \end{aligned}
\end{equation}

And thus, all moments are colission invariants at zeroth,~\eqref{eq:zeroth order in epsilon}, and first order,~\eqref{eq:first order analysis}, in $\epsilon$



\end{document}
