\documentclass{article}
\usepackage[utf8]{inputenc}
\usepackage{algorithm}
\usepackage{algorithmic}
\usepackage{amsfonts}
\usepackage{amsmath}
\usepackage{amssymb}
\usepackage{amsthm}
\usepackage{appendix}

\usepackage{bm}
\usepackage{booktabs}

\usepackage{framed}

\usepackage{graphicx}

\usepackage{hyperref}

\usepackage{mathtools}
\usepackage{multicol}
\usepackage{multirow}

\usepackage{tikz}
\usepackage{todonotes}

\usepackage{wrapfig}

\usepackage{xcolor}
\usepackage{xparse}


\usetikzlibrary{shapes,snakes}
\usetikzlibrary{positioning}


\renewcommand\thesection{\arabic{section}}

%Blindtext
\newcommand{\lorem}{Lorem ipsum dolor sit amet, consectetuer adipiscing
elit. Aenean commodo ligula eget dolor. Aenean massa. Cum sociis natoque
penatibus et magnis dis parturient montes, nascetur ridiculus mus. Donec
quam felis, ultricies nec, pellentesque eu, pretium quis, sem. Nulla
consequat massa quis enim. Donec pede justo, fringilla vel, aliquet nec,
vulputate eget, arcu. In enim justo, rhoncus ut, imperdiet a, venenatis
vitae, justo. Nullam dictum felis eu pede mollis pretium. Integer tincidunt.
Cras dapibus. Vivamus elementum semper nisi. Aenean vulputate eleifend
tellus. Aenean leo ligula, porttitor eu, consequat vitae, eleifend ac,
enim. Aliquam lorem ante, dapibus in, viverra quis, feugiat a, tellus.
Phasellus viverra nulla ut metus varius laoreet. Quisque rutrum. Aenean
imperdiet. Etiam ultricies nisi vel augue. Curabitur ullamcorper ultricies
nisi. Nam eget dui.}


%Arcusfunktionen
\newcommand{\arccsc}{\operatorname{arccsc}}
\newcommand{\arcsec}{\operatorname{arcsec}}
\newcommand{\arccot}{\operatorname{arccot}}

%Hyperbelfunktionen
\newcommand{\csch}{\operatorname{csch}}
\newcommand{\sech}{\operatorname{sech}}

%Areafunktionen
\newcommand{\arsinh}{\operatorname{arsinh}}
\newcommand{\arcosh}{\operatorname{arcosh}}
\newcommand{\artanh}{\operatorname{artanh}}
\newcommand{\arcsch}{\operatorname{arcsch}}
\newcommand{\arsech}{\operatorname{arsech}}
\newcommand{\arcoth}{\operatorname{arcoth}}

%Zahlenmengen
\newcommand{\N}{\ensuremath{\mathbb{N}}}
\newcommand{\R}{\ensuremath{\mathbb{R}}}
\newcommand{\Q}{\ensuremath{\mathbb{Q}}}
\newcommand{\C}{\ensuremath{\mathbb{C}}}
\newcommand{\RtoN}{\ensuremath{\R^n}}


\newcommand{\defined}{\mathrel{\mathop:}=}
\newcommand{\defines}{\mathrel{\mathop=}:}

\newcommand{\argmin}{\operatornamewithlimits{argmin}}
\newcommand{\argmax}{\operatornamewithlimits{argmax}}
\newcommand{\infimum}{\operatornamewithlimits{inf}}
\newcommand{\supremum}{\operatornamewithlimits{sup}}
\newcommand{\nin}{\notin}
\newcommand{\overbar}[1]{\mkern~1.5mu\overline{\mkern-1.5mu#1\mkern-1.5mu}\mkern~1.5mu}
\newcommand{\my}{\mu}
\newcommand{\ny}{\nu}
\newcommand{\boldepsilon}{\ensuremath{\boldsymbol\varepsilon}}
\newcommand{\op}{\operatorname}

%Helfer
\newcommand{\dotProd}[2]{\ensuremath{\langle #1,#2\rangle}}
\newcommand{\inAngles}[1]{\ensuremath{\langle #1\rangle}}

\newcommand{\mathMode}[1]{$ #1 $}
\newcommand{\abs}[1]{\left| #1 \right|}
\newcommand{\norm}[1]{ \left\Vert~#1 \right\Vert}
\newcommand{\newlinedouble}{\newline\newline}


%%%%%%%%%%%%%%%%%%%%%%%%%%%%%%%%%%%%%%%%%%%%%%%%%%%%%%%%%%%%%%%%%%%%%%%%%%%%%%
%                        Definitionen und Sätze                              %
%%%%%%%%%%%%%%%%%%%%%%%%%%%%%%%%%%%%%%%%%%%%%%%%%%%%%%%%%%%%%%%%%%%%%%%%%%%%%%



\DeclareDocumentCommand{\highlight}{ O{red} O{20} m }{
	\colorbox{#1!#2}{$\displaystyle#3$}
}


\newtheorem{mydef}{Definition}[subsection]
\newtheorem{mysatz}{Satz}[subsection]
\newtheorem{mylemma}{Lemma}[subsection]
\newtheorem{mybeispiel}{Beispiel}[subsection]
\newtheorem{mybemerkung}{Bemerkung}[subsection]
\newcommand{\myproof}[1]{
\begin{proof}[Beweis]
#1
\end{proof}
}
                                                                             %
\newtheoremstyle{definition}                                                 %
  {0.3cm}                 %Space above                                       %
  {0.3cm}                 %Space below                                       %
  {\normalfont}           %Body font                                         %
  {}                      %Indent amount                                     %
  {\normalfont\bfseries}  %Thm head font                                     %
  {}                      %Punctuation after thm head                        %
  {\newline}              %Space after thm head                              %
  {\thmname{#1}\thmnumber{ #2}: \thmnote{ #3}}                               %
                          %Thm head spec (can be left empty, meaning         %
                          %`normal')                                         %
                                                                             %
\newenvironment{defshaded}{                                                  %
\def\FrameCommand{\fcolorbox{defframecolor}{defshadecolor}}                  %
\MakeFramed~{\FrameRestore}}                                                 %
{\endMakeFramed}                                                             %
                                                                             %
\newenvironment{cdef}[1][]{\definecolor{defshadecolor}{RGB}{240,240,240}     % oder 224,255,255?
\definecolor{defframecolor}{RGB}{240,240,240}                                %
                                                                             %
\begin{defshaded}\begin{wdef}[#1]\hspace*{0.15cm}}{\end{wdef}\end{defshaded}}%
                                                                             %
    \theoremstyle{definition}                                                %
	\newtheorem{wdef}{Definition}[subsection]                                           %
	                                                                         %
%%%%%%%%%%%%%%%%%%%%%%%%%%%%%%%%%%%%%%%%%%%%%%%%%%%%%%%%%%%%%%%%%%%%%%%%%%%%%%
                                                                             %
\newenvironment{satzshaded}{                                                 %
\def\FrameCommand{\fcolorbox{satzframecolor}{satzshadecolor}}                %
\MakeFramed~{\FrameRestore}}                                                 %
{\endMakeFramed}                                                             %
                                                                             %
\newenvironment{csatz}[1][]{\definecolor{satzshadecolor}{RGB}{240,240,240}   %
\definecolor{satzframecolor}{RGB}{240,240,240}                               %
                                                                             %
\begin{satzshaded}\begin{satz}[#1]\hspace*{0.15cm}}{\end{satz}\end{satzshaded}}%
                                                                             %
    \theoremstyle{definition}                                                %
	\newtheorem{satz}{Theorem}[subsection]                                               %
                                                                             %
%%%%%%%%%%%%%%%%%%%%%%%%%%%%%%%%%%%%%%%%%%%%%%%%%%%%%%%%%%%%%%%%%%%%%%%%%%%%%%

%%%%%%%%%%%%%%%%%%%%%%%%%%%%%%%%%%%%%%%%%%%%%%%%%%%%%%%%%%%%%%%%%%%%%%%%%%%%%%
%                        Nummerierung	                            		 %
%%%%%%%%%%%%%%%%%%%%%%%%%%%%%%%%%%%%%%%%%%%%%%%%%%%%%%%%%%%%%%%%%%%%%%%%%%%%%%
																			 %
\numberwithin{equation}{subsection}											 %
																			 %
%%%%%%%%%%%%%%%%%%%%%%%%%%%%%%%%%%%%%%%%%%%%%%%%%%%%%%%%%%%%%%%%%%%%%%%%%%%%%%



\title{Notes concerning the paper}
\author{Florian Rohm}


\begin{document}
\maketitle

\begin{abstract}
\end{abstract}

\section{Definitions}
\label{sec:Definitions}

Laplace operator:
\begin{equation}
  \label{eq:Definition of Laplace}
  \mathcal{L}[g](\Xi) = \int_{-\infty}^\infty g(\xi) e^{-\Xi \cdot \xi}d\xi
\end{equation}

Continuous definition of f using the Diraque delta
\begin{equation}
  \label{eq:Definition of f xi}
  f(\xi) = \sum_{i,j} f_{ij}\delta(ic - \xi_1)\delta(jc - \xi_2)
\end{equation}
Definition of the Laplace transform of $f$
\begin{equation}
  \label{eq:Definition of F}
  \begin{aligned}
    F(\Xi_1, \Xi_2) & = \mathcal{L}[f](\Xi) = \int_{-\infty}^\infty f(\xi) e^{-\Xi \cdot \xi}d\xi \\
     & = \sum_{i,j}f_{ij} e^{-\Xi_1 ic} e^{-\Xi_2 jc}
  \end{aligned}
\end{equation}
Definition of the moments of $f$
\begin{equation}
  \label{eq:Definition of moments}
  m_{\alpha\beta} = \sum_{ij} i^\alpha j^\beta f_{ij}
\end{equation}
Taylor expansion of $F$
\begin{equation}
  \label{eq: taylor of F}
  \begin{aligned}
    F(\Xi_1, \Xi_2) & = \sum_{\alpha,\beta} \frac{1}{\alpha!\beta!} \frac{\partial^\alpha\partial^\beta}{{(\partial \Xi_1)}^\alpha{(\partial \Xi_2)}^\beta} F(\Xi_1, \Xi_2)\Bigr|_{\Xi_1=\Xi_2 = 0} \Xi_1^\alpha \Xi_2^\beta \\
    & = \sum_{\alpha,\beta} \frac{1}{\alpha!\beta!} \frac{\partial^\alpha\partial^\beta}
      {{(\partial \Xi_1)}^\alpha{(\partial \Xi_2)}^\beta}  \sum_{i,j}f_{ij} e^{-\Xi_1 ic} e^{-\Xi_2 jc} \Bigr|_{\Xi_1=\Xi_2 = 0} \Xi_1^\alpha \Xi_2^\beta \\
    & = \sum_{\alpha,\beta} \frac{1}{\alpha!\beta!} {(-c)}^{\alpha+\beta}  \sum_{i,j} i^\alpha j^\beta f_{ij} e^{-\Xi_1 ic} e^{-\Xi_2 jc} \Bigr|_{\Xi_1=\Xi_2 = 0} \Xi_1^\alpha \Xi_2^\beta \\
    & = \sum_{\alpha,\beta} \frac{1}{\alpha!\beta!} {(-c)}^{\alpha+\beta}  \sum_{i,j} i^\alpha j^\beta f_{ij} \Xi_1^\alpha \Xi_2^\beta \\
    & = \sum_{\alpha,\beta} \frac{1}{\alpha!\beta!} {(-c)}^{\alpha+\beta}  m_{\alpha\beta} \Xi_1^\alpha \Xi_2^\beta
  \end{aligned}
\end{equation}

Hence,
\begin{equation}
  \label{eq:alternative representation of moments}
  m_{\alpha\beta} = {(-c)}^{-\alpha-\beta} \frac{\partial^\alpha\partial^\beta}{{(\partial \Xi_1)}^\alpha{(\partial \Xi_2)}^\beta} F(\Xi_1, \Xi_2)\Bigr|_{\Xi_1=\Xi_2 = 0}
\end{equation}
and $F(\Xi_1, \Xi_2)$ is the moment generating function of $f(\xi_1, \xi_2)$.

Cumulants are defined as the coefficients of the Taylor-expansion of the logarithm of the moment generating function:
\begin{equation}
  \label{eq:Definition of cumulants}
  c_{\alpha\beta} = {(-c)}^{-\alpha-\beta} \frac{\partial^\alpha\partial^\beta}{{(\partial \Xi_1)}^\alpha{(\partial \Xi_2)}^\beta} \ln(F(\Xi_1, \Xi_2))\Bigr|_{\Xi_1=\Xi_2 = 0}
\end{equation}

\section{Lattice Boltzmann Equation for D2Q9}
\label{sec:Lattice Boltzmann Equation for D2Q9}

Velocities are given by
\begin{equation}
  c_{ij}= \begin{pmatrix}i \\ j\end{pmatrix}, \quad i,j\in \{-1, 0, 1\}
\end{equation}
With $c$ defined as the cell width, the lattice Boltzmann equation is given by
\begin{equation}
  f_{ij}(t + \Delta t, x + ic\Delta x , y + jc\Delta y) = f^*_{ij}(t,x,y), \quad i,j\in \{-1, 0, 1\}
\end{equation}
or equivalently by shifting $x$ and $y$
\begin{equation}
  \label{eq:Lattice Boltzmann Equation}
  f_{ij}(t + \Delta t, x, y) = f^*_{ij}(t,x - ic\Delta x , y - jc\Delta y), \quad i,j\in \{-1, 0, 1\}
\end{equation}
where $f_{ij}$ are the pre-collission distributions and $f^*_{ij}$ the post collission distributions.

\section{Asymptotic Analysis}
\label{sec:Asymptotic Analysis}
\subsection{Preparation}
\label{sub:Preparation}

A Taylor expansion of~\eqref{eq:Lattice Boltzmann Equation} in both time and space gives
\begin{equation*}
  \sum_{\tau = 0}^\infty \frac{{\Delta t}^\tau }{\tau!} \frac{\partial^\tau}{{\partial t}^\tau} f_{ij}(t, x, y) =
  \sum_{m,n = 0}^\infty \frac{{(-ic\Delta x)}^m{(-jc\Delta x)}^n} {m!n!} \frac{\partial^m \partial^n}{ {\partial x}^m{\partial y}^n} f^*_{ij}(t, x, y)
\end{equation*}
which is equivalent to
\begin{equation}
  \label{eq:Taylor LB1}
  \sum_{\tau = 0}^\infty \frac{{\Delta t}^\tau }{\tau!} \frac{\partial^\tau}{{\partial t}^\tau} f_{ij}(t, x, y) =
    \sum_{m,n = 0}^\infty \frac{{\Delta t}^{m+n}} {m!n!} \frac{\partial^m \partial^n}{ {\partial x}^m{\partial y}^n} i^m j^n f^*_{ij}(t, x, y),
\end{equation}
using $\frac{\Delta x} {\Delta{t}} = c$.

Equation~\eqref{eq:Taylor LB1} holds for all $f_{ij}$ and thus also for all linear combinations
\begin{equation*}
  \sum_{k}\lambda_k f_{ij},
\end{equation*}
especially for those, which yield the moments, $m_{\alpha\beta} $, of $f$:
\begin{equation*}
  \sum_{k}\lambda_k f_{ij} = \sum_{i,j}i^\alpha j^\beta f_{ij}.
\end{equation*}
Therefore, we rewrite the system of equations for the $f_{ij}$ to a system of equations for the $m_{\alpha\beta}$.

\begin{align}
    \nonumber
    & & \sum_{ij} \sum_{\tau = 0}^\infty \frac{{\Delta t}^\tau }{\tau!} \frac{\partial^\tau}{{\partial t}^\tau} i^\alpha j^\beta f_{ij} &=
    \sum_{ij}\sum_{m,n = 0}^\infty \frac{i^m j^n {\Delta t}^{m+n}} {m!n!} \frac{\partial^m \partial^n}{ {\partial x}^m{\partial y}^n}i^\alpha j^\beta f^*_{ij}\\
    \nonumber
    &\Leftrightarrow &
    \sum_{\tau = 0}^\infty \frac{{\Delta t}^\tau }{\tau!} \frac{\partial^\tau}{{\partial t}^\tau} \sum_{ij}i^\alpha j^\beta f_{ij} &=
    \sum_{m,n = 0}^\infty \frac{{\Delta t}^{m+n}} {m!n!} \frac{\partial^m \partial^n}{ {\partial x}^m{\partial y}^n}\sum_{ij}i^{(\alpha + m)} j^{(\beta+n)} f^*_{ij}\\
      \label{eq: Taylor of moments}
     &\Leftrightarrow &
     \sum_{\tau = 0}^\infty \frac{{\Delta t}^\tau }{\tau!} \frac{\partial^\tau}{{\partial t}^\tau} m_{\alpha\beta} &=
    \sum_{m,n = 0}^\infty \frac{{\Delta t}^{m+n}} {m!n!} \frac{\partial^m \partial^n}{ {\partial x}^m{\partial y}^n} m^*_{(\alpha + m)(\beta + n)}
\end{align}
For the scale analysis, we introduce a dimensionless scaling parameter $\epsilon$.
Adapting the widely used diffusive scaling, we nondimensionalize the equations with characteristic length and timescales, $L$ and $\iota$
\begin{equation}
  \label{eq:nondimensionalisation}
  \begin{aligned}
    \Delta x & = L\epsilon \\
    \Delta t & = \iota\epsilon^2 \\
    c & = \frac{L}{\iota\epsilon} \\
    \{x, y\} & \rightarrow \{\frac{x}{L}, \frac{y}{L}\} \\
    t & \rightarrow \frac{t}{\iota}.
  \end{aligned}
\end{equation}
Additionally, the moments are expanded according to the scale $\epsilon$
\begin{align}
    \label{eq:expansion of m}
    m_{\alpha\beta} & = \sum_{p=1}^{\infty} \epsilon^p m_{\alpha\beta}^{(p)} \\
    \label{eq:expansion of m*}
    m^*_{\alpha\beta} & = \sum_{p=1}^{\infty} \epsilon^p m_{\alpha\beta}^{*(p)}.
\end{align}
Those are not allowed to depend on $\epsilon$ neither implicitly nor explicitly.
Inserting~\eqref{eq:nondimensionalisation},~\eqref{eq:expansion of m} and~\eqref{eq:expansion of m*} into~\eqref{eq: Taylor of moments} yields
\begin{align}
  \sum_{\tau = 0}^\infty \frac{{\Delta t}^\tau }{\tau!} \frac{\partial^\tau}{{\partial t}^\tau} \sum_{p=1}^{\infty} \epsilon^p m_{\alpha\beta}^{(p)}  &=
 \sum_{m,n = 0}^\infty \frac{{\Delta t}^{m+n}} {m!n!} \frac{\partial^m \partial^n}{ {\partial x}^m{\partial y}^n} \sum_{p=1}^{\infty} \epsilon^p m_{(\alpha + m)(\beta + n)}^{*(p)}
\end{align}


\end{document}
