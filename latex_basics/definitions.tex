\usepackage{algorithm}
\usepackage{algorithmic}
\usepackage{amsfonts}
\usepackage{amsmath}
\usepackage{amssymb}
\usepackage{amsthm}
\usepackage{appendix}

\usepackage{bm}
\usepackage{booktabs}

\usepackage{framed}

\usepackage{graphicx}

\usepackage{hyperref}

\usepackage{mathtools}
\usepackage{minted}
\usepackage{multicol}
\usepackage{multirow}

\usepackage{tikz}
\usepackage{todonotes}

\usepackage{wrapfig}

\usepackage{xcolor}
\usepackage{xparse}
\allowdisplaybreaks{}

\usetikzlibrary{shapes, decorations}
\usetikzlibrary{positioning}


\renewcommand\thesection{\arabic{section}}

%Blindtext
\newcommand{\lorem}{Lorem ipsum dolor sit amet, consectetuer adipiscing
elit. Aenean commodo ligula eget dolor. Aenean massa. Cum sociis natoque
penatibus et magnis dis parturient montes, nascetur ridiculus mus. Donec
quam felis, ultricies nec, pellentesque eu, pretium quis, sem. Nulla
consequat massa quis enim. Donec pede justo, fringilla vel, aliquet nec,
vulputate eget, arcu. In enim justo, rhoncus ut, imperdiet a, venenatis
vitae, justo. Nullam dictum felis eu pede mollis pretium. Integer tincidunt.
Cras dapibus. Vivamus elementum semper nisi. Aenean vulputate eleifend
tellus. Aenean leo ligula, porttitor eu, consequat vitae, eleifend ac,
enim. Aliquam lorem ante, dapibus in, viverra quis, feugiat a, tellus.
Phasellus viverra nulla ut metus varius laoreet. Quisque rutrum. Aenean
imperdiet. Etiam ultricies nisi vel augue. Curabitur ullamcorper ultricies
nisi. Nam eget dui.}


%Arcusfunktionen
\newcommand{\arccsc}{\operatorname{arccsc}}
\newcommand{\arcsec}{\operatorname{arcsec}}
\newcommand{\arccot}{\operatorname{arccot}}

%Hyperbelfunktionen
\newcommand{\csch}{\operatorname{csch}}
\newcommand{\sech}{\operatorname{sech}}

%Areafunktionen
\newcommand{\arsinh}{\operatorname{arsinh}}
\newcommand{\arcosh}{\operatorname{arcosh}}
\newcommand{\artanh}{\operatorname{artanh}}
\newcommand{\arcsch}{\operatorname{arcsch}}
\newcommand{\arsech}{\operatorname{arsech}}
\newcommand{\arcoth}{\operatorname{arcoth}}

% Spaces
\newcommand{\N}{\ensuremath{\mathbb{N}}}
\newcommand{\R}{\ensuremath{\mathbb{R}}}
\newcommand{\Q}{\ensuremath{\mathbb{Q}}}
\newcommand{\C}{\ensuremath{\mathbb{C}}}
\newcommand{\RtoN}{\ensuremath{\R^n}}

% at least some define symbol
\newcommand{\defined}{\mathrel{\mathop:}=}
\newcommand{\defines}{\mathrel{\mathop=}:}

% miscelaneous operators and helpers
\newcommand{\argmin}{\operatornamewithlimits{argmin}}
\newcommand{\argmax}{\operatornamewithlimits{argmax}}
\newcommand{\infimum}{\operatornamewithlimits{inf}}
\newcommand{\supremum}{\operatornamewithlimits{sup}}
\newcommand{\nin}{\notin}
\newcommand{\overbar}[1]{\mkern~1.5mu\overline{\mkern-1.5mu#1\mkern-1.5mu}\mkern~1.5mu}
\newcommand{\my}{\mu}
\newcommand{\ny}{\nu}
\newcommand{\boldepsilon}{\ensuremath{\boldsymbol\varepsilon}}
\newcommand{\op}{\operatorname}

% useful features for faster texing
\newcommand{\dotProd}[2]{\ensuremath{\langle #1,#2\rangle}}
\newcommand{\inAngles}[1]{\ensuremath{\langle #1 \rangle}}
\newcommand{\abs}[1]{\left| #1 \right|}
\newcommand{\norm}[1]{ \left\Vert~#1 \right\Vert}
\newcommand{\equivalent}{\Leftrightarrow}

\newcommand{\newlinedouble}{\newline\newline}

% Matrix custom spacing
\makeatletter
\renewcommand*\env@matrix[1][\arraystretch]{%
  \edef\arraystretch{#1}%
  \hskip -\arraycolsep  % chktex 1 chktex 41
  \let\@ifnextchar\new@ifnextchar  % chktex 1 chktex 21
  \array{*\c@MaxMatrixCols c}}  % chktex 1
\makeatother

% Highlight command
\DeclareDocumentCommand{\highlight}{ O{red} O{20} m }{
	\colorbox{#1!#2}{$\displaystyle#3$}
}


%%%%%%%%%%%%%%%%%%%%%%%%%%%%%%%%%%%%%%%%%%%%%%%%%%%%%%%%%%%%%%%%%%%%%%%%%%%%%%
%                        Nummerierung	                            		 %
%%%%%%%%%%%%%%%%%%%%%%%%%%%%%%%%%%%%%%%%%%%%%%%%%%%%%%%%%%%%%%%%%%%%%%%%%%%%%%
																			 %
\numberwithin{equation}{subsection}
