% !TEX root = ../main.tex

\paragraph{Goal}
\label{par:Goal}
For convenience, we state the momentum equation of the Navier-Stokes-Equation.

\subsection{Incompressible momentum equation}
\label{sub:Incompressible momentum equation}
The incompressible momentum equation reads
\begin{equation}
  \partial_t \mathbf{v} + \left(\mathbf{v}\cdot\nabla\right)\mathbf{v} = - \frac{1}{\rho}\nabla p + \nu \Delta \mathbf{v} + \mathbf{f}
\end{equation}
or in separate equations
\begin{equation}
  \begin{aligned}
    \partial_t v_x + v_x \partial_x v_x + v_y \partial_y v_x
    &= -  \frac{1}{\rho}\partial_x p + \nu (\partial_{xx} v_x + \partial_{yy} v_y) + f_x \\
    \partial_t v_y + v_x \partial_x v_y + v_y \partial_y v_y
    &= -  \frac{1}{\rho}\partial_y p + \nu (\partial_{xx} v_x + \partial_{yy} v_y) + f_y
  \end{aligned}
\end{equation}

\subsection{Compressible momentum equation}
\label{sub:Compressible momentum equation}
The compressible momentum equation reads in conservative form
\begin{equation}
  \rho \left(\partial_t \mathbf{v} + \mathbf{v}\cdot \nabla \mathbf{v}\right)=-\nabla p + \nabla\cdot (\mu \cdot(\nabla\mathbf{v} + {(\nabla\mathbf{v})}^T))+\nabla(\lambda\nabla\cdot\mathbf{v}) + f
\end{equation}
\begin{equation}
  \label{eq:compressible NS}
  \begin{aligned}
    \partial_t \mathbf{m} + \nabla \cdot (\mathbf{m}\mathbf{v}^T)
    &= - \nabla p + \nabla \cdot (\mu\mathbb{S}) + \rho \mathbf{f} \\
    \Leftrightarrow\partial_t \mathbf{m} + (\nabla \cdot \mathbf{m})\mathbf{v} + (\nabla \cdot \mathbf{v})\mathbf{m}
    &= - \nabla p + \mu \nabla \cdot \mathbb{S} + \mathbb{S}^T \nabla\mu + \rho \mathbf{f}
  \end{aligned}
\end{equation}
where
\begin{equation}
  \mathbf{m} = \rho \mathbf{u}
\end{equation}
is the momentum density,
\begin{equation}
  \mathbb{S} \defined \left({(D\mathbf{v})}^T + D\mathbf{v} - \frac{2}{3}\mathbb{I}\nabla\cdot\mathbf{v} \right)
\end{equation}
with the abused notation of the divergence of a matrix
\begin{equation}
  \nabla \cdot M =
  \begin{pmatrix}
    \partial_x M_{11} + \partial_y M_{21} \\
    \partial_x M_{12} + \partial_y M_{22}
  \end{pmatrix}
\end{equation}
and the Jacobian Matrix of $\mathbf{v}$ denoted as $D\mathbf{v}$.
The split equations read
\begin{equation}
  \begin{aligned}
    & \partial_t \text{m}_x
    + v_x (\partial_x \text{m}_x + \partial_y \text{m}_y)
    + \text{m}_x (\partial_x v_x + \partial_y v_y)  \\
    & =
     - \partial_x p
     + \mu \left(\frac{4}{3}\partial_{xx}v_x
            + \frac{1}{3}\partial_x\partial_y v_y
            + \partial_{yy} v_x \right) \\
    &\quad + \frac{4}{3}\partial_x \mu \partial_x v_x
    - \frac{2}{3}\partial_x \mu \partial_y v_y
    + \partial_y \mu \partial_y v_x
    + \partial_y \mu \partial_x v_y
    + \rho f_x
  \end{aligned}
\end{equation}
\begin{equation}
  \begin{aligned}
    & \partial_t \text{m}_y
    + v_y (\partial_x \text{m}_x + \partial_y \text{m}_y)
    + \text{m}_y (\partial_x v_x + \partial_y v_y)  \\
    & =
     - \partial_y p
     + \mu \left(\frac{4}{3}\partial_{yy}v_y
            + \frac{1}{3}\partial_x\partial_y v_x
            + \partial_{xx} v_y \right) \\
    &\quad + \frac{4}{3}\partial_y \mu \partial_y v_y
    - \frac{2}{3}\partial_y \mu \partial_x v_x
    + \partial_y \mu \partial_x v_y
    + \partial_y \mu \partial_y v_x
    + \rho f_y
  \end{aligned}
\end{equation}

For our method, we need to insert the density, velocity, pressure and momentum density in terms of our moments.
Only the x-direction is stated:
\begin{equation}
  \begin{aligned}
    & \partial_t m_{10}
    + \frac{m_{10}}{m_{00}} (\partial_x m_{10} + \partial_y m_{01})
    + m_{10} (\partial_x \frac{m_{10}}{m_{00}} + \partial_y \frac{m_{01}}{m_{00}})  \\
    & =
     - \frac{1}{3} \partial_x \left(m_{20} - \frac{ m_{10}^2 }{ m_{00} } \right)
     + \mu \left(\frac{4}{3}\partial_{xx}\frac{m_{10}}{m_{00}}
            + \frac{1}{3}\partial_x\partial_y \frac{m_{01}}{m_{00}}
            + \partial_{yy} \frac{m_{10}}{m_{00}} \right) \\
    &\quad + \frac{4}{3}\partial_x \mu \partial_x \frac{m_{10}}{m_{00}}
    - \frac{2}{3}\partial_x \mu \partial_y \frac{m_{01}}{m_{00}}
    + \partial_y \mu \partial_y \frac{m_{10}}{m_{00}}
    + \partial_y \mu \partial_x \frac{m_{01}}{m_{00}}
    + m_{00}f_x
  \end{aligned}
\end{equation}
It should be noted, that the term $\nabla \mu$ will only be relevant, if the viscosity is changing. In ideal gases, $\mu$ is a variable of the temperature only. As our method only treats isothermal phenomena in this setting, we are save to assume $\mu$ to be constant.
For a general treatment, of compressible flow, this has to be accounted for, as temperature is coupled to pressure and density, thus $\mu$ won't be constant.
The resulting Navier Stokes equation reads now:
\begin{equation}
  \label{eq:compressible NS isotherm}
  \partial_t \mathbf{m} + (\nabla \cdot \mathbf{m})\mathbf{v} + (\nabla \cdot \mathbf{v})\mathbf{m}
    = - \nabla p + \mu \nabla \cdot \mathbb{S} + \rho \mathbf{f}
\end{equation}
Just in x-direction:
\begin{equation}
  \begin{aligned}
    & \partial_t \text{m}_x
    + v_x (\partial_x \text{m}_x + \partial_y \text{m}_y)
    + \text{m}_x (\partial_x v_x + \partial_y v_y)  \\
    & =
     - \partial_x p
     + \mu \left(\frac{4}{3}\partial_{xx}v_x
            + \frac{1}{3}\partial_x\partial_x v_y
            + \partial_{yy} v_x \right)  + \rho f_x
  \end{aligned}
\end{equation}
and finally with moments:
\begin{equation}
  \begin{aligned}
    & \partial_t m_{10}
    + \frac{m_{10}}{m_{00}} (\partial_x m_{10} + \partial_y m_{01})
    + m_{10} (\partial_x \frac{m_{10}}{m_{00}} + \partial_y \frac{m_{01}}{m_{00}})  \\
    & =
    - \frac{1}{3} \partial_x \left(m_{20} - \frac{ m_{10}^2 }{ m_{00} } \right)
     + \mu \left(\frac{4}{3}\partial_{xx}\frac{m_{10}}{m_{00}}
            + \frac{1}{3}\partial_x\partial_y \frac{m_{01}}{m_{00}}
            + \partial_{yy} \frac{m_{10}}{m_{00}} \right) + m_{00}f_x
  \end{aligned}
\end{equation}

\subsection{Derivation}
\label{sub:Derivation of momentum equation}

For $\epsilon^3$, $\alpha=1$, $\beta=0$ we get
\begin{equation}
  \label{eq: start of momentum equation derivation}
  \begin{aligned}
     \partial_t m_{10}^{(1)} & =
    m_{10}^{*(3)} - m_{10}^{(3)} - \partial_x \highlight{m_{20}^{*(2)}} \\
    &\quad - \partial_y \highlight[green]{m_{11}^{*(2)}} + \frac{1}{2}\partial_{xx} m_{30}^{*(1)} + \frac{1}{2} \partial_{yy} m_{12}^{*(1)} + \partial_{xy} m_{21}^{*(1)}
  \end{aligned}
\end{equation}
We find the term $\partial_t m_{10}^{(1)} $ which is the desired time derivative from the momentum equation and
\begin{equation}
  m_{10}^{*(3)} - m_{10}^{(3)}\defines F_x^{(3)}
\end{equation}
 which is the non-conserved term of the $x$-velocity, due to an external force. For all other moments, we need to derive further equations.

$\epsilon^2$, $\alpha=1$, $\beta=0$:
\begin{equation}
  \label{eq: m10 star pde}
  m_{10}^{(2)} + \partial_t m_{10}^{(0)} + \partial_y m_{11}^{*(1)} + \partial_x m_{20}^{*(1)} = m_{10}^{*(2)}
\end{equation}

$\epsilon^2$, $\alpha=0$, $\beta=1$:
\begin{equation}
  \label{eq: m01 star pde}
  m_{01}^{(2)} + \partial_t m_{01}^{(0)} + \partial_y m_{02}^{*(1)} + \partial_x m_{11}^{*(1)} = m_{01}^{*(2)}
\end{equation}

$\epsilon^2$, $\alpha=1$, $\beta=1$:
\begin{equation}
  \label{eq: m11 star pde}
  m_{11}^{(2)} + \partial_t m_{11}^{(0)} + \partial_y m_{12}^{*(1)} + \partial_x m_{21}^{*(1)} = \highlight[green]{m_{11}^{*(2)}}
\end{equation}

$\epsilon^2$, $\alpha=2$, $\beta=0$:
\begin{equation}
  \label{eq: m20 star pde}
  m_{20}^{(2)} + \partial_t m_{20}^{(0)} + \partial_y m_{21}^{*(1)} + \partial_x m_{30}^{*(1)} = \highlight{m_{20}^{*(2)}}
\end{equation}

$\epsilon^2$, $\alpha=0$, $\beta=2$:
\begin{equation}
  \label{eq: m02 star pde}
  m_{02}^{(2)} + \partial_t m_{02}^{(0)} + \partial_y m_{03}^{*(1)} + \partial_x m_{12}^{*(1)} = m_{02}^{*(2)}
\end{equation}

Combining~\eqref{eq: start of momentum equation derivation} with~\eqref{eq: m20 star pde} and~\eqref{eq: m11 star pde} results in

\begin{align}
  \phantom{\partial_t m_{10}^{(1)}}
  &\begin{aligned}
  \nonumber
    \mathllap{\partial_t m_{10}^{(1)}} & =
    m_{10}^{*(3)} - m_{10}^{(3)}
    - \left(\partial_x m_{20}^{(2)} + \partial_t\partial_x m_{20}^{(0)} + \partial_{xy} m_{21}^{*(1)} + \partial_{xx} m_{30}^{*(1)}\right) \\
    &
    - \left(\partial_y m_{11}^{(2)} + \partial_t\partial_y m_{11}^{(0)} + \partial_{yy} m_{12}^{*(1)} + \partial_{xy} m_{21}^{*(1)}\right) \\
    &
    + \frac{1}{2}\partial_{xx} m_{30}^{*(1)} + \frac{1}{2} \partial_{yy} m_{12}^{*(1)} + \partial_{xy} m_{21}^{*(1)}
  \end{aligned} \\
  %
  &\begin{aligned}
  \nonumber
   & =
    F_x^{(3)}
    - \left(\partial_x m_{20}^{(2)} + \partial_{xy} m_{21}^{*(1)} + \partial_{xx} m_{30}^{*(1)}\right) \\
    &
    - \left(\partial_y m_{11}^{(2)} + \partial_{yy} m_{12}^{*(1)} + \partial_{xy} m_{21}^{*(1)}\right) \\
    &
    + \frac{1}{2}\partial_{xx} m_{30}^{*(1)} + \frac{1}{2} \partial_{yy} m_{12}^{*(1)} + \partial_{xy} m_{21}^{*(1)}
  \end{aligned}\\
%
  &\begin{aligned}
  \nonumber
   & =
    F_x^{(3)}
    - \partial_x m_{20}^{(2)} - \partial_y m_{11}^{(2)}   \\
    &
    - \frac{1}{2} \left(\partial_{xx} m_{30}^{*(1)} + \partial_{yy} m_{12}^{*(1)} + 2\partial_{xy} m_{21}^{*(1)}\right)
  \end{aligned}
\end{align}

As stated in~\eqref{eq:first order collision invariant}, the first order moments, have to be equal to their equilibrium value. Additionally, with the aliasing of moments as stated in table~\ref{table:D2Q9 moments}, we get
\begin{equation}
  \begin{aligned}
  \partial_t m_{10}^{(1)}
   & =
    F_x^{(3)}
    - \partial_x m_{20}^{(2)} - \partial_y m_{11}^{(2)} \\
    &
    - \frac{1}{2} \left(\partial_{xx} m_{10}^{eq(1)} + \partial_{yy} m_{12}^{eq(1)} + 2\partial_{xy} m_{21}^{eq(1)}\right)
  \end{aligned}
\end{equation}
From here on, we have to consider our specific collision. To do so, we use the series expansion of the cumulants and relate them to the moments.
