% !TEX root = ../thesis.tex
We seek a representation of the cumulants in terms of moments, as the later ones are much more easy to compute, c.f.~\eqref{eq: moment definition} and are directly connected to the physical quantities we're after, c.f.~\eqref{eq: density definition} to~\eqref{eq: y velocity definition}.
In this section, we will just use the definitions of moments, central moments and cumulants independently of the \gls{lbm}.
Hence, all derivations are valid both in the pre- and post-collision setting, as well as the equilibrium.

To actually calculate the cumulants, we need the moment generating function for $f_{ij}$ first.
For this, we formulate our \glspl{pdf} $f_{ij}$ in a continuous way
\begin{equation}
  \label{eq:Definition of f xi}
  f(\vec{\xi}) \defined \sum_{i,j} f_{ij}\delta(ic - \xi_1)\delta(jc - \xi_2),
\end{equation}
using the Diraque delta distribution\footnote{For a quick overview, please refer to~\cite{weissteinDelta}} $\delta$, characterized by
\begin{equation}
  \label{eq: diraque feature}
  \int f(x)\delta(x)dx = f(0).
\end{equation}
Albeit not a function in the classical sense, we will use $f(\vec{\xi})$ only within an integral expression so we don't get any ambiguity.
To get the moment generating function from a density distribution, we have to introduce an integral transformation $\mathcal{L}$, similar to the Laplace transformation, c.f.~\cite{weissteinLaplace}, defined as
\begin{equation}
  \label{eq: Definition of integral transformation}
  \mathcal{L}[g](\vec{\Xi}) \defined \int_{\R^n} g(\vec{\xi}) e^{\vec{\Xi} \cdot \vec{\xi}}d\vec{\xi}.
\end{equation}

With the help of this transformation, we get the moment generating function $F$ of our \glspl{pdf}
\begin{equation}
  \label{eq:Definition of F}
  \begin{aligned}
    F(\Xi_1, \Xi_2) & \defined \mathcal{L}[f](\vec{\Xi}) = \int_{\R^2} f(\vec{\xi}) e^{\vec{\Xi} \cdot \vec{\xi}}d\vec{\xi} \\
     & = \sum_{i,j}f_{ij} \int_{\R^2} \delta(ic - \xi_1)\delta(jc - \xi_2) e^{\vec{\Xi} \cdot \vec{\xi}}d\vec{\xi}
  \\
     & = \sum_{i,j}f_{ij} e^{\Xi_1 ic} e^{\Xi_2 jc},
  \end{aligned}
\end{equation}
using equation~\eqref{eq: diraque feature}.
Using the Taylor expansion of $F$ and the original definition of moments,~\eqref{eq: moment definition}, we get
\begin{equation}
  \label{eq: taylor of F}
  \begin{aligned}
    F(\Xi_1, \Xi_2) & = \sum_{\alpha,\beta} \frac{1}{\alpha!\beta!} \frac{\partial^\alpha\partial^\beta}{{(\partial \Xi_1)}^\alpha{(\partial \Xi_2)}^\beta} F(\Xi_1, \Xi_2)\Bigr|_{\Xi_1=\Xi_2 = 0} \Xi_1^\alpha \Xi_2^\beta \\
    & = \sum_{\alpha,\beta} \frac{1}{\alpha!\beta!} \frac{\partial^\alpha\partial^\beta}
      {{(\partial \Xi_1)}^\alpha{(\partial \Xi_2)}^\beta}  \sum_{i,j} f_{ij} e^{\Xi_1 ic} e^{\Xi_2 jc} \Bigr|_{\Xi_1=\Xi_2 = 0} \Xi_1^\alpha \Xi_2^\beta \\
    & = \sum_{\alpha,\beta} \frac{1}{\alpha!\beta!}
    \sum_{i,j}  {(ic)}^\alpha {(jc)}^\beta f_{ij} e^{\Xi_1 ic} e^{\Xi_2 jc} \Bigr|_{\Xi_1=\Xi_2 = 0} \Xi_1^\alpha \Xi_2^\beta \\
    & = \sum_{\alpha,\beta} \frac{1}{\alpha!\beta!}
    \sum_{i,j} {(ic)}^\alpha {(jc)}^\beta f_{ij} \Xi_1^\alpha \Xi_2^\beta \\
    & = \sum_{\alpha,\beta} \frac{1}{\alpha!\beta!} m_{\alpha\beta} \Xi_1^\alpha \Xi_2^\beta
  \end{aligned}
\end{equation}
and hence
\begin{equation}
  \label{eq: alternative representation of moments}
  m_{\alpha\beta} = \frac{\partial^\alpha\partial^\beta}{{(\partial \Xi_1)}^\alpha{(\partial \Xi_2)}^\beta} F(\Xi_1, \Xi_2)\Bigr|_{\Xi_1=\Xi_2 = 0},
\end{equation}
verifying the equivalence of the definitions.

As mentioned in Section~\ref{sub: Cumulants}, the cumulants are the coefficients on the Taylor expansion of the logarithm of the moment generating function.

This yields for the \textbf{cumulant} $\kappa_{\alpha\beta}$:
\begin{equation}
  \label{eq: definition of cumulants}
  \kappa_{\alpha\beta} = \frac{\partial^\alpha\partial^\beta}{{(\partial \Xi_1)}^\alpha{(\partial \Xi_2)}^\beta} \ln(F(\Xi_1, \Xi_2))\Bigr|_{\Xi_1=\Xi_2 = 0}.
\end{equation}
