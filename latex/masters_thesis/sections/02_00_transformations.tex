% !TEX root = ../thesis.tex
We seek a representation of the cumulants $\kappa_{\alpha\beta}$ in terms of moments, as the later ones are much more easy to compute, c.f.~\eqref{eq: moment definition}.

To actually calculate the cumulants, we need the moment generating function for $f_{ij}$ first.
For this, we formulate our \glspl{pdf} $f_{ij}$ in a continuous way
\begin{equation}
  \label{eq:Definition of f xi}
  f(\vec{\xi}) \defined \sum_{i,j} f_{ij}\delta(ic - \xi_1)\delta(jc - \xi_2),
\end{equation}
using the Diraque delta distribution\footnote{For a quick overview, please refer to~\cite{weissteinDelta}} $\delta$, characterized by
\begin{equation}
  \int f(x)\delta(x)dx = f(0).
\end{equation}
Albeit not a function in the classical sense, we will use $f(\vec{\xi})$ only within an integral expression so we don't get any ambiguity.
With the help of the Laplace transformation, described in Appendix~\ref{appendix: Laplace transform}, we get
\begin{equation}
  \label{eq:Definition of F}
  \begin{aligned}
    F(\Xi_1, \Xi_2) & \defined \mathcal{L}[f](\vec{\Xi}) = \int_{-\infty}^\infty f(\vec{\xi}) e^{-\vec{\Xi} \cdot \vec{\xi}}d\vec{\xi} \\
     & = \sum_{i,j}f_{ij} e^{-\Xi_1 ic} e^{-\Xi_2 jc}.
  \end{aligned}
\end{equation}
The moments are now equivalently defined as the coefficients in the Taylor expansion of $F$.
We can check this equivalence by
\begin{equation}
  \label{eq: taylor of F}
  \begin{aligned}
    F(\Xi_1, \Xi_2) & = \sum_{\alpha,\beta} \frac{1}{\alpha!\beta!} \frac{\partial^\alpha\partial^\beta}{{(\partial \Xi_1)}^\alpha{(\partial \Xi_2)}^\beta} F(\Xi_1, \Xi_2)\Bigr|_{\Xi_1=\Xi_2 = 0} \Xi_1^\alpha \Xi_2^\beta \\
    & = \sum_{\alpha,\beta} \frac{1}{\alpha!\beta!} \frac{\partial^\alpha\partial^\beta}
      {{(\partial \Xi_1)}^\alpha{(\partial \Xi_2)}^\beta}  \sum_{i,j}f_{ij} e^{-\Xi_1 ic} e^{-\Xi_2 jc} \Bigr|_{\Xi_1=\Xi_2 = 0} \Xi_1^\alpha \Xi_2^\beta \\
    & = \sum_{\alpha,\beta} \frac{1}{\alpha!\beta!} {(-c)}^{\alpha+\beta}  \sum_{i,j} i^\alpha j^\beta f_{ij} e^{-\Xi_1 ic} e^{-\Xi_2 jc} \Bigr|_{\Xi_1=\Xi_2 = 0} \Xi_1^\alpha \Xi_2^\beta \\
    & = \sum_{\alpha,\beta} \frac{1}{\alpha!\beta!} {(-c)}^{\alpha+\beta}  \sum_{i,j} i^\alpha j^\beta f_{ij} \Xi_1^\alpha \Xi_2^\beta \\
    & = \sum_{\alpha,\beta} \frac{1}{\alpha!\beta!} {(-c)}^{\alpha+\beta}  m_{\alpha\beta} \Xi_1^\alpha \Xi_2^\beta
  \end{aligned}
\end{equation}
Hence,
\begin{equation}
  \label{eq: alternative representation of moments}
  m_{\alpha\beta} = {(-c)}^{-\alpha-\beta} \frac{\partial^\alpha\partial^\beta}{{(\partial \Xi_1)}^\alpha{(\partial \Xi_2)}^\beta} F(\Xi_1, \Xi_2)\Bigr|_{\Xi_1=\Xi_2 = 0}.
\end{equation}
The correction ${(-c)}^{-\alpha-\beta}$ we have to do, is due to the fact, that we do not use probability densities, but are in the more general space of (particle) distributions.
In the actual implementation, we will nearly always calculate with $c=1$, so even this small difference vanishes.

As mentioned in Section~\ref{sub: Cumulants}, the cumulants are the coefficients on the Taylor expansion of the logarithm of the moment generating function.
With the aforesaid correction, this yields for the cumulant $\kappa_{\alpha\beta}$:
\begin{equation}
  \label{eq: definition of cumulants}
  \kappa_{\alpha\beta} = {(-c)}^{-\alpha-\beta} \frac{\partial^\alpha\partial^\beta}{{(\partial \Xi_1)}^\alpha{(\partial \Xi_2)}^\beta} \ln(F(\Xi_1, \Xi_2))\Bigr|_{\Xi_1=\Xi_2 = 0}.
\end{equation}
