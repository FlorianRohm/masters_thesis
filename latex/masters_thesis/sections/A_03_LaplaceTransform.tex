% !TEX root = ../thesis.tex

The two sided Laplace transform $\mathcal{\hat{L}}$ of a function $g\colon \R^n \mapsto \R$ is defined, c.f.~\cite{weissteinLaplace}, as
\begin{equation}
  \label{eq: Definition of Laplace}
  \mathcal{\hat{L}}[g](\vec{\Xi}) = \int_{\R^n} g(\vec{\xi}) e^{-\vec{\Xi} \cdot \vec{\xi}}d\vec{\xi}.
\end{equation}
To get the moment generating function from a density distribution, we have to introduce a similar integral transformation $\mathcal{L}$ with
\begin{equation}
  \label{eq: Definition of integral transformation}
  \mathcal{L}[g](\vec{\Xi}) = \mathcal{\hat{L}}[g](-\vec{\Xi}) = \int_{\R^n} g(\vec{\xi}) e^{\vec{\Xi} \cdot \vec{\xi}}d\vec{\xi}.
\end{equation}

For the translation of $g$ by a vector $\vec{u}$ we get
\begin{equation}
  \label{eq: translated integral}
  \begin{aligned}
    \mathcal{L}[g(\cdot - \vec{u})](\vec{\Xi})
    & = \int_{-\infty}^\infty g(\vec{\xi}-\vec{u}) e^{\vec{\Xi} \cdot \vec{\xi}}d\vec{\xi} \\
    & = \int_{-\infty}^\infty g(\vec{\xi}) e^{\vec{\Xi} \cdot (\vec{\xi}+\vec{u})}d\vec{\xi} \\
    & =  e^{\vec{\Xi} \cdot \vec{u}} \int_{-\infty}^\infty g(\vec{\xi}) e^{\vec{\Xi} \cdot \vec{\xi}}d\vec{\xi} \\
    & =  e^{\vec{\Xi} \cdot \vec{u}} \mathcal{L}[g](\vec{\Xi})
  \end{aligned}
\end{equation}
