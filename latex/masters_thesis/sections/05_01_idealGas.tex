% !TEX root = ../thesis.tex
Our first assumption is, that the ideal gas law shall hold, connecting pressure and density.
As our small stencil can not represent temperature, we can assume it's connected by a constant factor $\theta$.
In fact, we already computed this link in equation~\eqref{eq: equilibrium cumulants}, where we stated
\begin{equation}
  \begin{aligned}
    \kappa_{20}^{eq} & = c_s^2  \\
    \kappa_{02}^{eq} & = c_s^2
  \end{aligned}
\end{equation}
which we can multiply by $m_{00}^{eq} = m_{00}$ and again identify $c_s^2=\frac{1}{3}$ to get
\begin{equation}
\label{eq: pressure density connection}
  \begin{aligned}
    K_{20}^{eq} & = \frac{m_{00}}{3}  \\
    K_{02}^{eq} & = \frac{m_{00}}{3}.
  \end{aligned}
\end{equation}
Identifying the normalized cumulants with the central moments via equation~\eqref{eq: all normalized cumulants from central moments}, we have the desired statement of $\theta=\frac{1}{3}$.
Interestingly enough, $\theta$ is a function of the speed of sound $c_s$ and hence the temperature $T$, like in the continuous ideal gas law, $p=\rho RT$.
This is yet another hint at our choice of using cumulants being good.

To use equation~\eqref{eq: pressure density connection} in our calculations, we still have to get the asymptotic expansion.
Luckily, we already did this in equations~\eqref{eq: expansions of cumulants zeroth order} through~\eqref{eq: expansions of cumulants second order}, yielding
\begin{align}
  \label{eq: ideal gas zeroth order}
  \theta m_{00}^{(0)} &= m_{20}^{(0)} = m_{02}^{(0)}
  \\
  \label{eq: ideal gas first order}
  \theta m_{00}^{(1)} &= m_{20}^{(1)} = m_{02}^{(1)}
  \\
  \label{eq: ideal gas second order}
  \theta m_{00}^{(2)} &= m_{20}^{(2)} - \frac{m_{10}^{{(1)}^2}}{m_{00}^{(0)}}  = m_{02}^{(1)} - \frac{m_{01}^{{(1)}^2}}{m_{00}^{(0)}}.
\end{align}
