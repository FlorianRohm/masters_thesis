% !TEX root = ../thesis.tex
Our first assumption is, that the ideal gas law shall hold, connecting pressure and density.
As our small stencil can not represent temperature, we can assume it's connected by a constant factor $\theta$.
In fact, we already computed this link in equation~\eqref{eq: equilibrium cumulants}, where we stated
\begin{equation}
  \begin{aligned}
    \kappa_{20}^{eq} & = c_s^2  \\
    \kappa_{02}^{eq} & = c_s^2
  \end{aligned}
\end{equation}
which we can multiply by $m_{00}^{eq} = m_{00}$ to get
\begin{equation}
\label{eq: pressure density connection}
  \begin{aligned}
    K_{20}^{eq} & = c_s^2 m_{00}  \\
    K_{02}^{eq} & = c_s^2 m_{00}.
  \end{aligned}
\end{equation}
Identifying the normalized cumulants with the central moments via equation~\eqref{eq: all normalized cumulants from central moments}, we have the desired statement of $\theta=c_s^2$.
Interestingly enough, $\theta$ is a function of the speed of sound $c_s$ and hence the temperature $T$, like in the continuous ideal gas law\footnote{$\gamma$ is the isentropic exponent },
\begin{equation}
  p=\rho RT = \rho \frac{c_s^2}{\gamma}.
\end{equation}
In contrary to the theory of gases, where $\gamma$ is strictly greater than one and a function of the temperature, our method calculates with $\gamma=1$.


To use equation~\eqref{eq: pressure density connection} in our calculations, we still have to get the asymptotic expansion.
Luckily, we already did this in equations~\eqref{eq: eq moments 20_0} and~\eqref{eq: eq moments 02_0} for the zeroth order and equations~\eqref{eq: eq moments 20_2} and~\eqref{eq: eq moments 02_2} for the second order.
The first order gives no new information, as these equations just result in $0=0$.
