% !TEX root = ../thesis.tex

Like in Section~\ref{sub: Summarizing the transformations}, we can use this point to recapitulate and take a look at the whole algorithm.
Starting from an initial set\footnote{
One way would be to prescribe a velocity and density for each cell, use them as equilibrium cumulants and calculate the \glspl{pdf}.
Another widely used way is to continue from a previous calculation.
} of \glspl{pdf}, we can stream the distributions in the neighboring cells, like described in Section~\ref{sub: Stream and Collide}.
The newly arrived \glspl{pdf} are now transformed into normalized cumulants and the post collision normalized cumulants are calculated with the collision rules from equations~\eqref{eq: final collision one relaxation}, presented in this section.
Now we can transform them back to our original space and repeat the whole process.

The remaining sections will now all cover the recovery of the Navier-Stokes equations from the cumulant LBM we described.
