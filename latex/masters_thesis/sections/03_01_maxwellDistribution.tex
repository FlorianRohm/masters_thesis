% !TEX root = ../thesis.tex
The transformation of the Maxwell-distribution~\eqref{eq: maxwell distribution raw} is done in multiple steps.
To prepare for the following calculation, we integrate\footnote{c.f.~\cite{gitrepo_MastersThesis} mathematica/integrate\_maxwell.nb}
\begin{equation}
  \label{eq:integrate exponential sage}
  \begin{aligned}
    \int_{-\infty}^{\infty} \exp \left(-C \xi_i^2 + (2Cu_i + \Xi_i)\cdot\xi_i \right) d\xi_i
    & = \sqrt{\frac{\pi}{C}}\exp \left( \frac{{(2Cu_i + \Xi_i)}^2}{4C}\right),\quad i=1,2
    .
  \end{aligned}
\end{equation}

With this sorted out, we can calculate the moment generating function of the Maxwell distribution~\eqref{eq: maxwell distribution raw} with
\begin{equation}
  C \defined \frac{m}{2 k_B T}
\end{equation}
as
\begin{equation}
  \label{eq: laplace of maxwell}
  \begin{aligned}
    F^{eq}(\vec{\Xi}) & = \mathcal{L}[f^{\text{maxwell}}](\vec{\Xi})
    \\& = \rho
      \exp \left( \Xi_1 u_1 + \Xi_2 u_2 + \frac{1}{4C}\left(\Xi_1^2 + \Xi_2^2 \right)\right).
  \end{aligned}
\end{equation}
The full derivation of~\eqref{eq: laplace of maxwell} is in the appendix, namely Appendix~\ref{appendix: Laplace transform of Maxwellian distribution}.
