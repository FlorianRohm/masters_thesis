% !TEX root = ../thesis.tex
The transformation of~\eqref{eq: maxwell distribution raw} is done in multiple steps.
To prepare for the following calculation, we integrate\footnote{cf.\ mathematica/integrate\_maxwell.nb}
\begin{equation}
  \label{eq:integrate exponential sage}
  \begin{aligned}
    \int_{-\infty}^{\infty} \exp \left(-C \xi^2 + (2Cu - \Xi)\cdot\xi \right) d\xi
    & = \sqrt{\frac{\pi}{C}}\exp \left(Cu^2 - \Xi u + \frac{1}{4} \frac{\Xi^2}{C}\right) \\
    & = \sqrt{\frac{\pi}{C}}\exp \left( \frac{{(2Cu-\Xi)}^2}{4C}\right).
  \end{aligned}
\end{equation}

With this sorted out, we can calculate the Laplace Transform of the Maxwell Distribution~\eqref{eq: maxwell distribution raw} with
\begin{equation}
  C \defined \frac{m}{2 k_B T}
\end{equation}
like
\begin{equation}
  \label{eq: laplace of maxwell}
  \begin{aligned}
    F^{eq}(\Xi) & = \mathcal{L}[f^{\text{maxwell}}](\Xi)
    \\& = \rho
      \exp \left( -\Xi_1 u_1 - \Xi_2 u_2 + \frac{1}{4C}\left(\Xi_1^2 + \Xi_2^2 \right)\right).
  \end{aligned}
\end{equation}
As usual, the full derivation is in the appendix, namely Appendix~\ref{appendix: Laplace transform of Maxwellian distribution}.
