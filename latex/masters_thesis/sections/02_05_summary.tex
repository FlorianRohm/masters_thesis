% !TEX root = ../thesis.tex

After we worked through the whole table of transformations in this section, it's time to take a step back and look on what we did and why.

Before collision, we have our particle distribution functions $f_{ij}^{\circ}$.

Through equations~\eqref{eq: density definition} to~\eqref{eq: y velocity definition} we get our macroscopic variables $\rho$ and $\vec{u}$ in this cell.
As our collision will be mass and momentum preserving\footnote{Ignoring gravity or other forces which can alter velocity in the collision.}, it does not matter if those are computed pre-collision.
Using the macroscopic variables, central moments are computed via~\eqref{eq:fast forward c i pipe beta matrix} and afterwards~\eqref{eq: fast forward c alpha beta matrix}.
Not much has to be done to get our normalized cumulants, as seen in~\eqref{eq: all normalized cumulants from central moments}.
Those will be collided according to the procedure presented in the next section.

After the collision, we calculate our post-collision central moments from~\eqref{eq: all central moments from normalized cumulants}
and transform them back to \glspl{pdf} via~\eqref{eq:fast backward c i pipe beta matrix} and~\eqref{eq:fast backward f i pipe beta matrix}.

The alert reader may have noticed, that we tediously calculated the relation between normalized cumulants and moments in~\eqref{eq: K 00 from moments} through~\eqref{eq: K 22 from moments} but apparently never used them up to now.
This labor was not in vain, as we will need those expressions, when we prepare the analysis of the method in Section~\ref{sec: Series expansions}.
