% !TEX root = ../thesis.tex
This section will lay the foundation of the analysis by expanding~\eqref{eq: lattice boltzmann equation} into a Taylor series, identifying and expanding the moments.

First of all,~\eqref{eq: lattice boltzmann equation} is reformulated to
\begin{equation}
  \label{eq: lattice boltzmann equation shifted}
  f_{ij}^{\circ}(x, y, t + \Delta t) = f^*_{ij}(x - i \Delta x , y - j\Delta y, t), \quad\forall i,j\in \{-1, 0, 1\}
\end{equation}
by shifting $x$ and $y$. Now, a Taylor expansion of~\eqref{eq: lattice boltzmann equation shifted} in both time and space gives
\begin{equation*}
  \sum_{\tau = 0}^\infty \frac{{\Delta t}^\tau }{\tau!} \frac{\partial^\tau}{{\partial t}^\tau} f_{ij}^{\circ}(x, y, t) =
  \sum_{m,n = 0}^\infty \frac{{(-i\Delta x)}^m{(-j\Delta y)}^n} {m!n!} \frac{\partial^m \partial^n}{ {\partial x}^m{\partial y}^n} f^*_{ij}(x, y, t)
\end{equation*}
which is equivalent to
\begin{equation}
  \label{eq: Taylor LB1}
  \sum_{\tau = 0}^\infty \frac{{\Delta t}^\tau }{\tau!} \frac{\partial^\tau}{{\partial t}^\tau} f_{ij}^{\circ}(x, y, t) =
    \sum_{m,n = 0}^\infty \frac{{(-\Delta t)}^{m+n}} {m!n!} \frac{\partial^m \partial^n}{ {\partial x}^m{\partial y}^n} {(ci)}^m {(cj)}^n f^*_{ij}(x, y, t),
\end{equation}
using~\eqref{eq: relation between stepsizes}.

In formulating Equation~\eqref{eq: Taylor LB1}, we did not need to restrict ourselves to specific $f_{ij}$ and thus it has to hold for all of them. Notably also for all linear combinations
\begin{equation*}
  \sum_{i,j}\lambda_{ij} f_{ij},
\end{equation*}
especially for those, which yield the moments, $m_{\alpha\beta} $, of $f$ as
\begin{equation}
  \sum_{i,j}\lambda_{ij} f_{ij} = \sum_{i,j}c^{\alpha + \beta}i^\alpha j^\beta f_{ij}=m_{\alpha\beta}.
\end{equation}
Therefore, we rewrite the system of equations for the $f_{ij}$ to a system of equations for the $m_{\alpha\beta}$ like
\begin{align}
    \nonumber
    & & \sum_{ij} \sum_{\tau = 0}^\infty \frac{{\Delta t}^\tau }{\tau!} \frac{\partial^\tau}{{\partial t}^\tau} {(ci)}^\alpha {(cj)}^\beta f_{ij}^{\circ}
    &=
    \sum_{ij}\sum_{m,n = 0}^\infty \frac{{(-\Delta t)}^{m+n}} {m!n!} \frac{\partial^m \partial^n}{ {\partial x}^m{\partial y}^n} {(ci)}^m {(cj)}^n {(ci)}^\alpha {(cj)}^\beta f^*_{ij}
    \\ \nonumber &\equivalent &
    \sum_{\tau = 0}^\infty \frac{{\Delta t}^\tau }{\tau!} \frac{\partial^\tau}{{\partial t}^\tau} \sum_{ij}{(ci)}^\alpha {(cj)}^\beta f_{ij}^{\circ}
    &=
    \sum_{m,n = 0}^\infty \frac{{(-\Delta t)}^{m+n}} {m!n!} \frac{\partial^m \partial^n}{ {\partial x}^m{\partial y}^n}\sum_{ij}{(ci)}^{(\alpha+m)} {(cj)}^{(\beta+n)} f^*_{ij}
    \\ \nonumber &\equivalent &
     \sum_{\tau = 0}^\infty \frac{{\Delta t}^\tau }{\tau!} \frac{\partial^\tau}{{\partial t}^\tau} m_{\alpha\beta}^{\circ} &=
    \sum_{m,n = 0}^\infty \frac{{(-\Delta t)}^{m+n}} {m!n!} \frac{\partial^m \partial^n}{ {\partial x}^m{\partial y}^n} m^*_{(\alpha + m)(\beta + n)}
     \\   &\equivalent &
     \sum_{\tau = 0}^\infty \frac{{\Delta t}^\tau }{\tau!} \frac{\partial^\tau}{{\partial t}^\tau} \frac{m_{\alpha\beta}^{\circ}}{c^{\alpha + \beta}} &=
    \sum_{m,n = 0}^\infty \frac{{(- c\Delta t)}^{m+n}} {m!n!} \frac{\partial^m \partial^n}{ {\partial x}^m{\partial y}^n} \frac{m^*_{(\alpha + m)(\beta + n)}}{c^{\alpha + m + \beta + n}}.
\end{align}
To make our lives a bit easier, we will rescale the moments according to their order,
\begin{equation}
  \label{eq: rescaling the moments}
  \frac{m_{\alpha\beta}}{c^{\alpha + \beta}} \rightarrow m_{\alpha\beta},
\end{equation}
to finally get
\begin{equation}
  \label{eq: Taylor of moments}
  \sum_{\tau = 0}^\infty \frac{{\Delta t}^\tau }{\tau!} \frac{\partial^\tau}{{\partial t}^\tau} m_{\alpha\beta}^{\circ} =
 \sum_{m,n = 0}^\infty \frac{{(- c\Delta t)}^{m+n}} {m!n!} \frac{\partial^m \partial^n}{ {\partial x}^m{\partial y}^n} m^*_{(\alpha + m)(\beta + n)}.
\end{equation}
This scaling is easily reverted by multiplying with same term again, but most of the time, we will calculate with $c=1$, so the effect of the scaling vanishes completely.
Additionally, this will take out the $c$-term in the aliasing of the moments,~\eqref{eq:aliasing of moments 1} and~\eqref{eq:aliasing of moments 2}, making it a $1:1$ relationship.

Finally, as the speed of sound squared is calculated as $\frac{dp}{d\rho}$, c.f.~\cite[page 175]{wolf2000lattice}, $c_s$ needs also to be rescaled.
As the pressure is a second order (central) moment it gets rescaled with $c^{-2}$, according to~\eqref{eq: rescaling the moments} whereas the density is the zeroth order moment which undergoes no scaling.
Hence, we have
\begin{equation}
   c_s^2 = \frac{1}{3}
\end{equation}

\subsubsection{Nondimensionalization}
\label{subs:Nondimensionalization}

For the scale analysis, we introduce a dimensionless scaling parameter $\epsilon$.
Using the characteristic length and timescales, $L$ and $\iota$, we adapt the widely used diffusive scaling
\begin{equation}
  \label{eq: nondimensionalisation}
  \begin{aligned}
    \Delta x & = L\epsilon \\
    \Delta t & = \iota\epsilon^2
  \end{aligned}
\end{equation}
and therefore
\begin{equation}
  \label{eq: nondimensionalisation 2}
  c = \frac{L}{\iota\epsilon}.
\end{equation}
Our spatial coordinates $x$ and $y$ together with the time coordinate $t$ will be nondimensionalized according to their characteristic length scales like
\begin{equation}
  \label{eq: nondimensionalisation 3}
  \begin{aligned}
    \{x, y\} & \rightarrow \{\frac{x}{L}, \frac{y}{L}\} \\
    t & \rightarrow \frac{t}{\iota}.
  \end{aligned}
\end{equation}
Furthermore, the moments are asymptotically expanded in space according to the scale $\epsilon$
\begin{align}
    \label{eq: expansion of m}
    m_{\alpha\beta} & = \sum_{p=0}^{\infty} \epsilon^p m_{\alpha\beta}^{\circ^{(p)}}
\end{align}
This procedure implicitly defines the coefficients of the series, $m_{\alpha\beta}^{\circ^{(p)}}$.
Those are still dependent of space and time, but as they are separated on their length scales, their spatial part is not allowed to depend on $\epsilon$ neither implicitly nor explicitly.

Finally, we can insert~\eqref{eq: expansion of m} and~\eqref{eq: nondimensionalisation}
through~\eqref{eq: nondimensionalisation 3} into~\eqref{eq: Taylor of moments}, yielding

\begin{alignat}{3}
  \nonumber
  &&\sum_{\tau = 0}^\infty \frac{{\Delta t}^\tau }{\tau!} \frac{\partial^\tau}{{\partial t}^\tau} m_{\alpha\beta}^{\circ} &=
 \sum_{m,n = 0}^\infty \frac{{(- c\Delta t)}^{m+n}} {m!n!} \frac{\partial^m \partial^n}{ {\partial x}^m{\partial y}^n} m^*_{(\alpha + m)(\beta + n)}\\
  \nonumber
  &\equivalent& \sum_{\tau = 0}^\infty \frac{{\Delta t}^\tau }{\tau!}  \frac{\partial^\tau}{{\partial t}^\tau} \sum_{p=0}^{\infty} \epsilon^p m_{\alpha\beta}^{\circ^{(p)}}
    = &{}\sum_{m,n = 0}^\infty \frac{{(-c\Delta t)}^{m+n}} {m!n!} \frac{\partial^m \partial^n}{ {\partial x}^m{\partial y}^n} \sum_{p=0}^{\infty} \epsilon^p m_{(\alpha + m)(\beta + n)}^{*^{(p)}}
  \\
  %
    \label{eq: nondimensionalised expansion series}
  &\equivalent&
    \sum_{\tau = 0}^\infty \frac{{(\iota\epsilon^2)}^\tau }{\tau!} \frac{\partial^\tau}{{\partial t}^\tau \iota^\tau} \sum_{p=0}^{\infty} \epsilon^p m_{\alpha\beta}^{\circ^{(p)}}
    = &{}\sum_{m,n = 0}^\infty \frac{{(-\frac{L}{\iota\epsilon}(\iota\epsilon^2))}^{m+n}} {m!n!}
    \frac{\partial^m \partial^n}{ {\partial x}^m{\partial y}^n L^{m+n}} \sum_{p=0}^{\infty} \epsilon^p m_{(\alpha + m)(\beta + n)}^{*^{(p)}}
   \\
 %
  \nonumber
  &\equivalent&
    \sum_{\tau = 0}^\infty \frac{\epsilon^{2\tau} }{\tau!} \frac{\partial^\tau}{{\partial t}^\tau} \sum_{p=0}^{\infty} \epsilon^p m_{\alpha\beta}^{\circ^{(p)}}
    = &{}\sum_{m,n = 0}^\infty \frac{{(-\epsilon)}^{m+n}} {m!n!}
    \frac{\partial^m \partial^n}{ {\partial x}^m{\partial y}^n } \sum_{p=0}^{\infty} \epsilon^p m_{(\alpha + m)(\beta + n)}^{*^{(p)}}
   \\
 %
    \label{eq: final nondimensionalised expansion series}
  &\equivalent&
    \sum_{\tau, p = 0}^\infty\epsilon^{2\tau + p} \frac{1} {\tau!} \frac{\partial^\tau}{{\partial t}^\tau} m_{\alpha\beta}^{\circ^{(p)}}
    = &{} \sum_{m,n,p' = 0}^\infty  \epsilon^{m+n+p'} \frac{{(-1)}^{m+n}} {m!n!}
    \frac{\partial^m \partial^n}{ {\partial x}^m{\partial y}^n } m_{(\alpha + m)(\beta + n)}^{*^{(p')}},
\end{alignat}
where in~\eqref{eq: final nondimensionalised expansion series}, the summation indices are changed to be unique for better recognition in Section~\ref{sub: Matching the coefficients}.

\subsubsection{Changing the coordinate system}
\label{subs: Changing the coordinate system}
Our choice of coordinates to this point was arbitrary.
Something interesting happens though, when we mirror each axis.
The general problem remains the same, but all involved velocities will change their sign, i.e.\ $c$ gets replaced with $-c$.
Now, our moments $\tilde{m}$ in the new coordinate system are
\begin{align}
  \tilde{m}_{\alpha\beta} &= \sum_{i,j} {(-ci)}^\alpha {(-cj)}^\beta f_{ij}
  ={(-1)}^{\alpha + \beta} \sum_{i,j} {(ci)}^\alpha {(cj)}^\beta f_{ij}
  ={(-1)}^{\alpha + \beta} m_{\alpha\beta}
\end{align}
but still represent the same value.
To get the same expansion in the nondimensionalized case, c.f.~\eqref{eq: nondimensionalisation}, our expansion parameter $\epsilon$ also needs to be negated and hence
\begin{equation}
     \sum_{p=0}^{\infty} {(-\epsilon)}^p \tilde{m}_{\alpha\beta}^{(p)}
     = \tilde{m}_{\alpha\beta}
     = {(-1)}^{\alpha + \beta}  m_{\alpha\beta}
     = {(-1)}^{\alpha + \beta} \sum_{p=0}^{\infty} \epsilon^p m_{\alpha\beta}^{(p)}.
\end{equation}
As we corrected the error we made with the introduction of $-\epsilon$, the expanded moments $\tilde{m}_{\alpha\beta}^{(p)}$ and $m_{\alpha\beta}^{(p)}$ have to be equal and thus
\begin{align}
  \label{eq: negativ expansion}
  \sum_{p=0}^{\infty} {(-\epsilon)}^p m_{\alpha\beta}^{(p)}
  = {(-1)}^{\alpha + \beta} \sum_{p=0}^{\infty} \epsilon^p m_{\alpha\beta}^{(p)}.
\end{align}
The implications from equation~\eqref{eq: negativ expansion} can be separated between even and odd order moments.
Even order moments, like the density $m_{00}$, have to fulfill
\begin{align}
  \label{eq: negativ expansion even}
  \sum_{p=0}^{\infty} {(-\epsilon)}^p m_{\alpha\beta}^{(p)}
  = \sum_{p=0}^{\infty} \epsilon^p m_{\alpha\beta}^{(p)} \quad \forall \alpha+\beta=2k, k\in\N
\end{align}
and thus all odd order parts of the expansion have to be zero, e.g.
\begin{equation}
  \label{eq: kill even moments odd order}
  \begin{aligned}
    m_{\alpha\beta}^{(p)} = 0 \quad
    \forall\alpha+\beta & = 2k, k\in\N, \\
    \forall p & = 2l+1, l\in\N.
  \end{aligned}
\end{equation}
With exactly the same reasoning, the odd order moments like $m_{10}$ may only have odd orders in their expansion, e.g.
\begin{equation}
  \label{eq: kill odd moments even order}
  \begin{aligned}
    m_{\alpha\beta}^{(p)} = 0 \quad
    \forall\alpha+\beta & = 2k+1, k\in\N, \\
    \forall p & = 2l, l\in\N.
  \end{aligned}
\end{equation}
The most prominent examples will be the density
\begin{equation}
  \label{eq: first order pressure zero}
  m_{00}^{(1)}=0
\end{equation}
and the velocities
\begin{equation}
  \label{eq: zeroth order velocity zero}
  m_{10}^{(0)} = m_{01}^{(0)} = 0
\end{equation}
and
\begin{equation}
  \label{eq: second order velocity zero}
  m_{10}^{(2)} = m_{01}^{(2)} = 0
\end{equation}
