% !TEX root = ../thesis.tex
After the presentation and analysis of the cumulant \gls{lbm}, it's time to test whether this method delivers.

We will have a look at three problems, increasingly complex and challenging.

The code is written in python 2.7.6 and based on the sample codes on the homepage of the open source framework palabos.

Another highly important topic in CFD and especially \gls{lbm} are boundaries.
After the streaming, the~\glspl{pdf} which would originate outside the computational domain are unknown.
At solid boundaries, the bounce back scheme\footnote{A nice overview can be found in~\cite{boundaries}} is most dominant, simply reflecting the unknown distributions.
For the inlet, a velocity bounce back scheme by Yu 2002
\todo[inline]{find paper, cite -.-}
is chosen, the outflow is a simple extrapolation scheme for the unknown distributions.
We forgo an analysis on the pros and cons of the boundaries, as this is besides the collision the second large subfield in \gls{lbm} where research is definitely not finished.
