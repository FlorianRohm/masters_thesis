% !TEX root = ../thesis.tex
After the presentation and analysis of the cumulant \gls{lbm}, it's time to test whether this method delivers.

We will have a look at three problems, increasingly complex and challenging and compare it to \gls{srt}\footnote{Our method will be called `Cumulant' in this section.}.

The code is written in python 2.7.6 and based on the sample codes on the \href{http://wiki.palabos.org/numerics:codes}{homepage} of the open source framework palabos.
For the ones interested and to verify the results, the whole code can be found on github.

Besides the collision, boundaries are another highly important topic in CFD and especially \gls{lbm}.
After the streaming, the~\glspl{pdf} which would originate outside the computational domain are unknown.
At solid boundaries, the bounce back scheme\footnote{A nice overview can be found in~\cite{boundaries}.} is most dominant, simply reflecting the known distributions.
For the inlet, a velocity bounce back scheme, c.f.~\cite{yu2003viscous} is chosen, the outflow is a simple extrapolation for the unknown distributions.
We forgo an analysis on the pros and cons of the boundaries, as this is besides the collision the second large subfield in \gls{lbm} where research is definitely not finished for a while.
