% !TEX root = ../thesis.tex
After describing the general structure of the LBM, this section will cover another important question of \gls{cfd}:\@ What do we want to know at the end?

The most basic thing one is interested in, is information about the flow field, i.e.\@ \textbf{density} $\rho$, \textbf{velocity}\footnote{In contrary to the microscopic velocities, denoted as $\vec{v}$ or $\vec{c}_{ij}$ in the discrete case, the averaged, macroscopic velocity will be denoted by $\vec{u}$.} $\vec{u}$ and \textbf{temperature} $T$ in the domain, to analyze the flow patterns.
Those are all averaged quantities and thus called macroscopic values.

When calculating with the Navier-Stokes equations, getting the macroscopic values is easy, as the equations are stated in exactly those.
Unfortunately, the Boltzmann equation and also the discretized equation~\eqref{eq: lattice boltzmann equation} calculate the particle distributions, describing the microscopic behavior.\glsadd{micVar}\glsadd{macVar}
The extraction of macroscopic values out of the available \glspl{pdf} will be treated in this section.

Before introducing (macroscopic) velocity and density, we have to take a small detour to the field of probability theory and introduce moments.
In analogy to bivariat probability distributions, we define the characteristic variables $m_{\alpha\beta}$, called \textbf{moments}, of a \gls{pdf} $f$ as
\begin{equation}
  \label{eq: moments continuous}
  m_{\alpha\beta}(\vec{x},t) \defined  \int v_x^\alpha v_y^\beta f(\vec{v},\vec{x},t) d\vec{v},\quad\forall\alpha,\beta\in\N,
\end{equation}\glsadd{n}
where $v_x$ and $v_y$ are the components of the particle velocity $\vec{v}$.
In contrary to probability distributions, the zeroth moment $m_{00}$ of $f$ is not one.
As those distributions will embody our microscopic variables, we will also exclude the outlier of $m_{00}=0$, as this would mean that there are no particles in this region.
The first moments $m_{10}$ and $m_{01}$ are still the distributions mean values in the corresponding directions.

The \glspl{pdf} and hence all derived characteristic variables will depend on space and time in the rest of this thesis.
For the sake of simplicity this dependence is going to be omitted every time when not strictly required and implicitly assumed. Hence
\begin{equation}
  \label{eq: leave out x and y}
  m_{\alpha\beta} \defined m_{\alpha\beta}(\vec{x},t).
\end{equation}
This is also applied to all other values like density and velocity.

Another set of characteristic variables are the centralized or \textbf{central moments} $c_{\alpha\beta}$ defined as
\begin{equation}
  \label{eq: central moments continuous}
  c_{\alpha\beta} \defined \int {\left(v_x - \frac{m_{10}}{m_{00}}\right)}^\alpha {\left(v_y-\frac{m_{01}}{m_{00}}\right)}^\beta f(\vec{v})  d\vec{v},\quad\forall\alpha,\beta\in\N.
\end{equation}
Here, the assumption associated with~\eqref{eq: leave out x and y} of course also holds, making the centralized moments dependent on space and time.

As deduced in~\cite[pages 23 ff.]{harris2004introduction}, we can calculate the density $\rho$, the velocity $\vec{u}=\begin{pmatrix}u_x\\u_y\end{pmatrix}$ and the
\textbf{pressure tensor} $\tensor{P}=\begin{pmatrix}\tensor{P}_{xx}&\tensor{P}_{xy}\\ \tensor{P}_{yx}&\tensor{P}_{yy}\end{pmatrix}$ as

\begin{equation}
  \label{eq: physical quantities continuous}
  \begin{aligned}
    \rho & = \int f(\vec{v}) d\vec{v} \\
    u_x
    & = \frac{1}{\rho}\int v_x f(\vec{v}) d\vec{v} \\
    u_y
    & = \frac{1}{\rho}\int v_y f(\vec{v}) d\vec{v}\\
    \tensor{P}_{xy}=\tensor{P}_{yx}
    & = \int (v_x - u_x)(v_y - u_y) f(\vec{v}) d\vec{v} \\
    \tensor{P}_{xx}
    & = \int {(v_x - u_x)}^2 f(\vec{v}) d\vec{v} \\
    \tensor{P}_{yy}
    & = \int {(v_y - u_y)}^2 f(\vec{v}) d\vec{v},
  \end{aligned}
\end{equation}
and thus with~\eqref{eq: moments continuous} and~\eqref{eq: central moments continuous} we get
\begin{equation}
  \label{eq: physical quantities continuous moments}
\rho = m_{00}, \quad \vec{u}=\frac{1}{m_{00}}\begin{pmatrix}m_{10}\\m_{01}\end{pmatrix},\quad \tensor{P}=\begin{pmatrix}c_{20}&c_{11}\\ c_{11}&c_{02}\end{pmatrix}.
\end{equation}
From the relations in~\eqref{eq: physical quantities continuous moments}, we can also conclude, that centralized moments are indeed the moments of equation~\eqref{eq: moments continuous} when observed from a frame of reference moving with the flow velocity $\vec{u}$.
\glsadd{frame of reference}

Upon switching to the discrete case, we exchange the continuous microscopic velocity $\vec{v}$ with our velocity set~\eqref{eq: definition of the velocities}.
Thus, the integrals in~\eqref{eq: moments continuous} and~\eqref{eq: central moments continuous} will degenerate to sums, leaving us with
\begin{align}
  m_{\alpha\beta} &= \mquad\sum_{i,j \in \{-1,0,1\}}\mquad {(ic)}^\alpha {(jc)}^\beta f_{ij}
  \label{eq: moment definition}\\
  c_{\alpha\beta} &= \mquad\sum_{i,j \in \{-1,0,1\}}
  {\left(ci - \frac{m_{10}}{m_{00}}\right)}^\alpha
  {\left(cj - \frac{m_{01}}{m_{00}}\right)}^\beta f_{ij}.
  \label{eq: central moment definition}
\end{align}
Additionally, we derive a characteristic property of central moments, namely
\begin{equation}
  \label{eq: low central moments}
  c_{00}=m_{00},\quad c_{10} = c_{01}=0,
\end{equation}
by inserting the corresponding definitions.

When looking at the derived quantities, we have to pay attention to another issue:
In the continuous case, a local equilibrium is attained nearly instantaneously, c.f.~\cite[page 218]{smits2000physical}.
Therefore the relations in~\eqref{eq: physical quantities continuous} are a priori only valid in the local equilibrium.

Back in the discrete case, we can no longer assume aforementioned local equilibrium.
In fact, we already assume to be in a non equilibrium state, as we want the collision to shift our particle distributions a bit towards the local equilibrium.
Consequently, we can state the macroscopic quantities in~\eqref{eq: physical quantities continuous} only in the equilibrium case, like
\begin{align}
  \rho & = \mquad\sum_{i,j \in \{-1,0,1\}}\mquad f_{ij}^{eq} = m_{00}^{eq}
  \label{eq: density definition raw}\\
  u_x  & = \frac{1}{\rho} \sum_{i,j \in \{-1,0,1\}}\mquad ic f_{ij}^{eq} = \frac{m_{10}^{eq}}{m_{00}^{eq}}
  \label{eq: x velocity definition raw}\\
  u_y  & = \frac{1}{\rho} \sum_{i,j \in \{-1,0,1\}}\mquad jc f_{ij}^{eq} = \frac{m_{01}^{eq}}{m_{00}^{eq}}.
  \label{eq: y velocity definition raw}
\end{align}

Luckily, the density and macroscopic velocity should not be altered by the microscopic collisions, due to mass and momentum conservation laws\footnote{Not the ones of Navier-Stokes, but continuum mechanics, as we work with particle distributions.} and are thus always equal to the ones in the corresponding local equilibrium.
Hence, we can identify
\begin{align}
  \rho & = m_{00}
  \label{eq: density definition}\\
  u_x  & = \frac{m_{10}}{m_{00}}
  \label{eq: x velocity definition}\\
  u_y  & = \frac{m_{01}}{m_{00}},
  \label{eq: y velocity definition}
\end{align}
with arbitrary\footnote{In the equilibrium, as well as pre- or post-collision.} moments and write the central moments again as
\begin{align}
  c_{\alpha\beta} &= \mquad\sum_{i,j \in \{-1,0,1\}}\mquad
  {\left(ci - u_x\right)}^\alpha
  {\left(cj - u_y\right)}^\beta f_{ij}.
  \label{eq: central moment definition velocities}
\end{align}

When looking at pressure, e.g.\ the momentum of the fluid particles normal to an imaginary wall, the discrete case looks quite different.
\begin{figure}
\centering
% !TEX root = ../../thesis.tex
\begin{tikzpicture}

\draw (-5,-3) rectangle (5,3);

\draw (-1,2) -- (1,2) node[right, scale=0.8] {$A$};
\draw[->] (0,2) -- (0,2.5) node[right, scale=0.8] {$F$};

\draw (-4,0) node[circle,minimum size=0.4pt,draw] (A1) {};
\draw (4,0.2) node[circle,minimum size=0.4pt,draw] (B1) {};

\draw (-0.17,0) node[circle,minimum size=0.4pt,draw, fill=gray] (A2) {};
\draw (0.17,0.2) node[circle,minimum size=0.4pt,draw, fill=gray] (B2) {};

\draw (0,-1.8) node[circle,minimum size=0.4pt,draw, fill] (A3) {};
\draw (0,1.8) node[circle,minimum size=0.4pt,draw, fill] (B3) {};

\begin{pgfonlayer}{bg}
\draw[->, shorten <=0.1pt] (A1) -- (A2) --(A3);
\draw[->, shorten <=0.1pt] (B1) -- (B2) -- (B3) -- (-0.1,1);
\end{pgfonlayer}
\end{tikzpicture}

\caption{Path of two particles during a given time, where the decreasing brightness depicts the temporal progress. In the time before the collision, the plate $A$ does not experience any pressure, but afterwards the right particle collides with $A$, applying the force $F$, and thus a pressure, to it.}
\label{fig: billard table}
\end{figure}
Here we cannot assume collision invariance, as Figure~\ref{fig: billard table} illustrates.
Hence we can assume the discrete analogon only when the distributions are equal to the equilibrium distributions $f_{ij}^{eq}$ which are by definition collision invariant.
This leaves us with
\begin{equation}
  \begin{aligned}
    \tensor{P}_{xx} = c_{20}^{eq}
    & = \mquad\sum_{i,j \in \{-1,0,1\}}\mquad  {(ic - u_x)}^2 f_{ij}^{eq}
    \\ & =
    \mquad\sum_{i,j \in \{-1,0,1\}}\mquad  i^2 c^2 f_{ij}^{eq} - 2 u_x \mquad \sum_{i,j \in \{-1,0,1\}}\mquad  ic f_{ij}^{eq} + u_x^2 \mquad\sum_{i,j \in \{-1,0,1\}} f_{ij}^{eq}\mquad
    \\ & = m_{20}^{eq} - \frac{m_{10}^2}{m_{00}},
  \end{aligned}
\end{equation}
using~\eqref{eq: density definition} through~\eqref{eq: y velocity definition} and thus the pressure\footnote{It can be shown, that the pressure is equal to one third the trace of the pressure tensor and that its diagonal elements are equal.
The third entry $\tensor{P}_{zz}$ would be the pressure in $z$-direction, which has to be accounted for when 2D simulations are regarded as 3D where we assume the flow to be invariant in one direction.
As pressure is isotropic, the pressure is also extended in the last direction, even when the flow velocity is not. When defining this tensor, the $z$-components were left out to not provoke confusion.}
\begin{equation}
  \label{eq: pressure}
  \begin{aligned}
    p &= \tensor{P}_{xx} =c_{20}^{eq}= m_{20}^{eq} - \frac{m_{10}^2}{m_{00}}\\
     &= \tensor{P}_{yy} =c_{02}^{eq} = m_{02}^{eq} - \frac{m_{01}^2}{m_{00}}.
  \end{aligned}
\end{equation}

If we wanted to include equations for energy and thus heat into our analysis, we would need higher order moments like $m_{30}$.
But as we are working with the nine speed model D2Q9 as depicted in Figure~\ref{fig: D2Q9 stencil}, we can only host nine independent moments.
This is easily seen with
\begin{equation}
  m_{30} = \mquad\sum_{i,j \in \{-1,0,1\}}\mquad i^3 c^3 f_{ij} = c^2\mquad\sum_{i,j \in \{-1,0,1\}}\mquad ic i^2 f_{ij} = c^2 \mquad\sum_{i,j \in \{-1,0,1\}}\mquad ic f_{ij} = c^2 m_{10},
\end{equation}
or in general
\begin{equation}
  \label{eq:aliasing of moments 1}
  m_{\alpha\beta}
  = \mquad\sum_{i,j \in \{-1,0,1\}}\mquad c^{\alpha+\beta} i^\alpha j^\beta f_{ij}
  = \mquad\sum_{i,j \in \{-1,0,1\}}\mquad c^{\alpha+\beta}i^{(\alpha+2k)} j^\beta f_{ij}
  = c^{-2k} m_{(\alpha+2k)\beta} \quad \forall k,\beta, \forall \alpha \neq 0
\end{equation}
and
\begin{equation}
  \label{eq:aliasing of moments 2}
  m_{\alpha\beta}
  = \mquad\sum_{i,j \in \{-1,0,1\}}\mquad c^{\alpha+\beta} i^\alpha j^\beta f_{ij}
  = \mquad\sum_{i,j \in \{-1,0,1\}}\mquad c^{\alpha+\beta} i^\alpha j^{(\beta+2l)} f_{ij}
  = c^{-2l}m_{\alpha(\beta+2l)} \quad \forall l, \alpha, \forall \beta \neq 0,
\end{equation}
explicitly using the range of the sum.
According to~\eqref{eq:aliasing of moments 1} and~\eqref{eq:aliasing of moments 2}, the full list of independent moments together with the first dependent moments of the D2Q9 stencil is displayed in Table~\ref{table:D2Q9 moments}.

As most sums will be dedicated to the calculation of moments of any kind in this thesis, we will abbreviate
\begin{equation*}
  \sum_{i,j \in \{-1,0,1\}} \rightarrow \sum_{i,j}
\end{equation*}
for all sums over $i$ and/or $j$, hence summing the complete discrete velocity space.

\setlength{\tabcolsep}{4.4pt}
\begin{table} [t]
  \centering
  \begin{tabular}{r rr rrr rr r rrrr r}
    \toprule
    \multicolumn{9}{c}{independent moments} & \multicolumn{5}{c}{dependent moments with aliases}   \\
    \cmidrule(lr){1-9}\cmidrule(lr){10-14} \\
    $m_{00}$
    & $m_{10}$
    & $m_{01}$
    & $m_{11}$
    & $m_{20}$
    & $m_{02}$
    & $m_{21}$
    & $m_{12}$
    & $m_{22}$
    & $c^{-2}m_{30}$
    & $c^{-2}m_{03}$
    & $c^{-2}m_{31}$
    & $c^{-2}m_{13}$
    & \ldots \\
    &&&&&&&&
    & = $m_{10}$
    & = $m_{01}$
    & = $m_{11}$
    & = $m_{11}$ & \\
    \bottomrule
  \end{tabular}
  \caption{D2Q9 moments}\label{table:D2Q9 moments}
\end{table}
