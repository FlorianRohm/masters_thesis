% !TEX root = ../thesis.tex
Now, after describing the general structure of the LBM, we can inspect another important question of \gls{cfd}:\@ What do we want to know at the end?

The most basic thing one is interested in, is information about the flow field, i.e.\@ density, velocity and temperature in my domain, to analyze the flow patterns.
Those are called macroscopic values.

When calculating with Navier-Stokes-Equations, getting those values is easy, as the equations are stated in exactly those variables.
Unfortunately, the Boltzmann Equation and also the discretized equation~\eqref{eq: lattice boltzmann equation} calculate the microscopic particle distributions. The extraction of macroscopic values out of those distributions will be treated in this Section.

Before introducing velocity and density, we have to take a small detour to the field of probability theory and introduce moments. The moment $m_{\alpha\beta}$ of a bivariat probability distribution $f$ is a characteristic variable, defined as
\begin{equation}
  m_{\alpha\beta}(\vec{x},t) \defined \int\int i^\alpha j^\beta f(i,j,\vec{x},t) didj.
\end{equation}
The zeroth moment $m_{00}$ of a probability is by definition one, the first moments $m_{10}$ and $m_{10}$ are its mean values in $i$ respectively $j$-direction.
The choice of $i$ and $j$ for continuous variables is intentional to avoid confusion, as we will later apply this definition to $f_{ij}$ and $f_{ij}^*$.
Another set of characteristic variables are the centralized moments $c_{\alpha\beta}$ defined as
\begin{equation}
  c_{\alpha\beta}(\vec{x},t) \defined \int\int {\left(i - \frac{m_{10}(\vec{x},t)}{m_{00}(\vec{x},t)}\right)}^\alpha {\left(j-\frac{m_{01}(\vec{x},t)}{m_{00}(\vec{x},t)}\right)}^\beta f(i,j) didj.
\end{equation}
As $m_{00}=1$ for probability distributions, the denominator seems superfluous. However, the particle distributions we will handle in a few moments do not satisfy this condition and thus the term is already introduced here.

Additionally, if a quantity only depends on $\vec{x}$ and $t$, the arguments will be omitted and implicitly assumed in the rest of the thesis, e.g.
\begin{equation}
    m_{\alpha\beta} \defined m_{\alpha\beta}(\vec{x},t)
\end{equation}
and
\begin{equation}
    c_{\alpha\beta} \defined c_{\alpha\beta}(\vec{x},t).
\end{equation}

As deduced in~\cite[pages 23 ff.]{harris2004introduction}, we can describe the macroscopic density $\rho$, the macroscopic velocity $\vec{u}$ and the pressure tensor $\tensor{P}$ as
\begin{equation}
  \begin{aligned}
    \rho & = \int f (\vec{v},\vec{x},t) d\vec{v} = m_{00} \\
    u_x
    & = \frac{1}{\rho}\int v_x f(\vec{v},\vec{x},t) d\vec{v} = \frac{ m_{10} }{ m_{00} }\\
    u_y
    & = \frac{1}{\rho}\int v_y f(\vec{v},\vec{x},t) d\vec{v} = \frac{ m_{10} }{ m_{00} }.\\
    \tensor{P}_{xy}=\tensor{P}_{yx}
    & = \int (v_x - u_x)(v_y - u_y) f(\vec{v},\vec{x},t) d\vec{v}
      = c_{11} \\
    \tensor{P}_{xx}
    & = \int {(v_x - u_x)}^2 f(\vec{v},\vec{x},t) d\vec{v}
      = c_{20} \\
    \tensor{P}_{yy}
    & = \int {(v_y - u_y)}^2 f(\vec{v},\vec{x},t) d\vec{v}
      = c_{02} .
  \end{aligned}
\end{equation}
Those connections do hold for the equilibrium case, which is locally attained nearly instantaneously for the continuous case, c.f.~\cite[page 218]{smits2000physical}. In the discrete case, this 


%When going further, one maybe wants to know the drag force applied to a body in the flow or the heat input into a specific surface.
