% !TEX root = ../thesis.tex
Up until now, we did not specify the collision part of~\eqref{eq: lattice boltzmann equation}.
As stated in Section~\ref{sec: History of Lattice Boltzmann}, instead of calculating the collision, we can model the effect of multiple collisions, which is the shift towards a local equilibrium.
Albeit not a physically sound reasoning, one can imagine a very big, frictionless billiard table.
Initially we roll all balls in the same direction but not perfectly parallel to the wall.
In the first few moments, those balls will go along in the same direction and the system will have a preferred direction and velocity, namely the direction and velocity we initiated.
After some time passed, the balls will have collided several times amongst themselves and they will have a plethora of different speeds and directions.
There's now no preferred direction and the distribution of speeds won't change anymore; the system is in an equilibrium.

Changing back to the continuous view, this equilibrium is known as the Maxwellian Distribution, which reads in two dimensions, c.f.~\cite{succi2001lattice}:
\begin{equation}
  \label{eq:maxwell distribution raw}
  f_{m, \rho, u, T}^{\text{maxwell}}(\xi) = \rho \frac{m}{2\pi k_B T} \exp \left( - \frac{m\abs{\xi-u}^2}{2 k_B T}\right)
\end{equation}
where
\begin{center}
  \begin{tabular}{@{}ll@{}}
    \toprule
    Symbol & Quantity  \\
    \midrule
    $\xi$  & Microscopic speed  \\
    $\rho$ & Macroscopic density     \\
    $u$    & Macroscopic velocity   \\
    $T$    & Temperature   \\
    $k_B$  & Boltzmann constant \\
    $m$    & Mass of the particles   \\
    \bottomrule
  \end{tabular}
\end{center}

We could now define $f_{ij}^{eq}$ due to some unspecified discretization of the velocity in~\eqref{eq:maxwell distribution raw} and write the collision of~\eqref{eq: lattice boltzmann equation} as
\begin{equation}
  \tilde{\Omega}(f_{ij}) = \omega \left(f_{ij}^{eq} - f_{ij}\right)
\end{equation}
and thus
\begin{equation}
  \label{eq: post collision discrete}
  f_{ij}^* = f_{ij} + \omega \left(f_{ij}^{eq} - f_{ij}\right) = (1-\omega)f_{ij} + f_{ij}^{eq}.
\end{equation}
Here we can perfectly see the linear interpolation for the distributions.
The most popular method of discretizing~\eqref{eq:maxwell distribution raw} is due to a Taylor series expansion, yielding
\begin{equation}
  f_{ij}^{eq}(\rho,\vec{u}) = W_{ij}\rho
  \left[
    1
    + 3\frac{\vec{c}_{ij} \cdot \vec{u}}{c^2}
    + \frac{9}{2}\frac{{(\vec{c}_{ij} \cdot \vec{u})}^2}{c^4}
    - \frac{3}{2}\frac{\vec{u} \cdot \vec{u}^2}{c^2}
  \right],
\end{equation}
where $W_{ij}$ is a weighting factor.
This yields the most simple, but most famous BGK-LBM, also called \gls{srt}, as it uses only one parameter $\omega$ to relax the~\glspl{pdf}.

When analyzing this method,~\cite[Section 5.2.3]{wolf2000lattice}, we see, that the macroscopic viscosity $\nu$ is actually calculated by
\begin{equation}
  \nu = \frac{1}{3}\left(\frac{1}{\omega} - \frac{1}{2}\right).
\end{equation}
Hence, to get to low viscosities, we actually have to overrelax the \glspl{pdf} in~\eqref{eq: post collision discrete}.
The root of this seeming inconsistency is elucidated in~\cite[Section 4]{karlin2006elements}, and stems from our discretization of~\eqref{eq: Boltzmann transport equation}.


%A bit about how fast we go to local equilibrium, coupling of independent processes
