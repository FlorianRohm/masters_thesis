% !TEX root = ../thesis.tex

As cumulants of low orders are easily derived from central moments, we seek a fast way to compute the latter and afterwards get the distribution functions back for streaming.
Our starting point is equation~\eqref{eq: central moment definition}.
The straight forward calculation will take $9$ summations for each central moment we need, for the full set of $9$ this will yield $81$ summations.
Keep in mind, that this has to be done in every cell and every timestep, so any savings we can get, will pay off greatly.

Therefore, we split~\eqref{eq: central moment definition} to get
\begin{equation}
  \begin{aligned}
    c_{\alpha\beta}
    & = \sum_{ij} {\left(i-\frac{u_1}{c}\right)}^\alpha {\left(j-\frac{u_2}{c}\right)}^\beta f_{ij} \\
    & = \sum_{i} {\left(i-\frac{u_1}{c}\right)}^\alpha \underbrace{\sum_{j} {\left(j-\frac{u_2}{c}\right)}^\beta f_{ij}}_{\defines c_{i|\beta}} \\
    & = \sum_{i} {\left(i-\frac{u_1}{c}\right)}^\alpha c_{i|\beta}.
  \end{aligned}
\end{equation}
Thus, by first computing the $c_{i|\beta}$ and then the $c_{\alpha\beta}$ we need $3$ (inner sum) times $9$ (number of  $c_{i|\beta}$ respectively $c_{\alpha\beta}$) calculations.
This yields a total of $54$ calculations, with the original count being $81$.
This is a reduction of one third; not bad for such a small calculation.
By chaining this technique in 3D, we can even come down to only one third of the costs, i.e.\ $243$ instead of $729$ summations.

We can further expand the $c_{\alpha\beta}$ and $c_{i|\beta}$ terms to get
\begin{equation}
  \begin{aligned}
    \label{eq: fast forward c alpha beta expanded}
    c_{\alpha\beta} & = \sum_{i} {\left(i-\frac{u_1}{c}\right)}^\alpha c_{i|\beta} \\
    & = {\left(-1-\frac{u_1}{c}\right)}^\alpha c_{-1|\beta} + {\left(-\frac{u_1}{c}\right)}^\alpha c_{0|\beta} + {\left(1-\frac{u_1}{c}\right)}^\alpha c_{1|\beta}
  \end{aligned}
\end{equation}
and
\begin{equation}
  \begin{aligned}
    \label{eq: fast forward c i pipe beta expanded}
    c_{i|\beta} & = \sum_{j} {\left(j-\frac{u_2}{c}\right)}^\beta f_{ij} \\
    & = {\left(-1-\frac{u_2}{c}\right)}^\beta f_{i-1} + {\left(-\frac{u_2}{c}\right)}^\beta f_{i0} + {\left(1-\frac{u_2}{c}\right)}^\beta f_{i1}.
  \end{aligned}
\end{equation}
Thus we can formulate~\eqref{eq: fast forward c alpha beta expanded} as a vector-matrix product
\begin{equation}
  \label{eq: fast forward c alpha beta matrix}
  \begin{pmatrix}
    c_{0\beta} \\
    c_{1\beta} \\
    c_{2\beta}
  \end{pmatrix}
  =
  \begin{pmatrix}
    1 & 1 & 1 \\
    -1-\frac{u_1}{c} & - \frac{u_1}{c} &   1-\frac{u_1}{c} \\
    {\left(-1-\frac{u_1}{c}\right)}^2 & {\left(\frac{u_1}{c}\right)}^2 &  {\left(1-\frac{u_1}{c}\right)}^2
  \end{pmatrix}
  \begin{pmatrix}
    c_{-1|\beta} \\
    c_{0|\beta} \\
    c_{1|\beta}
  \end{pmatrix}.
\end{equation}
In complete analogy,~\eqref{eq: fast forward c i pipe beta expanded} is written as
\begin{equation}
  \label{eq:fast forward c i pipe beta matrix}
  \begin{pmatrix}
    c_{i|0} \\
    c_{i|1} \\
    c_{i|2}
  \end{pmatrix}
  =
  \begin{pmatrix}
    1 & 1 & 1 \\
    -1-\frac{u_2}{c} & - \frac{u_2}{c} &   1-\frac{u_2}{c} \\
    {\left(-1-\frac{u_2}{c}\right)}^2 & {\left(\frac{u_2}{c}\right)}^2 &  {\left(1-\frac{u_2}{c}\right)}^2
  \end{pmatrix}
  \begin{pmatrix}
    f_{i-1} \\
    f_{i0} \\
    f_{i1}
  \end{pmatrix}.
\end{equation}

Of course, we also need to get the backward direction.
Thanks to the preliminary work of writing the transformation as two vector-matrix transformations, this is as easy as inverting those matrices.
In our case,~\eqref{eq: fast forward c alpha beta matrix} becomes\footnote{c.f.\ mathematica/backward\_transformation.nb}
\begin{equation}
  \begin{pmatrix}
    c_{-1|\beta} \\
    c_{0|\beta} \\
    c_{1|\beta}
  \end{pmatrix}
  =
  \begin{pmatrix}[1.5]
    -\frac{1}{2}(\frac{u_1}{c} - {(\frac{u_1}{c})}^2) &
    -\frac{1}{2}(1 - 2\frac{u_1}{c}) &
    \frac{1}{2} \\
    1 - {(\frac{u_1}{c})}^2  &
    -2\frac{u_1}{c}  &
    -1 \\
    \frac{1}{2}(\frac{u_1}{c} + {(\frac{u_1}{c})}^2)  &
    \frac{1}{2}(1 + 2\frac{u_1}{c})  &
    \frac{1}{2}
  \end{pmatrix}
  \begin{pmatrix}
    c_{0\beta} \\
    c_{1\beta} \\
    c_{2\beta}
  \end{pmatrix}
\end{equation}
and~\eqref{eq:fast forward c i pipe beta matrix} analogously
\begin{equation}
  \begin{pmatrix}
    f_{i-1} \\
    f_{i0} \\
    f_{i1}
  \end{pmatrix}
  =
  \begin{pmatrix}[1.5]
    -\frac{1}{2}(\frac{u_2}{c} - {(\frac{u_2}{c})}^2) &
    -\frac{1}{2}(1 - 2\frac{u_2}{c}) &
    \frac{1}{2} \\
    1 - {(\frac{u_2}{c})}^2  &
    -2\frac{u_2}{c}  &
    -1 \\
    \frac{1}{2}(\frac{u_2}{c} + {(\frac{u_2}{c})}^2)  &
    \frac{1}{2}(1 + 2\frac{u_2}{c})  &
    \frac{1}{2}
  \end{pmatrix}
  \begin{pmatrix}
      c_{i|0} \\
      c_{i|1} \\
      c_{i|2}
    \end{pmatrix}.
\end{equation}
