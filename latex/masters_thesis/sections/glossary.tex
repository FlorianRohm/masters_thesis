\newacronym{lbm}{LBM}{Lattice-Boltzmann Method}
\newacronym{cfd}{CFD}{computational fluid dynamics}
\newacronym[plural=pdfs,firstplural=particle distribution functions (pdfs)]{pdf}{pdf}{particle distribution function}
\newacronym{srt}{SRT}{Single Relaxation Time}
\newacronym{mrt}{MRT}{Multiple Relaxation Time}

\longnewglossaryentry{kn}
{
name={Knudsen number, $Kn$}
}
{Relation between a characteristic length scale $L$ and mean free path $\lambda$ a particle can travel before colliding with another particle,
\begin{equation*}
  Kn=\frac{\lambda}{L}.
\end{equation*}}

\longnewglossaryentry{macVar}
{
name={Macroscopic variable}
}
{A value/variable which makes only sense in a averaged view, e.g.\ macroscopic velocity means the averaged velocity of the particles, observed in the flow field
}

\longnewglossaryentry{n}
{
name={Natural numbers $\N$}
}
{Natural numbers are defined in this thesis as non-negative integers, hence include zero.
}

\longnewglossaryentry{micVar}
{
name={Microscopic variable}
}
{
A value which is present at microscopic scales, like the velocities of individual particles.
Their significance drops when viewing large numbers of particles and averaging, c.f.\ macroscopic variables
}

\longnewglossaryentry{galInv}
{
name={Galilean invariance}
}{States the independence of the flow field on the frame of reference}

\longnewglossaryentry{re}
{
name={Reynolds number, $Re$}
}
{Relates a reference velocity $u_0$, a characteristic length scale $L$ and the viscosity $\nu$ and is used to nondimensionalize the Navier-Stokes Equations
\begin{equation*}
  \label{eq: definition of reynolds number}
  Re=\frac{u_0 L}{\nu}.
\end{equation*}}

\longnewglossaryentry{ns}
{
name={Incompressible Navier-Stokes Equations}
}
{Equations for mass and momentum conservation in incompressible fluid flows, fully written out in Appendix~\ref{appendix: Navier Stokes Equations}
}

\longnewglossaryentry{frame of reference}
{
name={Frame of reference}
}
{
The coordinate system of the observer of a system.
The fluid physics should not change when our view of the system is moving.
A common visualization is to think of a surfer in the ocean, idealized to move exactly with the flow and a helicopter hovering over one fixed position.
Both see the same flow, but for the surfer, the fluid at exactly his position has no `relative' velocity and the coast is moving towards him. This is called Lagrangian point of view.
On the other side, the coast is stationary for the helicopter pilot but the surfer moves with a velocity greater zero. One describes this as the Eulerian view.
}
