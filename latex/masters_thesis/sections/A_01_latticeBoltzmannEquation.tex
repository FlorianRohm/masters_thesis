% !TEX root = ../thesis.tex
Our first appendix will cope with the discretization of the Boltzmann equation~\eqref{eq: Boltzmann transport equation}.
As we will heavily use the explicit and implicit Euler scheme, we provide a small reminder.
For some functions $g$ and $h$ with
\begin{equation}
  \label{eq: app_lbm_1}
  \partial_x f(x) = g(x),
\end{equation}
and two points $x_k$ and $x_k + \Delta x$,
\begin{equation}
  f(x_k+\Delta x) - f(x_k) = \int_{x_k}^{x_k+\Delta x} g(x) dx
\end{equation}
holds by integrating~\eqref{eq: app_lbm_1}.
We can now approximate the integral by evaluating it at a specific point and dividing the other parts with the length of the integration interval.
The explicit form chooses the starting point of the integral, whereas the implicit one takes the endpoint, yielding
\begin{equation}
  \frac{f(x_k+\Delta x) - f(x_k)}{\Delta x} =  g(x_k)
\end{equation}
for the explicit form and
\begin{equation}
  \frac{f(x_k+\Delta x) - f(x_k)}{\Delta x} =  g(x_k + \Delta x)
\end{equation}
for the implicit one.

To start deriving, we again state~\eqref{eq: Boltzmann transport equation} for easier oversight.
\begin{equation}
  \label{eq: app_lbm_2}
  \partial_t f(\vec{v},\vec{x},t) + \vec{v} \cdot \nabla_{\vec{x}} f(\vec{v},\vec{x},t) = \Omega\left(f(\vec{v},\vec{x},t)\right)
\end{equation}
First of all, we switch to the discrete velocity space and explicitly calculate the dot product, leaving us with
\begin{equation}
  \label{eq: app_lbm_3}
  \partial_t f_{ij}(\vec{x},t) + \vec{c}_{ij} \cdot \nabla_{\vec{x}} f_{ij}(\vec{x},t) = \Omega\left(f_{ij}(\vec{x},t)\right).
\end{equation}
Now, we discretize the time derivative, using the implicit scheme on the remaining summands on the left hand side and staying explicit on the right hand side.
\begin{equation}
  \label{eq: app_lbm_4}
  \frac{f_{ij}(\vec{x}, t+\Delta t) - f_{ij}(\vec{x}, t)}{\Delta t}
  +
  \underbrace {\vec{c}_{ij} \cdot \nabla_{\vec{x}} f_{ij}(\vec{x},t+\Delta t)
  }_{\text{implicit}}
   = \underbrace{\Omega\left(f_{ij}(\vec{x},t)\right)}_{\text{explicit}}.
\end{equation}
For the next step, we have to see, that the term $\vec{c}_{ij} \cdot \nabla_{\vec{x}}$ actually gives us the directional derivative in the direction $\begin{pmatrix}i\\j\end{pmatrix}$, scaled with $\norm{\vec{c}_{ij}}$.
Thus we can discretize~\eqref{eq: app_lbm_4} to
\begin{equation}
  \label{eq: app_lbm_5}
  \begin{aligned}
    \underbrace{\frac{f_{ij}(\vec{x}, t+\Delta t) - f_{ij}(\vec{x}, t)}{\Delta t}}_{\text{explicit}}
    & + \norm{\vec{c}_{ij}}\frac{f_{ij}\left(\vec{x}+\begin{pmatrix}i\Delta x\\j\Delta y\end{pmatrix}, t+\Delta t\right) - f_{ij}(\vec{x}, t+\Delta t)}{\sqrt{i^2\Delta x^2 + j^2\Delta y^2}}
    \\&
    = \underbrace{\Omega\left(f_{ij}(\vec{x},t)\right)}_{\text{explicit}}.
  \end{aligned}
\end{equation}
With the relation between the stepsizes,~\eqref{eq: relation between stepsizes}, we can reformulate~\eqref{eq: app_lbm_5} to
\begin{equation}
  \label{eq: app_lbm_6}
  \begin{aligned}
    \frac{f_{ij}(x,y, t+\Delta t) - f_{ij}(x,y, t)}{\Delta t}
    & + \frac{f_{ij}\left(x + ci\Delta t, y+ cj\Delta t, t+\Delta t\right) - f_{ij}(x,y, t+\Delta t)}{\Delta t}
    \\&
    = \Omega\left(f_{ij}(x,y,t)\right).
  \end{aligned} 
\end{equation}
and thus
\begin{equation}
  \label{eq: app_lbm_7}
  \begin{aligned}
    f_{ij}\left(x + ci\Delta t, y+ cj\Delta t, t+\Delta t\right)
    = \Delta t\ \Omega\left(f_{ij}(x,y,t)\right) + f_{ij}(x,y, t).
  \end{aligned}
\end{equation}
When inserting our modified collision, which we assume to be scaled with $\Delta t$, this is exactly equation~\eqref{eq: lattice boltzmann equation}.
