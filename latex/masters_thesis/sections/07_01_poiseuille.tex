% !TEX root = ../thesis.tex
The first problem will be a so called Poiseuille flow, which is one of the simplest nontrivial cases in incompressible fluid dynamics.
It's a stationary flow in a pipe, driven by a pressure gradient through it.
Due to the wall friction, this leads to a parabolic flow, which we will use as inflow condition.

\begin{figure}
  \centering
  % !TEX root = ../../thesis.tex

\begin{tikzpicture}
\fill[pattern=north east lines] (0,0) rectangle (10,-0.2);
\fill[pattern=north east lines] (0,3) rectangle (10,3.2);
\draw (0,0) .. controls (3,1) and (3,2) .. (0,3);
\draw (0,0) -- (10,0);
\draw (0,3) -- (10,3);
\draw (0,0) -- (0,3);

\draw[->] (0,1.5) -- (2.25,1.5);
\draw[->] (0,1) -- (2,1);
\draw[->] (0,2) -- (2,2);
\draw[->] (0,2.5) -- (1.22,2.5);
\draw[->] (0,0.5) -- (1.22,0.5);

\node (u) at (1.1,1.7) {$\vec{u}$};


\draw[->] (-0.5,-0.7) -- (-0.5,4);
\draw[->] (-0.5,-0.7) -- (10.5,-0.7);

\node (y) at (-0.5,4.2) {$y$};
\node (x) at (10.7,-0.7) {$x$};

\draw[-] (-0.55,1.5) -- (-0.45,1.5);
\draw[-] (0,-0.65) -- (0,-0.75);

\node (nully) at (0,-0.95) {$0$};
\node (nullx) at (-0.65,1.5) {$0$};
\end{tikzpicture}

  \caption{Poiseuille flow}
\label{fig: poiseuille}
\end{figure}

Figure~\ref{fig: poiseuille} shows a schematic of the setup.
The test now examines the influence of the length of the channel\footnote{To be exact, the following lengths were simulated:\par
 $100,\ 95,\ 90,\ 85,\ 80,\ 75,\ 70,\ 65,\ 60,\ 55,\ 50,\ 45,\ 40,\ 35,\ 30,\ 25,\ 20,\ 15,\ 11,\ 9,\ 7,\ 5,\ 3$
} on the velocity distribution.
The height was always chose to be $20$ cells.

As the \gls{lbm} is a mildly compressible scheme, the comparison with the incompressible, analytic results would not be very expressive.
Indeed, if compared to the analytical result, the velocities were decreasing when choosing longer channels.
Instead, the difference between the velocities in the middle of the channel length is taken and summed up over the channel height.

\begin{figure}
  \centering
  \includegraphics[width=0.8\linewidth]{../figures/poiseuille.pdf} % chktex 11
  \caption{Difference between Cumulants and SRT in percent of the initial inflow velocity in a Poiseuille Flow plotted agains the channel length}
\label{fig: poiseuille result}
\end{figure}

The result can be seen in Figure~\eqref{fig: poiseuille result}.
Two things are remarkable here.
First of all, the difference is below one percent of the inflow velocity which is a first sign, that this method is not completely off the track.

Secondly, we can look at the nature of the difference.
Here, we see that the cumulant method was generally a bit more slowed down than \gls{srt}, which indicates, that the cumulant method is a bit more compressible.
But as the difference is below one percent, let's move on to some more challenging tasks.
