% !TEX root = ../thesis.tex
As $\epsilon$ is arbitrary and the terms of the expansion are not allowed to depend on $\epsilon$, equality has to hold in every order of $\epsilon$.
Therefore, we will take a closer look on the equations arising, by extracting each order of $\epsilon$ from equation~\eqref{eq: final nondimensionalised expansion series}.

\subsubsection{Zero'th order in \texorpdfstring{$\epsilon$}{epsilon}}
\label{subs: Zeroth order in epsilon}

To achieve $\epsilon^0$, we have to choose $\tau=p=p'=m=n=0$, yielding
\begin{equation}
  \label{eq: zeroth order in epsilon}
  m_{\alpha\beta}^{(0)} = m_{\alpha\beta}^{*(0)}.
\end{equation}
The important part is, what does this actually mean?
When voiced, equation~\eqref{eq: zeroth order in epsilon} states, that all pre-collision zeroth order moments are equal to their post-collision counterparts.
This means, they have to be in equilibrium for the collision to have no effect on them.
Thus,
\begin{equation}
  \label{eq: zeroth order invariant}
  m_{\alpha\beta}^{(0)} = m_{\alpha\beta}^{*(0)} = m_{\alpha\beta}^{eq(0)}.
\end{equation}

One important thing to note is, that zeroth order terms may not depend on the expansion parameter $\epsilon$, which is arbitrary.
Additionally, we demanded that the expanded moments do not depend on $\epsilon$.
The only class of functions, which satisfies these conditions are constants in space.
Hence, we know
\begin{equation}
  \label{eq: derivative of zeroth order}
  \begin{aligned}
    \partial_x m_{\alpha\beta}^{(0)} &= 0 \\
    \partial_y m_{\alpha\beta}^{(0)} &= 0
  \end{aligned}
\end{equation}

\subsubsection{First order in \texorpdfstring{$\epsilon$}{epsilon}}
\label{subs: First order in epsilon}

For the first order terms, one has to choose $p=1$ and $\iota=0$ on the left side and one of $m$, $n$ and $p'$ equal to one.
Incorporating those three equations, we get
\begin{equation}
  \label{eq: first order in epsilon}
  \begin{aligned}
  m_{\alpha\beta}^{(1)}
  & = m_{\alpha\beta}^{*(1)}
  - \partial_x m_{(\alpha+1)\beta}^{*(0)} - \partial_y m_{\alpha(\beta+1)}^{*(0)} \\
  & = m_{\alpha\beta}^{*(1)},
  \end{aligned}
\end{equation}
using~\eqref{eq: derivative of zeroth order}.
We see again the equivalence of pre- and post-collision moments in analogy to~\eqref{eq: zeroth order in epsilon} and thus with the same reasoning that led to equation~\eqref{eq: zeroth order invariant}
\begin{equation}
  \label{eq: first order invariant}
  m_{\alpha\beta}^{(1)} = m_{\alpha\beta}^{*(1)} = m_{\alpha\beta}^{eq(1)}.
\end{equation}
In contrast to the zeroth orders, the first orders don't need to be constant in space and we can not set their derivatives to zero.

\subsubsection{Second order in \texorpdfstring{$\epsilon$}{epsilon}}
\label{subs: Second order in epsilon}

Getting second order terms in~\eqref{eq: final nondimensionalised expansion series} requires on the left hand side $\tau=1$ and $p=0$ or $\tau=0$ and $p=2$.
On the right hand side, we have six possibilities to finally get $\epsilon^2$, depicted in Table~\ref{table: second order epsilon}.
\begin{table}[h]
  \centering
  \begin{tabular} {r || c | *{2}{c} | *{2}{c} | c}
    m  & 0 & 1 & 2 & 0 & 0 & 1 \\
    n  & 0 & 0 & 0 & 1 & 2 & 1 \\
    p' & 2 & 1 & 0 & 1 & 0 & 0
  \end{tabular}
  \caption{All possibilities to achieve $\epsilon^2$ in equation~\eqref{eq: final nondimensionalised expansion series}}
\label{table: second order epsilon}
\end{table}
Resulting in
\begin{equation}
  \label{eq: second order in epsilon}
  \begin{aligned}
    \partial_t m_{\alpha\beta}^{(0)} + m_{\alpha\beta}^{(2)}
    & =  m_{\alpha\beta}^{*(2)} \\
    &\quad - \partial_x m_{(\alpha+1)\beta}^{*(1)} + \partial_{xx} m_{(\alpha+2)\beta}^{*(0)}/2 \\
    &\quad - \partial_y m_{\alpha(\beta+1)}^{*(1)} + \partial_{yy} m_{\alpha(\beta+2)}^{*(0)}/2 \\
    &\quad + \partial_{xy} m_{(\alpha+1)(\beta+1)}^{*(0)}\\
    & =  m_{\alpha\beta}^{*(2)} - \partial_x m_{(\alpha+1)\beta}^{*(1)} - \partial_y m_{\alpha(\beta+1)}^{*(1)},
  \end{aligned}
\end{equation}
again using equation~\eqref{eq: derivative of zeroth order}.
After~\eqref{eq: first order in epsilon} and~\eqref{eq: second order in epsilon}, this is the first equation to not result in collision invariants, but features different orders of moments and all partial derivatives.
This will later mark the first equation to examine for the Navier-Stokes derivation.

\subsubsection{Third order in \texorpdfstring{$\epsilon$}{epsilon}}
\label{subs: Third order in epsilon}
Now, things are getting complicated.
Third order in $\epsilon$ still means two possibilities on the left hand side, namely either $\begin{pmatrix}\tau \\ p\end{pmatrix} = \begin{pmatrix} 1 \\ 1 \end{pmatrix}$ or $\begin{pmatrix}\tau \\ p\end{pmatrix} = \begin{pmatrix} 0 \\ 3 \end{pmatrix}$.
The right hand side on the other hand, has ten possibilities, displayed in Table~\ref{table: third order epsilon}.
\begin{table}[h]
  \centering
  \begin{tabular} {r || c | *{3}{c} | *{3}{c} | *{3}{c} }
    m  & 0 & 1 & 2 & 3 & 0 & 0 & 0 & 1 & 2 & 1 \\
    n  & 0 & 0 & 0 & 0 & 1 & 2 & 3 & 1 & 1 & 2 \\
    p' & 3 & 2 & 1 & 0 & 2 & 1 & 0 & 1 & 0 & 0
  \end{tabular}
  \caption{All possibilities to achieve $\epsilon^3$ in equation~\eqref{eq: final nondimensionalised expansion series}}
\label{table: third order epsilon}
\end{table}
Fortunately, some of them are zeroth order terms with derivatives and hence they vanish, giving us
\begin{equation}
  \label{eq: third order in epsilon}
  \begin{aligned}
    \frac{\partial}{\partial t} m_{\alpha\beta}^{(1)} + m_{\alpha\beta}^{(3)}
    & =  m_{\alpha\beta}^{*(3)} \\
    &\quad - \partial_x m_{(\alpha+1)\beta}^{*(2)} + \partial_{xx} m_{(\alpha+2)\beta}^{*(1)}/2 - \partial_{xxx} m_{(\alpha+3)\beta}^{*(0)}/6 \\
    &\quad - \partial_y m_{\alpha(\beta+1)}^{*(2)} + \partial_{yy} m_{\alpha(\beta+2)}^{*(1)}/2 - \partial_{yyy} m_{\alpha(\beta+3)}^{*(0)}/6 \\
    &\quad + \partial_{xy} m_{(\alpha+1)(\beta+1)}^{*(1)} \\
    &\quad - \partial_{xxy} m_{(\alpha+2)(\beta+1)}^{*(0)}/2 - \partial_{xyy} m_{(\alpha+1)(\beta+2)}^{*(0)}/2 \\
    & =  m_{\alpha\beta}^{*(3)} \\
    &\quad - \partial_x m_{(\alpha+1)\beta}^{*(2)} - \partial_y m_{\alpha(\beta+1)}^{*(2)}  \\
    &\quad  + \partial_{xx} m_{(\alpha+2)\beta}^{*(1)}/2 + \partial_{xy} m_{(\alpha+1)(\beta+1)}^{*(1)} + \partial_{yy} m_{\alpha(\beta+2)}^{*(1)}/2.
  \end{aligned}
\end{equation}
Those equations will mark the starting point for the upcoming analysis.
Before, doing so, we have to do some more tedious work, expanding the normalized cumulants and equilibrium moments.
