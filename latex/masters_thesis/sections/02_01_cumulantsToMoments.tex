% !TEX root = ../thesis.tex

The link to moments can now be established with the chain rule for differentiaton.
From~\eqref{eq: alternative representation of moments} and~\eqref{eq: definition of cumulants} follows for $\alpha=\beta=0$:
\begin{equation}
  \label{eq: definition of kappa_00}
  \kappa_{00} = \ln(F(0,0)) = \ln(m_{00}) = \ln(\rho).
\end{equation}
For brevity, the following notations are introduced
\begin{equation}
  \label{eq: abbreviations for deriving cumulants from moments}
  \begin{aligned}
    F & \defined F(\Xi_1, \Xi_2) \\
    \vec{\Xi} &\defined \begin{pmatrix}\Xi_1 \\ \Xi_2  \end{pmatrix} \\
    \partial_i &\defined \frac{\partial}{\partial \Xi_i},\quad i=1,2.
  \end{aligned}
\end{equation}
To get the first cumulant, all we have to do is use~\eqref{eq: definition of cumulants} with $\alpha=1$, $\beta=0$ and secondly~\eqref{eq: alternative representation of moments}
for recognizing $m_{00} = \left. F \right|_{\vec{\Xi} = 0} $ and $m_{10} = \left. \partial_1 F \right|_{\vec{\Xi} = 0}$:
\begin{equation}
  \label{eq: definition of kappa_10}
  \begin{aligned}
    \kappa_{10} & = \left.\partial_1 \ln(F) \right|_{\vec{\Xi} = 0} \\
    & = \left. \frac{1}{F} \partial_1 F \right|_{\vec{\Xi} = 0} \\
    & = \frac{m_{10}}{m_{00}}
  \end{aligned}
\end{equation}
The cumulant $\kappa_{01}$ follows from symmetry.
To not inflate the text, all other derivations are calculated in Appendix~\ref{appendix: All moments from cumulants}.
After all calculations, we are left with
\begin{align}
  \kappa_{00} & = \ln(m_{00}) \label{eq: definition of kappa_00}\\
  \kappa_{10} & = \frac{m_{10}}{m_{00}} \label{eq: definition of kappa_10}\\
  \kappa_{01} & = \frac{m_{01}}{m_{00}} \label{eq: definition of kappa_01}\\
  \kappa_{11} & = \frac{m_{11}}{m_{00}} - \frac{m_{10}m_{01}}{m_{00}^2} \label{eq: definition of kappa_11}\\
  \kappa_{20} & = \frac{m_{20}}{m_{00}} - \frac{m_{10}^2}{m_{00}^2} \label{eq: definition of kappa_20}\\
  \kappa_{02} & = \frac{m_{02}}{m_{00}} - \frac{m_{01}^2}{m_{00}^2} \label{eq: definition of kappa_02}\\
  \kappa_{21} & = \frac{m_{21}}{m_{00}} - \frac{m_{20}m_{01}}{m_{00}^2}
       - 2\frac{m_{10}m_{11}}{m_{00}^2} + 2\frac{m_{10}^2 m_{01}}{m_{00}^3} \label{eq: definition of kappa_21}\\
  \kappa_{12} & = \frac{m_{12}}{m_{00}} - \frac{m_{10}m_{02}}{m_{00}^2}
       - 2\frac{m_{11}m_{01}}{m_{00}^2} + 2\frac{m_{10} m_{01}^2}{m_{00}^3} \label{eq: definition of kappa_12}\\
  \kappa_{22} & = \frac{1}{m_{00}} m_{22} \label{eq: definition of kappa_22}\\
    &\quad
    - \frac{1}{m_{00}^2}
    \left(
       2 m_{10}m_{12}  + 2m_{21}m_{01} + 2 m_{11}^2 + m_{20}m_{02}
    \right) \\
    &\quad
    + \frac{2}{m_{00}^3}
      \left(
        m_{10}^2 m_{02} + 4 m_{10}m_{11}m_{01} + m_{20}m_{01}^2
      \right)\\
    &\quad
    - \frac{6}{m_{00}^4} m_{10}^2 m_{01}^2.
\end{align}
Several things can be noticed here.
Firstly, the zeroth cumulant $\kappa_{00}$ is somehow the odd one out in featuring a logarithm in it's calculation.
At first sight, this may spell doom for our computation costs.
Fortunately, we won't need to calculate this cumulant in the actual algorithm later on.

Further, we see the the price we have to pay for not working with probability distributions: Everything has to be normed by the density $m_{00}$.
We can remedy part of this by passing over to normalized cumulants $K_{\alpha\beta}$ defined as
\begin{equation}
  \label{eq: definition normalized cumulants}
  K_{\alpha\beta} \defined m_{00}\kappa_{\alpha\beta}.
\end{equation}
Hence, we get
\begin{align}
  K_{00} & = \ln(m_{00})m_{00} \label{eq: K 00 from moments}\\
  K_{10} & = m_{10} \label{eq: K 10 from moments}\\
  K_{01} & = m_{01} \label{eq: K 01 from moments}\\
  K_{11} & = m_{11} - \frac{m_{10}m_{01}}{m_{00}} \label{eq: K 11 from moments} \\
  K_{20} & = m_{20} - \frac{m_{10}^2}{m_{00}} \label{eq: K 20 from moments}\\
  K_{02} & = m_{02} - \frac{m_{01}^2}{m_{00}} \label{eq: K 02 from moments}\\
  K_{21} & = m_{21} - \frac{m_{20}m_{01}}{m_{00}}
       - 2\frac{m_{10}m_{11}}{m_{00}} + 2\frac{m_{10}^2 m_{01}}{m_{00}^2} \label{eq: K 21 from moments}\\
  K_{12} & = m_{12} - \frac{m_{10}m_{02}}{m_{00}}
       - 2\frac{m_{11}m_{01}}{m_{00}} + 2\frac{m_{10} m_{01}^2}{m_{00}^2} \label{eq: K 12 from moments}\\
  K_{22} & = m_{22} \nonumber \\
       & \quad - 2 \frac{m_{10}m_{12}}{m_{00}} - 2\frac{m_{21}m_{01}}{m_{00}}
        - 2 \frac{m_{11}^2}{m_{00}} - \frac{m_{20}m_{02}}{m_{00}}\nonumber \\
       & \quad + 2 \frac{m_{10}^2 m_{02}}{m_{00}^2} + 8 \frac{m_{10}m_{11}m_{01}}{m_{00}^2}
        + 2 \frac{m_{20}m_{01}^2}{m_{00}^2} \nonumber \\
       & \quad - 6 \frac{m_{10}^2 m_{01}^2}{m_{00}^3}\label{eq: K 22 from moments}.
\end{align}
This definitely is more pleasant to look at and saves us a few computations.
Down the line, we will also formulate the collision in terms of normalized cumulants.

Further, the structure behind cumulants becomes more clear, being equal to their corresponding order moments plus a nonlinear, polynomial to be exact, correction term.
One can think of this term as the correction we have to do when decoupling~\eqref{eq: moment generating function for independent variables}
to~\eqref{eq: cumulant generating function for independent variables}.
