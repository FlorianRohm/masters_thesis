% !TEX root = ../thesis.tex

Like in the derivation of the cumulants, our equilibrium cumulants need the Taylor series of the logarithm of the Laplace transformed equilibrium distribution.
With the speed of sound $c_s$ defined as
\begin{equation}
  c_s \defined \sqrt{\frac{k_B T}{m}}
\end{equation}
we can take the logarithm of~\eqref{eq: laplace of maxwell} to get
\begin{equation}
  \label{eq: log of laplace of maxwell}
  \begin{aligned}
    \ln(F^{eq}(\Xi))
      & = \ln(\rho) - \Xi_1 u_1 - \Xi_2 u_2 + \frac{1}{4C}\left(\Xi_1^2 + \Xi_2^2 \right) \\
      & = \ln(\rho) - \Xi_1 u_1 - \Xi_2 u_2 + \frac{k_B T}{2m}\left(\Xi_1^2 + \Xi_2^2 \right) \\
      & = \ln(\rho) - \Xi_1 u_1 - \Xi_2 u_2 + \frac{c_s^2}{2}\left(\Xi_1^2 + \Xi_2^2 \right).
  \end{aligned}
\end{equation}
As~\eqref{eq: log of laplace of maxwell} is a finite polynomial, its Taylor series is finite and equal to the function itself.
Hence, we can directly read of the cumulants, following~\eqref{eq: definition of cumulants}, to be
\begin{equation}
  \label{eq: equilibrium cumulants}
  \begin{aligned}
    \kappa_{00}^{eq} & = \ln(\rho) \\
    \kappa_{10}^{eq} & = - {(-c)}^{-1} u_1 \\
    \kappa_{01}^{eq} & = - {(-c)}^{-1} u_2 \\
    \kappa_{11}^{eq} & = 0 \\
    \kappa_{20}^{eq} & = {(-c)}^{-2} c_s^2  \\
    \kappa_{02}^{eq} & = {(-c)}^{-2} c_s^2  \\
    \kappa_{21}^{eq} & = 0 \\
    \kappa_{12}^{eq} & = 0 \\
    \kappa_{22}^{eq} & = 0.
  \end{aligned}
\end{equation}
Additionally, we can simplify the $\kappa_{02}$ and $\kappa_{20}$.
As their equilibria are the same, their difference has zero equilibrium and their sum double the equilibrium, i.e.
\begin{equation}
  \begin{aligned}
    \kappa_{20}^{eq} - \kappa_{02}^{eq} & = 0  \\
    \kappa_{20}^{eq} + \kappa_{02}^{eq} & = 2 {(-c)}^{-2} c_s^2.
  \end{aligned}
\end{equation}
%
This leads to the post collision cumulants $\kappa_{\alpha\beta}^*$
\begin{equation}
  \label{eq: post equilibrium cumulants}
  \begin{aligned}
    \kappa_{00}^{*} & = \kappa_{00} + \omega_1 \left( \ln(\rho) - \kappa_{00} \right) \\
    \kappa_{10}^{*} & = \kappa_{10} + \omega_2 \left( - {(-c)}^{-1} u_1 - \kappa_{10} \right) \\
    \kappa_{01}^{*} & = \kappa_{01} + \omega_3 \left( - {(-c)}^{-1} u_2 - \kappa_{01} \right) \\
    \kappa_{11}^{*} & = \kappa_{11} + \omega_4 \left( - \kappa_{11} \right) \\
    \kappa_{20}^{*} - \kappa_{02}^{*}
      & = \kappa_{20} - \kappa_{02} + \omega_5 \left( - \kappa_{20} + \kappa_{02} \right) \\
    \kappa_{20}^{*} + \kappa_{02}^{*}
      & = \kappa_{20} + \kappa_{02} + \omega_6 \left( 2 {(-c)}^{-2} c_s^2 - \kappa_{20} - \kappa_{02} \right) \\
    \kappa_{21}^{*} & = \kappa_{21} + \omega_7 \left( - \kappa_{21} \right) \\
    \kappa_{12}^{*} & = \kappa_{12} + \omega_8 \left( - \kappa_{12} \right) \\
    \kappa_{22}^{*} & = \kappa_{22} + \omega_9 \left( - \kappa_{22} \right)
  \end{aligned}
\end{equation}
which can be simplified to
\begin{equation}
  \begin{aligned}
    \kappa_{00}^{*} & = \kappa_{00} \\
    \kappa_{10}^{*} & = \kappa_{10} \\
    \kappa_{01}^{*} & = \kappa_{01} \\
    \kappa_{11}^{*} & = (1-\omega_4)\kappa_{11} \\
    \kappa_{20}^{*} - \kappa_{02}^{*}
      & = (1-\omega_5) (\kappa_{20} - \kappa_{02}) \\
    \kappa_{20}^{*} + \kappa_{02}^{*}
      & = (1-\omega_6)(\kappa_{20} + \kappa_{02}) + \omega_6 \left( 2 {(-c)}^{-2} c_s^2 \right) \\
    \kappa_{21}^{*} & = (1-\omega_7)\kappa_{21} \\
    \kappa_{12}^{*} & = (1-\omega_8)\kappa_{12} \\
    \kappa_{22}^{*} & = (1-\omega_9)\kappa_{22},
  \end{aligned}
\end{equation}
as the first three cumulants are the conserved quantities, compare~\eqref{eq: K 00 from moments} and following.
The recovered post equilibrium values $\kappa_{20}^{*}$ and $\kappa_{02}^{*}$ are
\begin{equation}
  \begin{pmatrix}
    \kappa_{20}^{*} \\
    \kappa_{02}^{*}
  \end{pmatrix}
  = \frac{1}{2}
  \begin{pmatrix}
    1 & 1 \\ -1 & 1
  \end{pmatrix}
  \begin{pmatrix}
    a\\
    b
  \end{pmatrix}
\end{equation}
where
\begin{equation}
 \begin{aligned}
   a \defined \kappa_{20}^{*} - \kappa_{02}^{*}
     & = (1-\omega_5) (\kappa_{20} - \kappa_{02}) \\
   b \defined \kappa_{20}^{*} + \kappa_{02}^{*}
     & = (1-\omega_6)(\kappa_{20} + \kappa_{02}) + \omega_6 \left( 2 {(-c)}^{-2} c_s^2 \right).
 \end{aligned}
\end{equation}
To ensure rotational invariance, we have to relate some relaxation parameters, yielding
\begin{equation}
  \label{eq: collision equation system full}
  \begin{aligned}
    \kappa_{00}^{*} & = \kappa_{00} \\
    \kappa_{10}^{*} & = \kappa_{10} \\
    \kappa_{01}^{*} & = \kappa_{01} \\
    \kappa_{11}^{*} & = (1-\omega_1)\kappa_{11} \\
    \kappa_{20}^{*} - \kappa_{02}^{*}
      & = (1-\omega_1) (\kappa_{20} - \kappa_{02}) \\
    \kappa_{20}^{*} + \kappa_{02}^{*}
      & = (1-\omega_2)(\kappa_{20} + \kappa_{02}) + \omega_2 \left( 2 {(-c)}^{-2} c_s^2 \right) \\
    \kappa_{21}^{*} & = (1-\omega_3)\kappa_{21} \\
    \kappa_{12}^{*} & = (1-\omega_3)\kappa_{12} \\
    \kappa_{22}^{*} & = (1-\omega_4)\kappa_{22},
  \end{aligned}
\end{equation}
where the $\omega_i$ are not related to the ones used in~\eqref{eq: post equilibrium cumulants} and afterwards.
