% !TEX root = ../thesis.tex

Like in the derivation of the cumulants, our equilibrium cumulants need the Taylor series of the logarithm of the transformed equilibrium distribution,~\eqref{eq: laplace of maxwell}.
With $c_s$ defined as
\begin{equation*}
  c_s \defined \sqrt{\frac{k_B T}{m}},
\end{equation*}
which is the definition of the speed of sound, we can take the logarithm of~\eqref{eq: laplace of maxwell} to get
\begin{equation}
  \label{eq: log of laplace of maxwell}
  \begin{aligned}
    \ln(F^{eq}(\vec{\Xi}))
      & = \ln(\rho) + \Xi_1 u_1 + \Xi_2 u_2 + \frac{1}{4C}\left(\Xi_1^2 + \Xi_2^2 \right) \\
      & = \ln(\rho) + \Xi_1 u_1 + \Xi_2 u_2 + \frac{k_B T}{2m}\left(\Xi_1^2 + \Xi_2^2 \right) \\
      & = \ln(\rho) + \Xi_1 u_1 + \Xi_2 u_2 + \frac{c_s^2}{2}\left(\Xi_1^2 + \Xi_2^2 \right).
  \end{aligned}
\end{equation}
As~\eqref{eq: log of laplace of maxwell} is a finite polynomial, its Taylor series is finite and equal to the function itself.
Hence, we can directly read of the equilibrium cumulants, following~\eqref{eq: definition of cumulants}, to be
\begin{equation}
  \label{eq: equilibrium cumulants}
  \begin{aligned}
    \kappa_{00}^{eq} & = \ln(\rho) = \ln(m_{00}) \\
    \kappa_{10}^{eq} & = u_1 = \frac{m_{10}}{m_{00}} \\
    \kappa_{01}^{eq} & = u_2 = \frac{m_{01}}{m_{00}}\\
    \kappa_{11}^{eq} & = 0 \\
    \kappa_{20}^{eq} & = c_s^2  \\
    \kappa_{02}^{eq} & = c_s^2  \\
    \kappa_{21}^{eq} & = 0 \\
    \kappa_{12}^{eq} & = 0 \\
    \kappa_{22}^{eq} & = 0,
  \end{aligned}
\end{equation}
where we used the moment representation of the macroscopic variables~\eqref{eq: density definition} and following.

Comparing~\eqref{eq: equilibrium cumulants} to~\eqref{eq: equilibrium particle distributions}, we can identify several advantages of using cumulants.
First and foremost, the Maxwell distribution is fully incorporated and we don't introduce a truncation error\footnote{On closer analysis, this error is of order $O(u^2)$ in the \gls{srt} scheme, c.f.~\cite[page 178]{wolf2000lattice}, making it only viable at low speeds}.
Going further, we don't need to calculate any weighting $W_{ij}$ and as we incorporated the cell width $c$ in the calculation of the central moments and hence the cumulants, it is hidden from the collision calculation.


Additionally, we can simplify the $\kappa_{02}$ and $\kappa_{20}$.
As their equilibria are the same, their difference has zero equilibrium and their sum double the equilibrium, i.e.
\begin{equation}
  \label{eq: altered cumulants 20 and 02}
  \begin{aligned}
    \kappa_{20}^{eq} - \kappa_{02}^{eq} & = 0  \\
    \kappa_{20}^{eq} + \kappa_{02}^{eq} & = 2 c_s^2.
  \end{aligned}
\end{equation}
