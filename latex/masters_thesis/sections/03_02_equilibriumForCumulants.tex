% !TEX root = ../thesis.tex

Like in the derivation of the cumulants, our equilibrium cumulants need the Taylor series of the logarithm of the transformed equilibrium distribution,~\eqref{eq: laplace of maxwell}.
With $c_s$ defined as
\begin{equation}
  c_s \defined \sqrt{\frac{k_B T}{m}},
\end{equation}
which is the definition of the speed of sound, we can take the logarithm of~\eqref{eq: laplace of maxwell} to get
\begin{equation}
  \label{eq: log of laplace of maxwell}
  \begin{aligned}
    \ln(F^{eq}(\vec{\Xi}))
      & = \ln(\rho) + \Xi_1 u_1 + \Xi_2 u_2 + \frac{1}{4C}\left(\Xi_1^2 + \Xi_2^2 \right) \\
      & = \ln(\rho) + \Xi_1 u_1 + \Xi_2 u_2 + \frac{k_B T}{2m}\left(\Xi_1^2 + \Xi_2^2 \right) \\
      & = \ln(\rho) + \Xi_1 u_1 + \Xi_2 u_2 + \frac{c_s^2}{2}\left(\Xi_1^2 + \Xi_2^2 \right).
  \end{aligned}
\end{equation}
As~\eqref{eq: log of laplace of maxwell} is a finite polynomial, its Taylor series is finite and equal to the function itself.
Hence, we can directly read of the cumulants, following~\eqref{eq: definition of cumulants}, to be
\begin{equation}
  \label{eq: equilibrium cumulants}
  \begin{aligned}
    \kappa_{00}^{eq} & = \ln(\rho) \\
    \kappa_{10}^{eq} & = u_1 \\
    \kappa_{01}^{eq} & = u_2 \\
    \kappa_{11}^{eq} & = 0 \\
    \kappa_{20}^{eq} & = c_s^2  \\
    \kappa_{02}^{eq} & = c_s^2  \\
    \kappa_{21}^{eq} & = 0 \\
    \kappa_{12}^{eq} & = 0 \\
    \kappa_{22}^{eq} & = 0.
  \end{aligned}
\end{equation}
Comparing~\eqref{eq: equilibrium cumulants} to~\eqref{eq: equilibrium particle distributions}, we can identify several advantages.
First and foremost, the Maxwell distribution is fully incorporated and we don't introduce a truncation error\footnote{On closer analysis, this error is of order $O(u^2)$, c.f.~\cite[page 178]{wolf2000lattice}, and thus making the \gls{srt} only viable at low speeds}. Going further, we don't need to calculate any weighting $W_{ij}$ and as we incorporated the cell width $c$ in the calculation of the central moments and hence the cumulants, it is hidden from the collision calculation.


Additionally, we can simplify the $\kappa_{02}$ and $\kappa_{20}$.
As their equilibria are the same, their difference has zero equilibrium and their sum double the equilibrium, i.e.
\begin{equation}
  \begin{aligned}
    \kappa_{20}^{eq} - \kappa_{02}^{eq} & = 0  \\
    \kappa_{20}^{eq} + \kappa_{02}^{eq} & = 2 c_s^2.
  \end{aligned}
\end{equation}
%
Now, we can finally write a collision for this method.
We will do again a linear interpolation between our values and their equilibrium like in equation~\eqref{eq: post collision discrete}.
This time though, we have no need to restrict ourselves to just one parameter, but chose one for each cumulant.
Hence, we get the post collision cumulants $\kappa_{\alpha\beta}^*$
\begin{equation}
  \label{eq: post equilibrium cumulants}
  \begin{aligned}
    \kappa_{00}^{*} & = \kappa_{00} + \omega_1 \left( \ln(\rho) - \kappa_{00} \right) \\
    \kappa_{10}^{*} & = \kappa_{10} + \omega_2 \left( u_1 - \kappa_{10} \right) \\
    \kappa_{01}^{*} & = \kappa_{01} + \omega_3 \left( u_2 - \kappa_{01} \right) \\
    \kappa_{11}^{*} & = \kappa_{11} + \omega_4 \left( - \kappa_{11} \right) \\
    \kappa_{20}^{*} - \kappa_{02}^{*}
      & = (\kappa_{20} - \kappa_{02}) + \omega_5 \left( - \kappa_{20} + \kappa_{02} \right) \\
    \kappa_{20}^{*} + \kappa_{02}^{*}
      & = (\kappa_{20} + \kappa_{02}) + \omega_6 \left( 2 c_s^2 - \kappa_{20} - \kappa_{02} \right) \\
    \kappa_{21}^{*} & = \kappa_{21} + \omega_7 \left( - \kappa_{21} \right) \\
    \kappa_{12}^{*} & = \kappa_{12} + \omega_8 \left( - \kappa_{12} \right) \\
    \kappa_{22}^{*} & = \kappa_{22} + \omega_9 \left( - \kappa_{22} \right)
  \end{aligned}
\end{equation}
which can be simplified to
\begin{equation}
  \begin{aligned}
    \kappa_{00}^{*} & = \kappa_{00} \\
    \kappa_{10}^{*} & = \kappa_{10} \\
    \kappa_{01}^{*} & = \kappa_{01} \\
    \kappa_{11}^{*} & = (1-\omega_4)\kappa_{11} \\
    \kappa_{20}^{*} - \kappa_{02}^{*}
      & = (1-\omega_5) (\kappa_{20} - \kappa_{02}) \\
    \kappa_{20}^{*} + \kappa_{02}^{*}
      & = (1-\omega_6)(\kappa_{20} + \kappa_{02}) + \omega_6 \left( 2 c_s^2 \right) \\
    \kappa_{21}^{*} & = (1-\omega_7)\kappa_{21} \\
    \kappa_{12}^{*} & = (1-\omega_8)\kappa_{12} \\
    \kappa_{22}^{*} & = (1-\omega_9)\kappa_{22},
  \end{aligned}
\end{equation}
as the first three cumulants are the conserved quantities and equal to their equilibrium values, compare~\eqref{eq: definition of kappa_00} and following.
The recovered post equilibrium values $\kappa_{20}^{*}$ and $\kappa_{02}^{*}$ are
\begin{equation}
  \begin{pmatrix}
    \kappa_{20}^{*} \\
    \kappa_{02}^{*}
  \end{pmatrix}
  = \frac{1}{2}
  \begin{pmatrix}
    1 & 1 \\ -1 & 1
  \end{pmatrix}
  \begin{pmatrix}
    a\\
    b
  \end{pmatrix}
\end{equation}
where
\begin{equation}
 \begin{aligned}
   a \defined \kappa_{20}^{*} - \kappa_{02}^{*}
     & = (1-\omega_5) (\kappa_{20} - \kappa_{02}) \\
   b \defined \kappa_{20}^{*} + \kappa_{02}^{*}
     & = (1-\omega_6)(\kappa_{20} + \kappa_{02}) + \omega_6 \left( 2 c_s^2 \right).
 \end{aligned}
\end{equation}
As the cumulants $\kappa_{21}$ and $\kappa_{21}$ only vary in their direction, we want their relaxation to be the same.
In the analysis, Section~\ref{sub:Deriving the equation}, we will discover, that $\omega_4$ and $\omega_5$ should be equal and $\omega_6$ should be one.
This yields
\begin{equation}
  \label{eq: collision equation system full}
  \begin{aligned}
    \kappa_{00}^{*} & = \kappa_{00} \\
    \kappa_{10}^{*} & = \kappa_{10} \\
    \kappa_{01}^{*} & = \kappa_{01} \\
    \kappa_{11}^{*} & = (1-\omega_1)\kappa_{11} \\
    \kappa_{20}^{*} - \kappa_{02}^{*}
      & = (1-\omega_1) (\kappa_{20} - \kappa_{02}) \\
    \kappa_{20}^{*} + \kappa_{02}^{*}
      & = 2 c_s^2 \\
    \kappa_{21}^{*} & = (1-\omega_2)\kappa_{21} \\
    \kappa_{12}^{*} & = (1-\omega_2)\kappa_{12} \\
    \kappa_{22}^{*} & = (1-\omega_3)\kappa_{22},
  \end{aligned}
\end{equation}
where the $\omega_i$ are relabeled and not related to the ones used in~\eqref{eq: post equilibrium cumulants} and afterwards.
Multiplying equations~\eqref{eq: collision equation system full} with $m_{00}$ and introducing again the normalized cumulants,~\eqref{eq: definition normalized cumulants}, we get
\begin{equation}
  \label{eq: final collision all relaxations}
  \begin{aligned}
    K_{00}^{*} & = K_{00} \\
    K_{10}^{*} & = K_{10} \\
    K_{01}^{*} & = K_{01} \\
    K_{11}^{*} & = (1-\omega_1)K_{11} \\
    K_{20}^{*} - K_{02}^{*} & = (1-\omega_1) (K_{20} - K_{02}) \\
    K_{20}^{*} + K_{02}^{*} & = 2 m_{00} c_s^2 \\
    K_{21}^{*} & = (1-\omega_2)K_{21} \\
    K_{12}^{*} & = (1-\omega_2)K_{12} \\
    K_{22}^{*} & = (1-\omega_3)K_{22}.
  \end{aligned}
\end{equation}
