% !TEX root = ../thesis.tex
The equations in~\eqref{eq: collision final before simplification} resemble the most general form we can have and we seek to simplify it and determine the used relaxation parameters.
Thus, the asymptotic analysis in the next sections will base upon this equation.
Consequently, we present the final version of the collision in this section, anticipating the findings yet to come.

First of all, we turn our attention towards $\kappa_{21}$ and $\kappa_{21}$.
As they only vary in their direction, we want their relaxation to be the same.
In fact, neither $\omega_7$ nor $\omega_8$ nor $\omega_9$ play any role in our analysis up to second order and are safe to set equal to one according to this analysis.
Higher order errors are neglected in this view.

In the derivation of the Navier-Stokes equations in Section~\ref{sec: Momentum equations} and following, the only parameter which will be needed is $\omega_1$.
Like in \gls{srt}, it will play the role of the viscosity, respectively Reynolds number.

One should pause here for a second, and wonder, why we did not relax $\kappa_{20}$ and $\kappa_{02}$ separately and set their relaxation parameters equal, too.
In fact, we could still achieve this effect by setting $\omega_5=\omega_6$.

However, we will discover in the analysis, Section~\ref{sub: Deriving the equations}, that all terms with $\omega_6$ cancel out and $\omega_4$ and $\omega_5$ should be equal.
When setting $\omega_6=1$, we comply with the ideal gas law, c.f.~\ref{sub: Ideal gas law}.
This yields
\begin{equation}
  \label{eq: collision equation system full}
  \begin{aligned}
    \kappa_{00}^{*} & = \kappa_{00} \\
    \kappa_{10}^{*} & = \kappa_{10} \\
    \kappa_{01}^{*} & = \kappa_{01} \\
    \kappa_{11}^{*} & = (1-\omega)\kappa_{11} \\
    \kappa_{20}^{*} - \kappa_{02}^{*}
      & = (1-\omega) (\kappa_{20} - \kappa_{02}) \\
    \kappa_{20}^{*} + \kappa_{02}^{*}
      & = 2 c_s^2 \\
    \kappa_{21}^{*} & = 0 \\
    \kappa_{12}^{*} & = 0 \\
    \kappa_{22}^{*} & = 0,
  \end{aligned}
\end{equation}
Like in \gls{srt}, this remaining parameter will play the role of the viscosity, respectively Reynolds number.

Multiplying equations~\eqref{eq: collision equation system full} with $m_{00}$ and introducing again the normalized cumulants,~\eqref{eq: definition normalized cumulants}, we get
\begin{equation}
  \label{eq: final collision all relaxations}
  \begin{aligned}
    K_{00}^{*} & = K_{00} \\
    K_{10}^{*} & = K_{10} \\
    K_{01}^{*} & = K_{01} \\
    K_{11}^{*} & = (1-\omega)K_{11} \\
    K_{20}^{*} - K_{02}^{*} & = (1-\omega) (K_{20} - K_{02}) \\
    K_{20}^{*} + K_{02}^{*} & = 2 m_{00} c_s^2 \\
    K_{21}^{*} & = 0 \\
    K_{12}^{*} & = 0 \\
    K_{22}^{*} & = 0.
  \end{aligned}
\end{equation}

Finally, we can calculate the speed of sound for the D2Q9 as $c_s^2=\frac{c^2}{3}$, c.f.~\cite[page 175]{wolf2000lattice} and solve the small system of equations to finally get
\begin{equation}
  \label{eq: final collision one relaxation}
  \begin{aligned}
    K_{00}^{*} & = K_{00} \\
    K_{10}^{*} & = K_{10} \\
    K_{01}^{*} & = K_{01} \\
    K_{11}^{*} & = (1-\omega_1)K_{11} \\
    K_{20}^{*} & = \frac{m_{00}c^2}{3} + \frac{1}{2}(1-\omega_1) (K_{20} - K_{02}) \\
    K_{02}^{*} & = \frac{m_{00}c^2}{3} - \frac{1}{2}(1-\omega_1) (K_{20} - K_{02}) \\
    K_{21}^{*} & = 0 \\
    K_{12}^{*} & = 0 \\
    K_{22}^{*} & = 0.
  \end{aligned}
\end{equation}
