% !TEX root = ../thesis.tex

Now, it's time for a final wrap-up and outlook what might be done.

We started with a general description of Lattice Boltzmann and showed that the space of cumulants would be an intuitive one to formulate our collision.
Maybe here, we should think about turning our perception of Lattice Boltzmann simulations upside down.
Instead of formulating everything in the space of particle distributions and switch to another space for the collision, we could think of the problem being formulated naturally in the macroscopic variables\footnote{In the sense of cumulants.} and only being turned into particle distributions for the sake of streaming.
Of course, the actual code would mostly remain the same, but this could amongst others lead to new boundaries, the second large field of research in \gls{lbm}.

After we formulated the method and prove the convergence, we put it to the test and found, that cumulants are superior to \gls{srt} across the board.

As mentioned over the course of the thesis, it could be very rewarding to examine bigger stencils to incorporate energy flux and the full set of compressible Navier-Stokes equations.
