% !TEX root = ../thesis.tex
We now start with the expansion one order higher, the equation for $\epsilon^3$,~\eqref{eq: third order in epsilon}.
Taking $\alpha=1$ and $\beta=0$ we get
\begin{equation}
  \label{eq: start of momentum equation derivation}
  \begin{aligned}
     \partial_t m_{10}^{(1)} & =
    m_{10}^{*(3)} - m_{10}^{(3)} - \partial_x m_{20}^{*(2)} \\
    &\quad - \partial_y m_{11}^{*(2)} + \frac{1}{2}\partial_{xx} m_{30}^{*(1)} + \frac{1}{2} \partial_{yy} m_{12}^{*(1)} + \partial_{xy} m_{21}^{*(1)}.
  \end{aligned}
\end{equation}
We find the term $\partial_t m_{10}^{(1)} $ which is the desired time derivative from the momentum equation and
\begin{equation}
  m_{10}^{*(3)} - m_{10}^{(3)}\defines F_x^{(3)}
\end{equation}
which is the non-conserved term of the $x$-velocity, due to an external force.
We can already simplify this equation by using the aliasing of moments for $m_{30}$ and the collision of moments for $m_{21}$,~\eqref{eq: collide moments 21_1} and $m_{12}$,~\eqref{eq: collide moments 12_1} respectively to get
\begin{equation}
  \label{eq: start of momentum equation derivation}
  \begin{aligned}
    \partial_t m_{10}^{(1)} & =
    F_x^{(3)} - \partial_x \highlight{m_{20}^{*(2)}} - \partial_y \highlight[green]{m_{11}^{*(2)}} + \frac{1}{2}\partial_{xx} m_{10}^{(1)} + \frac{\theta}{2} \partial_{yy} m_{10}^{(1)} + \theta\partial_{xy} m_{01}^{(1)}.
  \end{aligned}
\end{equation}
This is already looking quite promising, but the green and red highlighted terms need some additional treatment.
