% !TEX root = ../thesis.tex
We now start with the expansion one order higher, the equation for $\epsilon^3$,~\eqref{eq: third order in epsilon}.
Taking $\alpha=1$ and $\beta=0$ we get
\begin{equation}
  \begin{aligned}
     \partial_t m_{10}^{(1)} & =
    m_{10}^{*(3)} - m_{10}^{(3)} - \partial_x m_{20}^{*^{(2)}} \\
    &\quad - \partial_y m_{11}^{*^{(2)}} + \frac{1}{2}\partial_{xx} m_{30}^{*^{(1)}} + \frac{1}{2} \partial_{yy} m_{12}^{*^{(1)}} + \partial_{xy} m_{21}^{*^{(1)}}.
  \end{aligned}
\end{equation}
We find the term $\partial_t m_{10}^{(1)} $ which is the desired time derivative from the momentum equation and
\begin{equation}
  m_{10}^{*(3)} - m_{10}^{(3)}\defines F_x^{(3)}
\end{equation}
which is the non-conserved term of the $x$-velocity, due to an external force.
We can already simplify this equation by using the aliasing of moments for $m_{30}$ and the identity~\eqref{eq: eq moments 21_1} and~\eqref{eq: eq moments 12_1} for $m_{21}$ respectively $m_{12}$, to get
\begin{equation}
  \label{eq: start of momentum equation derivation}
  \begin{aligned}
    \partial_t m_{10}^{(1)} & =
    F_x^{(3)} - \partial_x \highlight{m_{20}^{*^{(2)}}} - \partial_y \highlight[green]{m_{11}^{*^{(2)}}} + \frac{1}{2}\partial_{xx} m_{10}^{(1)} + \frac{\theta}{2} \partial_{yy} m_{10}^{(1)} + \theta\partial_{xy} m_{01}^{(1)}.
  \end{aligned}
\end{equation}
This is already looking quite promising, but the green and red highlighted terms need some additional treatment.

\subsubsection{Auxiliary equations}
\label{subs:Auxiliary equations}
To cope with those rouge terms, we take a look back at the second order in $\epsilon$ equation~\eqref{eq: second order in epsilon}.
We can extract three\footnote{We could insert $\alpha=1$, $\beta=0$ et vice versa, but this would lead to two equations where everything is zero due to equations~\eqref{eq: kill even moments odd order} and~\eqref{eq: kill odd moments even order}.
Choosing $\alpha$ or $\beta$ higher than two just results in the known equations due to moment aliasing.} more equations out of it.
Namely, we get for $\alpha=2$, $\beta=0$
\begin{equation}
  \label{eq: auxiliaries 1 raw}
  \partial_t m_{20}^{(0)} + m_{20}^{(2)} =  m_{20}^{*^{(2)}} - \partial_x m_{30}^{*^{(1)}} - \partial_y m_{21}^{*^{(1)}},
\end{equation}
for $\alpha=0$, $\beta=2$
\begin{equation}
  \label{eq: auxiliaries 2 raw}
  \partial_t m_{02}^{(0)} + m_{02}^{(2)} =  m_{02}^{*^{(2)}} - \partial_x m_{12}^{*^{(1)}} - \partial_y m_{03}^{*^{(1)}},
\end{equation}
and last but not least for $\alpha=1$, $\beta=1$
\begin{equation}
  \label{eq: auxiliaries 3 raw}
  \partial_t m_{11}^{(0)} + m_{11}^{(2)} =  m_{11}^{*^{(2)}} - \partial_x m_{21}^{*^{(1)}} - \partial_y m_{12}^{*^{(1)}}.
\end{equation}
We can reformulate equations~\eqref{eq: auxiliaries 1 raw} through~\eqref{eq: auxiliaries 3 raw} with the help of the aliasing of moments together with~\eqref{eq: eq moments 20_1} through~\eqref{eq: eq moments 12_1} to get to
\begin{alignat}{5}
  \label{eq: auxiliaries 1}
  m_{20}^{(2)} +&\, \partial_t c_s^2& m_{00}^{(0)} +&\, &\partial_x m_{10}^{(1)} +&\, c_s^2&\partial_y m_{01}^{(1)} &=  m_{20}^{*^{(2)}}&
  \\
  \label{eq: auxiliaries 2}
  m_{02}^{(2)} +&\, \partial_t c_s^2& m_{00}^{(0)} +&\, c_s^2&\partial_x m_{10}^{(1)} +&\, &\partial_y m_{01}^{(1)} &=  m_{02}^{*^{(2)}}&
  \\
  \label{eq: auxiliaries 3}
  m_{11}^{(2)} +&\, \partial_t &m_{11}^{(0)} +&\, c_s^2&\partial_x m_{01}^{(1)} +&\, c_s^2&\partial_y m_{10}^{(1)} &=  m_{11}^{*^{(2)}}&.
\end{alignat}
