% !TEX root = ../thesis.tex
Following Section~\ref{sub: Deriving the continuity equation} we start with the third order terms of
$\epsilon$ in~\eqref{eq: final nondimensionalised expansion series}, namely equation~\eqref{eq: third order in epsilon}.
Taking $\alpha=1$ and $\beta=0$ we get
\begin{equation}
  \begin{aligned}
     \partial_t m_{10}^{\circ^{(1)}} & =
    m_{10}^{*(3)} - m_{10}^{\circ{(3)}} - \partial_x m_{20}^{*^{(2)}} \\
    &\quad - \partial_y m_{11}^{*^{(2)}} + \frac{1}{2}\partial_{xx} m_{30}^{*^{(1)}} + \frac{1}{2} \partial_{yy} m_{12}^{*^{(1)}} + \partial_{xy} m_{21}^{*^{(1)}}.
  \end{aligned}
\end{equation}
We see the term $\partial_t m_{10}^{(1)} $ which is the desired time derivative from the momentum equation and define
\begin{equation}
   F_x^{(3)}\defined m_{10}^{*(3)} - m_{10}^{\circ{(3)}}
\end{equation}
which is the non-conserved term of the $x$-velocity, due to an external force.
We can instantly simplify this equation by using the aliasing of moments\footnote{This is one of the few things which will go wrong, when going to a bigger stencil.
In this case, we could expand $K_{30}^{eq}$ and seek to replace this term.
In our case though, this is not needed.} for $m_{30}$, c.f.~\eqref{eq:aliasing of moments 1}\footnote{Keep in mind, that we rescaled the moments making the aliasing a one to one relationship without the $c$.}, and the identities~\eqref{eq: aliasing all} to get
\begin{equation}
  \label{eq: start of momentum equation derivation}
  \begin{aligned}
    \partial_t m_{10}^{(1)} & =
    F_x^{(3)} - \partial_x \highlight{m_{20}^{*^{(2)}}} - \partial_y \highlight[green]{m_{11}^{*^{(2)}}} + \frac{1}{2}\partial_{xx} m_{10}^{(1)} + \frac{c_s^2}{2} \partial_{yy} m_{10}^{(1)} + c_s^2\partial_{xy} m_{01}^{(1)}.
  \end{aligned}
\end{equation}
This is already looking quite promising, but the green and red highlighted terms need some additional treatment.

\subsubsection{Auxiliary equations}
\label{subs:Auxiliary equations}
To cope with those rogue terms, we take a look back at the second order in $\epsilon$ equation~\eqref{eq: second order in epsilon}.
We can extract three\footnote{We could insert $\alpha=1$, $\beta=0$ et vice versa, but this would lead to two equations where everything is zero due to equations~\eqref{eq: kill even moments odd order} and~\eqref{eq: kill odd moments even order}.
Choosing $\alpha$ or $\beta$ higher than two yields the previously stated equations due to moment aliasing.} more equations out of it.
Namely, we get for $\alpha=2$, $\beta=0$
\begin{equation}
  \label{eq: auxiliaries 1 raw}
  \partial_t m_{20}^{\circ^{(0)}} + m_{20}^{\circ^{(2)}} =  m_{20}^{*^{(2)}} - \partial_x m_{30}^{*^{(1)}} - \partial_y m_{21}^{*^{(1)}},
\end{equation}
for $\alpha=0$, $\beta=2$
\begin{equation}
  \label{eq: auxiliaries 2 raw}
  \partial_t m_{02}^{\circ^{(0)}} + m_{02}^{\circ^{(2)}} =  m_{02}^{*^{(2)}} - \partial_x m_{12}^{*^{(1)}} - \partial_y m_{03}^{*^{(1)}},
\end{equation}
and last but not least for $\alpha=1$, $\beta=1$
\begin{equation}
  \label{eq: auxiliaries 3 raw}
  \partial_t m_{11}^{\circ^{(0)}} + m_{11}^{\circ^{(2)}} =  m_{11}^{*^{(2)}} - \partial_x m_{21}^{*^{(1)}} - \partial_y m_{12}^{*^{(1)}}.
\end{equation}
We can reformulate equations~\eqref{eq: auxiliaries 1 raw} through~\eqref{eq: auxiliaries 3 raw} with the aliasing of moments~\eqref{eq:aliasing of moments 1}
and~\eqref{eq:aliasing of moments 2} together with~\eqref{eq: aliasing all} to finally get
\begin{alignat}{5}
  \label{eq: auxiliaries 1}
  m_{20}^{\circ^{(2)}} +&\, \partial_t c_s^2& m_{00}^{(0)} +&\, &\partial_x m_{10}^{(1)} +&\, c_s^2&\partial_y m_{01}^{(1)} &=  m_{20}^{*^{(2)}}&
  \\
  \label{eq: auxiliaries 2}
  m_{02}^{\circ^{(2)}} +&\, \partial_t c_s^2& m_{00}^{(0)} +&\, c_s^2&\partial_x m_{10}^{(1)} +&\, &\partial_y m_{01}^{(1)} &=  m_{02}^{*^{(2)}}&
  \\
  \label{eq: auxiliaries 3}
  m_{11}^{\circ^{(2)}} +&\, \partial_t &m_{11}^{(0)} +&\, c_s^2&\partial_x m_{01}^{(1)} +&\, c_s^2&\partial_y m_{10}^{(1)} &=  m_{11}^{*^{(2)}}&.
\end{alignat}
Adding respectively subtracting equation~\eqref{eq: auxiliaries 2} from equation~\eqref{eq: auxiliaries 1} and inserting the collision of
moments,~\eqref{eq: collide moments 20p02_2} and~\eqref{eq: collide moments 20m02_2} results in
\begin{align}
  \nonumber
  &m_{20}^{\circ^{(2)}} + m_{02}^{\circ^{(2)}} + 2\partial_t c_s^2 m_{00}^{(0)} + (1+c_s^2)(\partial_x m_{10}^{(1)} + \partial_y m_{01}^{(1)})
  \\=\,&
  \label{eq: auxiliaries 4}
  (1-\omega_6)\left(  m_{20}^{\circ^{(2)}} + m_{02}^{\circ^{(2)}}\right)
  + \omega_6 \left( \frac{ m_{10}^{{(1)}^2} + m_{01}^{{(1)}^2}}{m_{00}^{(0)}}
  + 2 c_s^2 m_{00}^{(2)} \right)
\end{align}
respectively
\begin{align}
  \nonumber
  &m_{20}^{\circ^{(2)}} - m_{02}^{\circ^{(2)}} + (1 - c_s^2)\partial_x m_{10}^{(1)} + (c_s^2 - 1)\partial_y m_{01}^{(1)}
  \\=\,&
  \label{eq: auxiliaries 5}
  (1-\omega_5) \left(m_{20}^{\circ^{(2)}}-m_{02}^{\circ^{(2)}}\right) + \omega_5 \frac{ m_{10}^{{(1)}^2} - m_{01}^{{(1)}^2}}{m_{00}^{(0)}}.
\end{align}

Solving for $m_{20}^{\circ^{(2)}} + m_{02}^{\circ^{(2)}}$ and equally $m_{20}^{\circ^{(2)}} + m_{02}^{\circ^{(2)}}$ is now as easy as
\begin{align}
  \label{eq: auxiliaries 6}
  m_{20}^{\circ^{(2)}} + m_{02}^{\circ^{(2)}}
  =
  2 c_s^2 m_{00}^{(2)}
  + \frac{ m_{10}^{{(1)}^2} + m_{01}^{{(1)}^2}}{m_{00}^{(0)}}
  - \frac{2}{\omega_6}\partial_t c_s^2 m_{00}^{(0)}
  - \frac{1}{\omega_6}(1+c_s^2)(\partial_x m_{10}^{(1)} + \partial_y m_{01}^{(1)})
\end{align}
respectively
\begin{align}
  \label{eq: auxiliaries 7}
  m_{20}^{\circ^{(2)}} - m_{02}^{\circ^{(2)}}
  =
   \frac{ m_{10}^{{(1)}^2} - m_{01}^{{(1)}^2}}{m_{00}^{(0)}}
   - \frac{1}{\omega_5} (1 - c_s^2)\partial_x m_{10}^{(1)}
   - \frac{1}{\omega_5} (c_s^2 - 1)\partial_y m_{01}^{(1)}.
\end{align}

The same procedure is now applied to equation~\eqref{eq: auxiliaries 3} with the corresponding collision~\eqref{eq: collide moments 11_2} to get
\begin{align}
  \label{eq: auxiliaries 8}
  &m_{11}^{\circ^{(2)}} + \partial_t m_{11}^{(0)} + c_s^2\partial_x m_{01}^{(1)} + c_s^2\partial_y m_{10}^{(1)} =  (1-\omega_4)m_{11}^{(2)} + \omega_4 \frac{ m_{10}^{(1)}m_{01}^{(1)}}{m_{00}^{(0)}}
  \\
  \equivalent\quad
  &m_{11}^{\circ^{(2)}}
  =
  \frac{ m_{10}^{(1)}m_{01}^{(1)}}{m_{00}^{(0)}}
  - \frac{1}{\omega_4} \partial_t m_{11}^{(0)}
  - \frac{1}{\omega_4} c_s^2\partial_x m_{01}^{(1)}
  - \frac{1}{\omega_4} c_s^2\partial_y m_{10}^{(1)}
  .
\end{align}
As a last step before we actually insert anything, we bring those equations in the right form to directly insert them in equation~\eqref{eq: start of momentum equation derivation}.

Therefore, we add equations~\eqref{eq: auxiliaries 6} and~\eqref{eq: auxiliaries 7} and take $-\frac{1}{2}\partial_x$ of the whole equation, resulting in
\begin{align}
  -\partial_x m_{20}^{\circ^{(2)}}
  =
  - \partial_x c_s^2 m_{00}^{(2)}
  &\,
  - \partial_x\frac{1}{2}\frac{ m_{10}^{{(1)}^2} + m_{01}^{{(1)}^2}}{m_{00}^{(0)}}
  + \partial_x\frac{1}{\omega_6}\partial_t c_s^2 m_{00}^{(0)}
  \nonumber
  \\&\,
  + \partial_x\frac{1}{2\omega_6}(1+c_s^2)(\partial_x m_{10}^{(1)} + \partial_y m_{01}^{(1)})
  \nonumber
  \\&\,
  - \partial_x\frac{1}{2}\frac{ m_{10}^{{(1)}^2} - m_{01}^{{(1)}^2}}{m_{00}^{(0)}}
  \nonumber
  \\&\,
  + \partial_x\frac{1}{2\omega_5} (1 - c_s^2)\partial_x m_{10}^{(1)}
  + \partial_x\frac{1}{2\omega_5} (c_s^2 - 1)\partial_y m_{01}^{(1)}
  \nonumber
  \\
  =
  - \partial_x c_s^2 m_{00}^{(2)}
  &\,
  - \partial_x\frac{ m_{10}^{{(1)}^2} }{m_{00}^{(0)}}
  + \partial_x\left(\frac{1}{2\omega_6}(1+c_s^2)(\partial_x m_{10}^{(1)} + \partial_y m_{01}^{(1)})\right)
  \nonumber
  \\&\,
  \nonumber
  + \partial_x\left(\frac{1}{2\omega_5} (1 - c_s^2)\partial_x m_{10}^{(1)}\right)
  + \partial_x\left(\frac{1}{2\omega_5} (c_s^2 - 1)\partial_y m_{01}^{(1)}\right),
\end{align}
where we used~\eqref{eq: derivative of zeroth order} to get rid of the time derivative.

Up to this point, we did not need any assumptions on our relaxation parameters $\omega_i$.
For further simplification, we will assume $\omega_5$, $\omega_6$ and $c_s^2$ to be constant in space.
In our isothermal setting, this is just fine, as the the $\omega_i$ will be connected to the viscosity which is a function of temperature\footnote{Interestingly,
upon looking at the compressible momentum equations~\eqref{eq: compressible NS}, the viscosity is also present in the first derivative, further hinting at an easy expansion to fully compressible flows.
}; as is $c_s^2$.
Hence, we get
\begin{align}
  \label{eq: auxiliaries 9}
  -\partial_x m_{20}^{\circ^{(2)}}
  =
  - c_s^2 \partial_x  m_{00}^{(2)}
  &\,
  - \partial_x\frac{ m_{10}^{{(1)}^2} }{m_{00}^{(0)}}
  \begin{aligned}[t]
    &+ \partial_{xx} m_{10}^{(1)} \left(\frac{1}{2\omega_6}(1+c_s^2) + \frac{1}{2\omega_5} (1 - c_s^2)\right)
    \\&
    + \partial_{xy} m_{01}^{(1)} \left(\frac{1}{2\omega_6}(1+c_s^2) + \frac{1}{2\omega_5} (c_s^2 - 1)\right).
  \end{aligned}
\end{align}

The same treatment is applied to~\eqref{eq: auxiliaries 8}, this time multiplied with $-\partial_y$, resulting in
\begin{align}
  \label{eq: auxiliaries 10}
  -\partial_y m_{11}^{\circ^{(2)}}
  =
  -\partial_y \frac{ m_{10}^{(1)}m_{01}^{(1)}}{m_{00}^{(0)}}
  + \frac{c_s^2}{\omega_4}\partial_{xy} m_{01}^{(1)}
  + \frac{c_s^2}{\omega_4}\partial_{yy} m_{10}^{(1)}
  .
\end{align}
With this, the preliminary work is finally done and we can go over to actually derive the momentum equations.

\subsubsection{Finalizing the derivation}
\label{subs:Finalizing the derivation}
First of all, we insert equations~\eqref{eq: auxiliaries 1} and~\eqref{eq: auxiliaries 3} into~\eqref{eq: start of momentum equation derivation}
and afterwards the final equations~\eqref{eq: auxiliaries 9} and~\eqref{eq: auxiliaries 10} to get
\begin{align}
  \nonumber
  \partial_t m_{10}^{(1)} =&\,
  F_x^{(3)}
  - \partial_x m_{20}^{*^{(2)}}
  - \partial_y m_{11}^{*^{(2)}}
  + \frac{1}{2}\partial_{xx} m_{10}^{(1)}
  + \frac{c_s^2}{2} \partial_{yy} m_{10}^{(1)} + c_s^2\partial_{xy} m_{01}^{(1)}
  \\\nonumber =&\,
  F_x^{(3)}
  - \partial_x m_{20}^{\circ^{(2)}}
  - \partial_{xx} m_{10}^{(1)}
  - c_s^2\partial_{xy} m_{01}^{(1)}
  \\\nonumber &\,
  - \partial_y m_{11}^{\circ^{(2)}}
  - c_s^2 \partial_{xy} m_{01}^{(1)}
  - c_s^2 \partial_{yy} m_{10}^{(1)}
  + \frac{1}{2}\partial_{xx} m_{10}^{(1)}
  + \frac{c_s^2}{2} \partial_{yy} m_{10}^{(1)} + c_s^2\partial_{xy} m_{01}^{(1)}
  \\\nonumber =&\,
  F_x^{(3)}
  - c_s^2 \partial_x  m_{00}^{(2)}
  \\\nonumber &\,
  - \partial_x\frac{ m_{10}^{{(1)}^2} }{m_{00}^{(0)}}
  + \partial_{xx} m_{10}^{(1)} \left(\frac{1}{2\omega_6}(1+c_s^2)
  + \frac{1}{2\omega_5} (1 - c_s^2)\right)
  \\\nonumber &\,
  + \partial_{xy} m_{01}^{(1)} \left(\frac{1}{2\omega_6}(1+c_s^2)
  + \frac{1}{2\omega_5} (c_s^2 - 1)\right)
  - \partial_{xx} m_{10}^{(1)}
  - c_s^2\partial_{xy} m_{01}^{(1)}
  \\\nonumber &\,
  -\partial_y \frac{ m_{10}^{(1)}m_{01}^{(1)}}{m_{00}^{(0)}}
  + \frac{c_s^2}{\omega_4}\partial_{xy} m_{01}^{(1)}
  + \frac{c_s^2}{\omega_4}\partial_{yy} m_{10}^{(1)}
  - c_s^2 \partial_{xy} m_{01}^{(1)}
  - c_s^2 \partial_{yy} m_{10}^{(1)}
  \\\nonumber &\,
  + \frac{1}{2}\partial_{xx} m_{10}^{(1)}
  + \frac{c_s^2}{2} \partial_{yy} m_{10}^{(1)} + c_s^2\partial_{xy} m_{01}^{(1)}
  \end{align}
  \begin{align}
    \nonumber
    \equivalent\quad
  \partial_t m_{10}^{(1)}
  + \partial_x \frac{ m_{10}^{{(1)}^2} }{m_{00}^{(0)}}
  + \partial_y \frac{ m_{10}^{(1)}m_{01}^{(1)}}{m_{00}^{(0)}}
  =&\,
  F_x^{(3)}
  - c_s^2 \partial_x  m_{00}^{(2)}
  \\\nonumber &\,
  + \partial_{xx} m_{10}^{(1)} \left(\frac{1}{2\omega_6}(1+c_s^2)
  + \frac{1}{2\omega_5} (1 - c_s^2)\right)
  \\\nonumber &\,
  + \partial_{xy} m_{01}^{(1)} \left(\frac{1}{2\omega_6}(1+c_s^2)
  + \frac{1}{2\omega_5} (c_s^2 - 1)\right)
  \\\nonumber &\,
  - \partial_{xx} m_{10}^{(1)}
  - c_s^2\partial_{xy} m_{01}^{(1)}
  \\\nonumber &\,
  + \frac{c_s^2}{\omega_4}\partial_{xy} m_{01}^{(1)}
  + \frac{c_s^2}{\omega_4}\partial_{yy} m_{10}^{(1)}
  - c_s^2 \partial_{xy} m_{01}^{(1)}
  - c_s^2 \partial_{yy} m_{10}^{(1)}
  \\ &\,
  + \frac{1}{2}\partial_{xx} m_{10}^{(1)}
  + \frac{c_s^2}{2} \partial_{yy} m_{10}^{(1)} + c_s^2\partial_{xy} m_{01}^{(1)}
  .\label{eq: navier stokes skeleton 1}
\end{align}
This is good news so far.

To go further we will tackle the equation in separate parts.
We start with the spatial derivatives on the left hand side and take the partial derivative
\begin{align}
  \label{eq: convective parts}
  \partial_x \frac{ m_{10}^{{(1)}^2} }{m_{00}^{(0)}}
  + \partial_y \frac{ m_{10}^{(1)}m_{01}^{(1)}}{m_{00}^{(0)}}
  =
  \frac{ m_{10}^{{(1)}} }{m_{00}^{(0)}} \partial_x  m_{10}^{{(1)}}
  +  m_{10}^{{(1)}} \partial_x \frac{ m_{10}^{{(1)}} }{m_{00}^{(0)}}
  + \frac{ m_{10}^{(1)}}{m_{00}^{(0)}}\partial_y m_{01}^{(1)}
  + m_{01}^{(1)}\partial_y \frac{ m_{10}^{(1)}}{m_{00}^{(0)}},
\end{align}
to see that those are precisely the second order convective   terms of~\eqref{eq: navier stokes goal}.
Additionally, the first line of equation~\eqref{eq: navier stokes skeleton 1} hosts the time derivative of the momentum density and the external force, which has to be normed by the density.

Last but definitely not least, the pressure derivative is hidden in the density derivative $-c_s^2 \partial_x  m_{00}^{(2)}$ as density is equal to the pressure at the local equilibrium up to the factor $c_s^2$.

As the density is independent on the direction in both the equilibrium and non equilibrium case, we would like to set the pressure magnitude\footnote{Which is up to scaling $K_{20} + K_{02}$} also to the equilibrium.

Turning to our second derivatives, we can cancel out the $\frac{1}{2\omega_6}(1+c_s^2)$ term with the help of equation~\eqref{eq: continuity derivative x} and hence we're safe to set $K_{20} + K_{02}$ to equilibrium via $\omega_6=1$.
The remaining terms with second order derivatives are with $c_s^2=\frac{1}{3}$
\begin{align}
  \label{eq: diffusive start}
  &\partial_{xx} m_{10}^{(1)} \left(
     \frac{1}{2\omega_5} (1 - c_s^2)
     - 1
     + \frac{1}{2}
  \right)
  \\&
  \nonumber
  + c_s^2\partial_{yy} m_{10}^{(1)} \left(
    \frac{1}{\omega_4}
    - 1
    + \frac{1}{2}
  \right)
  \\&
  \nonumber
  + \partial_{xy} m_{01}^{(1)} \left(
    \frac{1}{2\omega_5} (c_s^2 - 1)
    - c_s^2
    - c_s^2
    + c_s^2
    + \frac{c_s^2}{\omega_4}
  \right)
  \\=\ &
  \nonumber
  \partial_{xx} m_{10}^{(1)} \left(
    \frac{1}{3}\frac{1}{\omega_5}
    - \frac{1}{2}
  \right)
  \\&
  \nonumber
  + \partial_{yy} m_{10}^{(1)} \frac{1}{3}\left(
    \frac{1}{\omega_4}
    - \frac{1}{2}
  \right)
  \\&
  \nonumber
  + \partial_{xy} m_{01}^{(1)} \left(
    - \frac{1}{3\omega_5}
    - \frac{1}{3}
    + \frac{1}{3\omega_4}
  \right)
  \\=\ &
  \nonumber
  \partial_{xx} m_{10}^{(1)} \left(
    \frac{1}{3}\frac{1}{\omega_5}
    - \frac{1}{2}
    + \frac{1}{3\omega_5}
    + \frac{1}{3}
    - \frac{1}{3\omega_4}
  \right)
  \\&
  \nonumber
  + \partial_{yy} m_{10}^{(1)} \frac{1}{3}\left(
    \frac{1}{\omega_4}
    - \frac{1}{2}
  \right)
  \\=\ &
  \nonumber
  \partial_{xx} m_{10}^{(1)} \frac{1}{3} \left(
    \frac{2}{\omega_5}
    - \frac{1}{\omega_4}
    - \frac{1}{2}
  \right)
  + \partial_{yy} m_{10}^{(1)} \frac{1}{3}\left(
    \frac{1}{\omega_4}
    - \frac{1}{2}
  \right),
\end{align}
where we again used the identity~\eqref{eq: continuity derivative x}.

When choosing now $\omega_4=\omega_5$, we get, using the same identity
\begin{align}
  \nonumber
  &\partial_{xx} m_{10}^{(1)} \frac{1}{3} \left(
    \frac{2}{\omega_5}
    - \frac{1}{\omega_4}
    - \frac{1}{2}
  \right)
  + \partial_{yy} m_{10}^{(1)} \frac{1}{3}\left(
    \frac{1}{\omega_4}
    - \frac{1}{2}
  \right)
  \\=\
  \nonumber
  &\partial_{xx} m_{10}^{(1)} \frac{1}{3} \left(
    \frac{1}{\omega_4}
    - \frac{1}{2}
  \right)
  + \partial_{yy} m_{10}^{(1)} \frac{1}{3}\left(
    \frac{1}{\omega_4}
    - \frac{1}{2}
  \right)
  \\=\ &
  \nonumber
  \frac{1}{3}\left(
    \frac{1}{\omega_4}
    - \frac{1}{2}
  \right) \left(\frac{4}{3}\partial_{xx} m_{10}^{(1)} + \partial_{yy} m_{10}^{(1)} + \frac{1}{3}\partial_{xy} m_{01}^{(1)}\right)
  \\=\ &
  \label{eq: diffusive final}
  \frac{m_{00}^{(0)}}{3}\left(
    \frac{1}{\omega_4}
    - \frac{1}{2}
  \right) \left(\frac{4}{3}\partial_{xx} \frac{m_{10}^{(1)}}{m_{00}^{(0)}}
  + \partial_{yy} \frac{m_{10}^{(1)}}{m_{00}^{(0)}} + \frac{1}{3}\partial_{xy} \frac{m_{01}^{(1)}}{m_{00}^{(0)}}\right)
\end{align}
which is our desired diffusive term.

Inserting this back in~\eqref{eq: navier stokes skeleton 1}, we get
\begin{align}
  \nonumber
& \partial_t m_{10}^{(1)}
+\frac{ m_{10}^{{(1)}} }{m_{00}^{(0)}} \partial_x  m_{10}^{{(1)}}
+  m_{10}^{{(1)}} \partial_x \frac{ m_{10}^{{(1)}} }{m_{00}^{(0)}}
+ \frac{ m_{10}^{(1)}}{m_{00}^{(0)}}\partial_y m_{01}^{(1)}
+ m_{01}^{(1)}\partial_y \frac{ m_{10}^{(1)}}{m_{00}^{(0)}},
\\=&\,
- c_s^2 \partial_x  m_{00}^{(2)}
+ \frac{m_{00}^{(0)}}{3}\left(
    \frac{1}{\omega_4}
    - \frac{1}{2}
  \right) \left(\frac{4}{3}\partial_{xx} \frac{m_{10}^{(1)}}{m_{00}^{(0)}}
  + \partial_{yy} \frac{m_{10}^{(1)}}{m_{00}^{(0)}} + \frac{1}{3}\partial_{xy} \frac{m_{01}^{(1)}}{m_{00}^{(0)}}\right)
+ F_x^{(3)},
\label{eq: navier stokes derived}
\end{align}
which is the first order of the anticipated $x$-direction of the compressible momentum equations.

Here, the dynamic viscosity $\mu$ in~\eqref{eq: navier stokes goal} is represented by $\frac{m_{00}^{(0)}}{3}\left(\frac{1}{\omega_4} - \frac{1}{2}\right)$.
The dynamic viscosity $\mu$  is related to the kinematic viscosity $\nu$ like $\nu\rho=\mu$.
Hence, our kinematic viscosity is
\begin{equation}
  \nu = \frac{1}{3}\left(
      \frac{1}{\omega_4}
      - \frac{1}{2}
    \right),
\end{equation}
just like in \gls{srt}.

The next order of the equations is again trivially fulfilled as all terms are zero due to~\eqref{eq: kill even moments odd order} and~\eqref{eq: kill odd moments even order},
just like in equation~\eqref{eq: conti second order}, which makes the cumulant \gls{lbm} second order accurate.
\todo{something?}

One caveat dampens the mood however and should not be left unspoken.
The main reason which makes compressible flows interesting is the link between pressure and temperature, the second of which cannot be represented in our lattice.
Together with equation~\eqref{eq: continuity derivative x} the momentum equations are equal in their compressible and noncompressible form.
This can be seen perfectly when comparing~\eqref{eq:compressible NS isotherm reduced} to~\eqref{eq: navier stokes} and identifying again $\nu \rho = \mu$.
It remains to be examined whether this will vanish if we incorporate a bigger stencil and are able to represent more moments and thus perhaps some higher order approximation of the aforementioned equations.
