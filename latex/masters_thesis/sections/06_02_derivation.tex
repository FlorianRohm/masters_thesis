% !TEX root = ../thesis.tex
We now start with the expansion one order higher, the equation for $\epsilon^3$,~\eqref{eq: third order in epsilon}.
Taking $\alpha=1$ and $\beta=0$ we get
\begin{equation}
  \begin{aligned}
     \partial_t m_{10}^{(1)} & =
    m_{10}^{*(3)} - m_{10}^{(3)} - \partial_x m_{20}^{*^{(2)}} \\
    &\quad - \partial_y m_{11}^{*^{(2)}} + \frac{1}{2}\partial_{xx} m_{30}^{*^{(1)}} + \frac{1}{2} \partial_{yy} m_{12}^{*^{(1)}} + \partial_{xy} m_{21}^{*^{(1)}}.
  \end{aligned}
\end{equation}
We find the term $\partial_t m_{10}^{(1)} $ which is the desired time derivative from the momentum equation and
\begin{equation}
  m_{10}^{*(3)} - m_{10}^{(3)}\defines F_x^{(3)}
\end{equation}
which is the non-conserved term of the $x$-velocity, due to an external force.
We can already simplify this equation by using the aliasing of moments\footnote{This is one of the few things which will go bad, when going to a bigger stencil.
In this case, we could expand $K_{30}^{eq}$ and seek to replace this term. In our case though, this is not needed.} for $m_{30}$ and the identity~\eqref{eq: eq moments 21_1} and~\eqref{eq: eq moments 12_1} for $m_{21}$ respectively $m_{12}$, to get
\begin{equation}
  \label{eq: start of momentum equation derivation}
  \begin{aligned}
    \partial_t m_{10}^{(1)} & =
    F_x^{(3)} - \partial_x \highlight{m_{20}^{*^{(2)}}} - \partial_y \highlight[green]{m_{11}^{*^{(2)}}} + \frac{1}{2}\partial_{xx} m_{10}^{(1)} + \frac{\theta}{2} \partial_{yy} m_{10}^{(1)} + \theta\partial_{xy} m_{01}^{(1)}.
  \end{aligned}
\end{equation}
This is already looking quite promising, but the green and red highlighted terms need some additional treatment.

\subsubsection{Auxiliary equations}
\label{subs:Auxiliary equations}
To cope with those rouge terms, we take a look back at the second order in $\epsilon$ equation~\eqref{eq: second order in epsilon}.
We can extract three\footnote{We could insert $\alpha=1$, $\beta=0$ et vice versa, but this would lead to two equations where everything is zero due to equations~\eqref{eq: kill even moments odd order} and~\eqref{eq: kill odd moments even order}.
Choosing $\alpha$ or $\beta$ higher than two just results in the known equations due to moment aliasing.} more equations out of it.
Namely, we get for $\alpha=2$, $\beta=0$
\begin{equation}
  \label{eq: auxiliaries 1 raw}
  \partial_t m_{20}^{(0)} + m_{20}^{(2)} =  m_{20}^{*^{(2)}} - \partial_x m_{30}^{*^{(1)}} - \partial_y m_{21}^{*^{(1)}},
\end{equation}
for $\alpha=0$, $\beta=2$
\begin{equation}
  \label{eq: auxiliaries 2 raw}
  \partial_t m_{02}^{(0)} + m_{02}^{(2)} =  m_{02}^{*^{(2)}} - \partial_x m_{12}^{*^{(1)}} - \partial_y m_{03}^{*^{(1)}},
\end{equation}
and last but not least for $\alpha=1$, $\beta=1$
\begin{equation}
  \label{eq: auxiliaries 3 raw}
  \partial_t m_{11}^{(0)} + m_{11}^{(2)} =  m_{11}^{*^{(2)}} - \partial_x m_{21}^{*^{(1)}} - \partial_y m_{12}^{*^{(1)}}.
\end{equation}
We can reformulate equations~\eqref{eq: auxiliaries 1 raw} through~\eqref{eq: auxiliaries 3 raw} with the help of the aliasing of moments together with~\eqref{eq: eq moments 20_1} through~\eqref{eq: eq moments 12_1} to get to
\begin{alignat}{5}
  \label{eq: auxiliaries 1}
  m_{20}^{(2)} +&\, \partial_t c_s^2& m_{00}^{(0)} +&\, &\partial_x m_{10}^{(1)} +&\, c_s^2&\partial_y m_{01}^{(1)} &=  m_{20}^{*^{(2)}}&
  \\
  \label{eq: auxiliaries 2}
  m_{02}^{(2)} +&\, \partial_t c_s^2& m_{00}^{(0)} +&\, c_s^2&\partial_x m_{10}^{(1)} +&\, &\partial_y m_{01}^{(1)} &=  m_{02}^{*^{(2)}}&
  \\
  \label{eq: auxiliaries 3}
  m_{11}^{(2)} +&\, \partial_t &m_{11}^{(0)} +&\, c_s^2&\partial_x m_{01}^{(1)} +&\, c_s^2&\partial_y m_{10}^{(1)} &=  m_{11}^{*^{(2)}}&.
\end{alignat}
Adding respectively subtracting equation~\eqref{eq: auxiliaries 2} from equation~\eqref{eq: auxiliaries 1} and inserting the collision of moments,~\eqref{eq: collide moments 20m02_2} and~\eqref{eq: collide moments 20p02_2} results in
\begin{align}
  \label{eq: auxiliaries 4}
  &m_{20}^{(2)} + m_{02}^{(2)} + 2\partial_t c_s^2 m_{00}^{(0)} + (1+c_s^2)(\partial_x m_{10}^{(1)} + \partial_y m_{01}^{(1)})
  \\=\,&
  (1-\omega_6)\left(  m_{20}^{(2)} + m_{02}^{(2)}\right)
  + \omega_6 \left( \frac{ m_{10}^{{(1)}^2} + m_{01}^{{(1)}^2}}{m_{00}^{(0)}}
  + 2 c_s^2 m_{00}^{(2)} \right)
\end{align}
respectively
\begin{align}
  \label{eq: auxiliaries 5}
  &m_{20}^{(2)} - m_{02}^{(2)} + (1 - c_s^2)\partial_x m_{10}^{(1)} + (c_s^2 - 1)\partial_y m_{01}^{(1)}
  \\=\,&
  (1-\omega_5) \left(m_{20}^{(2)}-m_{02}^{(2)}\right) + \omega_5 \frac{ m_{10}^{{(1)}^2} - m_{01}^{{(1)}^2}}{m_{00}^{(0)}}.
\end{align}

Solving for $m_{20}^{(2)} + m_{02}^{(2)}$ and equally $m_{20}^{(2)} + m_{02}^{(2)}$ is now as easy as
\begin{align}
  \label{eq: auxiliaries 6}
  m_{20}^{(2)} + m_{02}^{(2)}
  =
  2 c_s^2 m_{00}^{(2)}
  + \frac{ m_{10}^{{(1)}^2} + m_{01}^{{(1)}^2}}{m_{00}^{(0)}}
  - \frac{2}{\omega_6}\partial_t c_s^2 m_{00}^{(0)}
  - \frac{1}{\omega_6}(1+c_s^2)(\partial_x m_{10}^{(1)} + \partial_y m_{01}^{(1)})
\end{align}
respectively
\begin{align}
  \label{eq: auxiliaries 7}
  m_{20}^{(2)} - m_{02}^{(2)}
  =
   \frac{ m_{10}^{{(1)}^2} - m_{01}^{{(1)}^2}}{m_{00}^{(0)}}
   - \frac{1}{\omega_5} (1 - c_s^2)\partial_x m_{10}^{(1)}
   - \frac{1}{\omega_5} (c_s^2 - 1)\partial_y m_{01}^{(1)}.
\end{align}

The same procedure is now applied to equation~\eqref{eq: auxiliaries 3} with the corresponding collision~\eqref{eq: collide moments 11_2} to get
\begin{align}
  \label{eq: auxiliaries 8}
  &m_{11}^{(2)} + \partial_t m_{11}^{(0)} + c_s^2\partial_x m_{01}^{(1)} + c_s^2\partial_y m_{10}^{(1)} =  (1-\omega_4)m_{11}^{(2)} + \omega_4 \frac{ m_{10}^{(1)}m_{01}^{(1)}}{m_{00}^{(0)}}
  \\
  \equivalent\quad
  &m_{11}^{(2)}
  =
  \frac{ m_{10}^{(1)}m_{01}^{(1)}}{m_{00}^{(0)}}
  - \partial_t m_{11}^{(0)}
  - c_s^2\partial_x m_{01}^{(1)}
  - c_s^2\partial_y m_{10}^{(1)}
  .
\end{align}
As a last step before we actually insert anything, we bring those equations in the right form to directly insert them in equation~\eqref{eq: start of momentum equation derivation}.
Therefore, we add equations~\eqref{eq: auxiliaries 6} and~\eqref{eq: auxiliaries 7} and take $-\frac{1}{2}\partial_x$ of the whole equation, resulting in
\begin{align}
  -\partial_x m_{20}^{(2)}
  =
  - \partial_x c_s^2 m_{00}^{(2)}
  &\,
  - \partial_x\frac{1}{2}\frac{ m_{10}^{{(1)}^2} + m_{01}^{{(1)}^2}}{m_{00}^{(0)}}
  + \partial_x\frac{1}{\omega_6}\partial_t c_s^2 m_{00}^{(0)}
  \nonumber
  \\&\,
  + \partial_x\frac{1}{2\omega_6}(1+c_s^2)(\partial_x m_{10}^{(1)} + \partial_y m_{01}^{(1)})
  \nonumber
  \\&\,
  - \partial_x\frac{1}{2}\frac{ m_{10}^{{(1)}^2} - m_{01}^{{(1)}^2}}{m_{00}^{(0)}}
  + \partial_x\frac{1}{2\omega_5} (1 - c_s^2)\partial_x m_{10}^{(1)}
  + \partial_x\frac{1}{2\omega_5} (c_s^2 - 1)\partial_y m_{01}^{(1)}
  \nonumber
  \\
  =
  - \partial_x c_s^2 m_{00}^{(2)}
  &\,
  - \partial_x\frac{ m_{10}^{{(1)}^2} }{m_{00}^{(0)}}
  + \partial_x\left(\frac{1}{2\omega_6}(1+c_s^2)(\partial_x m_{10}^{(1)} + \partial_y m_{01}^{(1)})\right)
  \nonumber
  \\&\,
  + \partial_x\left(\frac{1}{2\omega_5} (1 - c_s^2)\partial_x m_{10}^{(1)}\right)
  + \partial_x\left(\frac{1}{2\omega_5} (c_s^2 - 1)\partial_y m_{01}^{(1)}\right)
  \nonumber
  \\
  \label{eq: auxiliaries 9}
  =
  - c_s^2 \partial_x  m_{00}^{(2)}
  &\,
  - \partial_x\frac{ m_{10}^{{(1)}^2} }{m_{00}^{(0)}}
  \begin{aligned}[t]
    &+ \partial_{xx} m_{10}^{(1)} \left(\frac{1}{2\omega_6}(1+c_s^2) + \frac{1}{2\omega_5} (1 - c_s^2)\right)
    \\&
    + \partial_{xy} m_{01}^{(1)} \left(\frac{1}{2\omega_6}(1+c_s^2) + \frac{1}{2\omega_5} (c_s^2 - 1)\right),
  \end{aligned}
\end{align}
where we used~\eqref{eq: derivative of zeroth order} to get rid of the time derivative.
One notable thing is, that we assumed the first time, that $\omega_5$, $\omega_6$ and $c_s^2$ are constant in space and hence can be put in front of the derivative.
In our isothermal setting, this is just fine, as the $\omega$s will be connected to the viscosity which is a function of temperature; just as $c_s^2$.

The same treatment is applied to~\eqref{eq: auxiliaries 8}, this time multiplied with $-\partial_y$, resulting in
\begin{align}
  \label{eq: auxiliaries 10}
  -\partial_y m_{11}^{(2)}
  =
  -\partial_y \frac{ m_{10}^{(1)}m_{01}^{(1)}}{m_{00}^{(0)}}
  + c_s^2\partial_{xy} m_{01}^{(1)}
  + c_s^2\partial_{yy} m_{10}^{(1)}
  .
\end{align}
