% !TEX root = ../thesis.tex
As our collision is defined in the space of normalized cumulants but our analysis uses moments, we have to link their corresponding expansions.
We start by expanding the first nontrivial normalized cumulant $K_{11}$, defined in~\eqref{eq: K 11 from moments}, the same way we expanded the moments in~\eqref{eq: expansion of m} and explain a few concepts, before diving into all the expansions.
We can reformulate~\eqref{eq: K 11 from moments} like
\begin{equation}
  \label{eq: first expansion}
  \begin{aligned}
    K_{11} & = m_{11} - \frac{m_{10}m_{01}}{m_{00}}\\
    \Leftrightarrow
    \sum_{i=0}^\infty \epsilon^i K_{11}^{(i)}
    & = \sum_{i=0}^\infty \epsilon^i m_{11}^{(i)} -
    \frac{\sum_{i,j=0}^\infty \epsilon^{i+j} m_{10}^{(i)}m_{01}^{(j)}}
        {\sum_{i=0}^\infty \epsilon^i m_{00}^{(i)}}.
  \end{aligned}
\end{equation}
We have to stop here and cope with the denominator first.
In writing
\begin{equation}
  \frac{1}{\sum_{i=0}^\infty \epsilon^i m_{00}^{(i)}}=
  \frac{1}{m_{00}^{(0)}}
  \frac{1}{1 - \sum_{i=1}^\infty \epsilon^i \frac{ - m_{00}^{(i)}}{ m_{00}^{(0)}}}.
\end{equation}

we prepared the denominator to be written as a geometric series.
This is valid as long as we assume $\epsilon$ to be `small enough' to ensure $\abs{\sum_{i=1}^\infty \epsilon^i \frac{ m_{00}^{(i)}}{ m_{00}^{(0)}} } < 1$, which is by no means a bold statement.
Firstly the zeroth order of $m_{00}$ is left out and all other terms are divided by the constant portion of $m_{00}$.
Secondly, our nondimensionalisation in~\eqref{eq: nondimensionalisation} states that $\epsilon = \frac{\Delta x}{L}$, where $L$ is a typical length in our project.
As our typical length scales have to be some orders of magnitude larger than our cell spacing\footnote{Let's say we discretize a 2D wing with a given chord length, we have to take small cells in its vicinity just to represent the shape accurately.}, our prerequesite is even less of a problem.

Thus, we get
\begin{equation}
  \label{eq: definition of g}
  g(\epsilon) \defined \sum_{j=0}^\infty {\left(\sum_{i=1}^\infty \epsilon^i \frac{ - m_{00}^{(i)}}{ m_{00}^{(0)}}\right)}^j = \frac{1}{1 - \sum_{i=1}^\infty \epsilon^i \frac{ - m_{00}^{(i)}}{ m_{00}^{(0)}}}.
\end{equation}
Expanding this term and using the even and odd order properties of equation~\eqref{eq: kill even moments odd order} and following, we get
\begin{equation}
  \label{eq: expand g}
  \begin{aligned}
    g(\epsilon) = & \underbrace{1}_{j=0} - \epsilon \underbrace{ \frac{m_{00}^{(1)}}{m_{00}^{(0)}}}_{j=1,i=1}
    + \epsilon^2 \Bigg(
      \underbrace{\frac{m_{00}^{{(1)}^2}}{m_{00}^{{(0)}^2}}}_{j=2,i=1}
      - \underbrace{\frac{m_{00}^{(2)}}{m_{00}^{(0)}}}_{j=1,i=2} \Bigg) + O(\epsilon^3)
      \\=&
      1 - \epsilon^2 \frac{m_{00}^{(2)}}{m_{00}^{(0)}}  + O(\epsilon^3)
  \end{aligned}
\end{equation}
% 1 - \epsilon^2 \frac{m_{00}^{(2)}}{m_{00}^{(0)}}  + O(\epsilon^3)

Now we finally analyze the cumulant $K_{11}$.
Restarting at equation~\eqref{eq: first expansion} and inserting equations~\eqref{eq: definition of g} and~\eqref{eq: expand g}, we get

\begin{equation}
  \begin{aligned}
    \sum_{i=0}^\infty \epsilon^i K_{11}^{(i)}
    & = \sum_{i=0}^\infty \epsilon^i m_{11}^{(i)} -
    \frac{\sum_{i,j=0}^\infty \epsilon^{i+j} m_{10}^{(i)}m_{01}^{(j)}}
        {\sum_{i=0}^\infty \epsilon^i m_{00}^{(i)}} \\
    \Leftrightarrow
    \sum_{i=0}^\infty \epsilon^i K_{11}^{(i)}
    & = \sum_{i=0}^\infty \epsilon^i m_{11}^{(i)} -
    \frac{\sum_{i,j=0}^\infty \epsilon^{i+j} m_{10}^{(i)}m_{01}^{(j)}}
    {m_{00}^{(0)}}
    g(\epsilon)\\
    \Leftrightarrow
  \sum_{i=0}^\infty \epsilon^i K_{11}^{(i)}
   & = \sum_{i=0}^\infty \epsilon^i m_{11}^{(i)} -
  \frac{\sum_{i,j=0}^\infty \epsilon^{i+j} m_{10}^{(i)}m_{01}^{(j)}}
      {m_{00}^{(0)}}
  \left(1 - \epsilon^2 \frac{m_{00}^{(2)}}{m_{00}^{(0)}}  + O(\epsilon^3)\right).
\end{aligned}
\end{equation}
Collecting all zeroth order terms, we get using~\eqref{eq: zeroth order velocity zero}
\begin{equation}
  K_{11}^{(0)} = m_{11}^{(0)} - \frac{m_{01}^{(0)}m_{10}^{(0)}}{m_{00}^{(0)}} =  m_{11}^{(0)}.
\end{equation}
Analogously, for first order terms,
\begin{equation}
  \begin{aligned}
    K_{11}^{(1)} & = m_{11}^{(1)} - \frac{m_{01}^{(1)}m_{10}^{(0)}+m_{01}^{(0)}m_{10}^{(1)}}{m_{00}^{(0)}}
    \\
    &= m_{11}^{(1)} = 0
  \end{aligned}
\end{equation}
holds.
Thus, the normalized cumulant $K_{11}$ is equal to the corresponding moment at order zero and one.

Going on to the second order terms, we find
\begin{equation}
  \begin{aligned}
    K_{11}^{(2)}
    = &m_{11}^{(2)}
    - \frac{
      m_{10}^{(1)}m_{01}^{(1)}
    + m_{10}^{(0)}m_{01}^{(2)}
    + m_{10}^{(2)}m_{01}^{(0)}
    }{m_{00}^{(0)}}
    + \frac{m_{10}^{(0)}m_{01}^{(0)}}{m_{00}^{(0)}}
    \frac{m_{00}^{(2)}}{{m_{00}^{(0)}}}
    \\ = &
    m_{11}^{(2)}
    - \frac{ m_{10}^{(1)}m_{01}^{(1)}}{m_{00}^{(0)}},
  \end{aligned}
\end{equation}
which depicts a first discrepancy of the corresponding moment.

Due to the fact that the remaining calculations are just tedious but not really difficult, they are relegated to Appendix~\ref{appendix: Expansions of the normalized cumulants}.
As we will simplify the expansions with the information of the next section, the complete list of expansions is also just the appendix, namely in equations~\eqref{eq: expansions of cumulants zeroth order raw} through~\eqref{eq: expansions of cumulants second order raw}.
