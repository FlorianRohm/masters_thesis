% !TEX root = ../thesis.tex
Sources and sinks play an important role in fluid dynamics as soon as we have mixed fluids interacting with each other.
One can for example model a chemical reaction with sinks for the educts and sources for the products.
As we focus on one species, the need for sources and sinks is of little importance and thus will be ignored.
Therefore we need to preserve mass and thus density in the course of the collision which leads to
\begin{equation}
    \rho  = m_{00} = m_{00}^*
\end{equation}
and consequently
\begin{equation}
    \rho^{(p)} = m_{00}^{(p)} = m_{00}^{*(p)}.
\end{equation}
Concerning momentum, it is of course a conserved quantity, but we have `sources' and `sinks' in the sense of forces with gravity being the most prominent example\footnote{The viscous term is often called a traction force, but is in fact `only' a diffusion of momentum, which leads to traction at the walls.}.
Hence, velocity may be altered from before to after the collision, although we did not add this term into our collision an can be thought of like a correction we have to additionally incorporate\footnote{Which we didn't do in our formulation, but is definitely possible.}.
