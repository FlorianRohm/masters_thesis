% !TEX root = ../thesis.tex
Last but not least, we take a look at the collision we formulated using normalized cumulants, c.f.~\eqref{eq: collision final before simplification normalized},
and insert the findings of Section~\ref{sub: Expanding the normalized cumulants}.

At zeroth order, only $K_{20}$ and $K_{02}$ play any role, c.f.~\eqref{eq: expansions of cumulants zeroth order}.
When inserting their expansion into the related collisions, we get
\begin{equation}
  \begin{aligned}
    &&K_{20}^{*^{(0)}} - K_{02}^{*^{(0)}}
      & = (1-\omega_5) (K_{20}^{(0)} - K_{02}^{(0)}) \\
    &\equivalent& c_s^2 m_{00}^{*^{(0)}} - c_s^2 m_{00}^{*^{(0)}}
        & = (1-\omega_5) (c_s^2 m_{00}^{*^{(0)}} - c_s^2 m_{00}^{*^{(0)}})
  \end{aligned}
\end{equation}
and
\begin{equation}
  \begin{aligned}
    &&K_{20}^{*^{(0)}} + K_{02}^{*^{(0)}}
      & = (1-\omega_6)(K_{20}^{(0)} + K_{02}^{(0)}) + \omega_6 \left( 2 c_s^2 m_{00}^{(0)} \right) \\
    &\equivalent& c_s^2 m_{00}^{*^{(0)}} + c_s^2 m_{00}^{*^{(0)}}
      & = (1-\omega_6)(c_s^2 m_{00}^{*^{(0)}} + c_s^2 m_{00}^{*^{(0)}}) + \omega_6 \left( 2 c_s^2 m_{00}^{(0)} \right) \\
  \end{aligned}
\end{equation}
and thus not gain any new information.

At first order, all non conserved cumulants are zero and also $m_{00}^{(1)}=0$, hence nothing new can be learned from~\eqref{eq: collision final before simplification} at this order.

The most important part will be the collision of the second order terms.
Here, we insert~\eqref{eq: expansions of cumulants second order} into~\eqref{eq: collision final before simplification normalized} to get
\begin{align}
  \label{eq: collide moments 11_2 raw}
  m_{11}^{*^{(2)}} - \frac{ m_{10}^{*^{(1)}}m_{01}^{*^{(1)}}}{m_{00}^{*^{(0)}}} =&\ (1-\omega_4)\left(m_{11}^{(2)} - \frac{ m_{10}^{(1)}m_{01}^{(1)}}{m_{00}^{(0)}}\right)
  \\
  \label{eq: collide moments 20m02_2 raw}
  m_{20}^{*^{(2)}}-m_{02}^{*^{(2)}} - \frac{ m_{10}^{*^{{(1)}^2}} - m_{01}^{*^{{(1)}^2}}}{m_{00}^{*^{(0)}}} =&\ (1-\omega_5) \left(m_{20}^{(2)}-m_{02}^{(2)} - \frac{ m_{10}^{{(1)}^2} - m_{01}^{{(1)}^2}}{m_{00}^{(0)}}\right)
  \\
  \label{eq: collide moments 20p02_2 raw}
  m_{20}^{*^{(2)}}+m_{02}^{*^{(2)}} - \frac{ m_{10}^{*^{{(1)}^2}} + m_{01}^{*^{{(1)}^2}}}{m_{00}^{*^{(0)}}}
  =&\ (1-\omega_6)\left(  m_{20}^{(2)}+m_{02}^{(2)} - \frac{ m_{10}^{{(1)}^2} + m_{01}^{{(1)}^2}}{m_{00}^{(0)}}\right)
  \\\nonumber&\ + 2\omega_6 c_s^2 m_{00}^{(2)}
  \\
  \label{eq: collide moments 22_2 raw}
  m_{22}^{*^{(2)}}
  + c_s^2\left(c_s^2 m_{00}^{*^{(2)}}
  - m_{02}^{*^{(2)}}
  - m_{20}^{*^{(2)}} \right)
  =&\ (1-\omega_9)m_{22}^{(2)}
  + c_s^2\left(c_s^2 m_{00}^{(2)}
  - m_{02}^{(2)}
  - m_{20}^{(2)}    \right).
\end{align}
The first three equations can be simplified by using~\eqref{eq: first order invariant}, resulting in
\begin{align}
  \label{eq: collide moments 11_2}
  m_{11}^{*^{(2)}}  =&\ (1-\omega_4)m_{11}^{(2)} + \omega_4 \frac{ m_{10}^{(1)}m_{01}^{(1)}}{m_{00}^{(0)}}
  \\
  \label{eq: collide moments 20m02_2}
  m_{20}^{*^{(2)}}-m_{02}^{*^{(2)}} =&\ (1-\omega_5) \left(m_{20}^{(2)}-m_{02}^{(2)}\right) + \omega_5 \frac{ m_{10}^{{(1)}^2} - m_{01}^{{(1)}^2}}{m_{00}^{(0)}}
  \\
  \label{eq: collide moments 20p02_2}
  m_{20}^{*^{(2)}}+m_{02}^{*^{(2)}}
  =&\ (1-\omega_6)\left(  m_{20}^{(2)} + m_{02}^{(2)}\right)
  + \omega_6 \left( \frac{ m_{10}^{{(1)}^2} + m_{01}^{{(1)}^2}}{m_{00}^{(0)}}
  + 2 c_s^2 m_{00}^{(2)} \right).
\end{align}


This finally completes all the preliminary work needed and we can start with the analysis in the next section.
